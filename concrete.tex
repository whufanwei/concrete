\documentclass[12pt,a4paper]{book}
\usepackage{amsmath}
\usepackage{ntheorem}
\usepackage{thmtools}
\declaretheoremstyle[
spaceabove=6pt, spacebelow=6pt,
headfont=\normalfont\bfseries,
% notefont=\mdseries, notebraces={(}{)},
bodyfont=\normalfont,
postheadspace=1em,
numberwithin=chapter
]{exstyle}
\declaretheorem[style=exstyle]{example}
\usepackage{amsfonts}
\usepackage{mathspec}
\usepackage{xltxtra,xunicode}
\usepackage[CJKnumber,CJKchecksingle,BoldFont]{xeCJK}
\usepackage[utf8]{inputenc}
\usepackage{mathpazo}
\usepackage[OT1]{eulervm}
\renewcommand{\mathbf}{\mathbold}
\defaultfontfeatures{Mapping=tex-text}
\setmainfont[Mapping=tex-text]{URW Palladio L}
\setsansfont[Mapping=tex-text]{URW Palladio L}
\setmonofont{Consolas}
\setCJKmainfont[BoldFont={SimHei},ItalicFont={KaiTi}]{SimSun}
\setCJKsansfont{SimHei}
\setCJKmonofont{FangSong}
\linespread{1.3}
\setlength{\parindent}{2.4em} 
\usepackage{graphicx}
\usepackage{geometry}
\usepackage[shortlabels]{enumitem}
\setlist[enumerate]{labelindent=\parindent,leftmargin=*,itemsep=-4pt,topsep=0mm}
\usepackage{siunitx}
\sisetup{math-rm=\mathrm,allow-number-unit-breaks}
\usepackage{tikz}
\usetikzlibrary{arrows}
% \usepackage{minted}
\geometry{a4paper, textwidth=6.5in, textheight=10in,marginparsep=7pt,marginparwidth=.6in}
\usepackage[xetex,bookmarksnumbered=true,bookmarksopen=true,colorlinks=true,citecolor=blue,linkcolor=blue,anchorcolor=green,urlcolor=magenta,breaklinks=true,CJKbookmarks=true,]{hyperref}
\usepackage{cleveref}
\crefname{equation}{公式}{公式}
\crefname{section}{节}{节}

\title{}
\author{R.PARK}
\date{\today}
\title{钢筋混凝土结构}
\hypersetup{
  pdfkeywords={},
  pdfsubject={},
  pdfcreator={Emacs 24.2.1 (Org mode 8.0.2)}}
\begin{document}

\maketitle

\chapter*{序言}

我们希望在这本书中为钢筋混凝土结构这门学科所选择的内容和论述方式能够对结构工程界
的学生、教师以及实际工作者具有吸引力。

这本书的前身是坎特伯雷大学为在新西兰从事结构工程实际工作的工程师举办的进修讲座所
印行的《钢筋混凝土极限强度设计》(第一册)的前后两版讲义。在内容上对这两个版本做了
相当大的扩充与更新。我们多年来在理论与设计的教学方面以及在设计和研究方面积累的经
验有助于为本书形成思路和准备背景材料。

这本教材突出了钢筋混凝土构件和结构的性能,特别是它们直至极限荷载为止的强度与变形
特征,并力求为读者提供有关钢筋混凝土的基本知识。这样的基础知识对于全面恰当的理解
规范及设计方法是必不可少的。结构工程师也许对这本教材没有把一系列图标和实例包括进
来而感到失望。可是这类材料是随处可见的。这本教材的主要目的就在于使读者能从根本上
理解这类实用资料的背景。

现行的美国混凝土学会建筑规范(ACI318-71)是应用最广泛的规范之一。它已被一些国家所采
用,并对另一些国家的规范产生了强有力的影响。为此,本书在很多地方提到了ACI规范的条
款,但在必要处也和其他规范做对比。虽然如此,本书却并不是以仅靠规范为准的。它的重
点实在为何作出某种工程决策,而非如何执行这些决策上面。\textbf{我们的看法是,工程
  师应该有能力对设计方法作出合理的评价,而不应该盲从规范条款。}

本书突出用于强度设计及使用性能设计的计算方法,我们认为这是最有现实意义的方法。

本书以讨论基本设计准则和混凝土及钢材的性能为起点,然后介绍了受弯、受弯兼受轴向力、
受剪及受扭钢筋混凝土结构构件的强度与变形。接着是粘结与锚固。随后又以对变形和裂缝
的控制为重点,探讨了钢筋混凝土构件在使用荷载下的性能。之后是对框架及剪力墙的论述。
由于我们认为仅仅恰当的选用各部件的截面还不足以保证一个成功的设计,因而本书以结构
部件和节点构造结尾。

我们没有试图讨论特种结构的设计问题。因为对钢筋混凝土构件受力性能及结构分析的透彻
理解应使设计人员有能力队常用范围的结构进行设计,并且找得到解决特殊问题的办法。

这本书有别于其他大多数钢筋混凝土教材的一个方面是它关于地震荷载的影响和关于完成抗
震结构设计过程的方法论述。由于认识到地震区可能比在此以前假定的范围更大,抗震设计
就显得尤为重要。抗震设计包含的内容并不仅在于结构考虑附加静力水平荷载的作用。如果
想把结构设计成能够经受住较大的地震,还必须恰当地注意细部构造问题和理解可能出现的
破坏机理。在强烈地震荷载下的受力性能方面所考虑的问题应包括对构件和结构在非弹性范
围内的变形特征以及对强度发挥过程的理解,而这些领域在本书中都得到了适当的重视。

由于正在准备专门出一本书来论述关于板的问题,这里就省略了关于这类构件的详细讨论。

我们希望这本书能作为一本有用的教材为准备大学本科钢筋混凝土课程教学大纲的教师服务。
这本书对每一个主要课题的论述都具有足够的深度,使它有可能用于研究生的高等钢筋混凝
土课程。我们还希望这本书对于很多从事实际工作的工程师,特别是对于面临设计抗震结构
这一困难任务的工程师将是一本有用的参考书。

\tableofcontents

\chapter{设计方法}

\section{允许应力设计法和极限强度设计法的发展}

一些早期的钢筋混凝土构件的研究工作是以极限强度理论为基础的。例如1897年瑟
利(Thullie)提出的弯曲理论和1899年里特尔(Ritter)提出的抛物线形应力分布理论。但是到
了1900年前后科瓦涅(Coignet)和特德斯科(Tedesco)的直线(弹性)理论却得到了公认。这主
要是因为当时弹性理论已经成为其他结构的通用设计方法,同时也因为考虑到直线应力分布
可以在数学上带来简化。此外,当时的一些试验也已表明,只要仔细地选择容许的资用应力,
采用弹性理论也能使结构在使用荷载下表现出令人满意的性能,而且对破坏具有足够的安全
储备。因此,在很长一段时间里,弹性理论一直使钢筋混凝土设计的依据。

近年来,已极限强度理论作为设计依据的问题又重新引起了人们的兴趣。在经过半个世纪的
积累实践经验和进行实验室试验之后,人们对钢筋混凝土性能的了解大大加深了,而且弹性
理论(资用应力)设计方法的缺点也变得更加明显了。这就导致了对资用应力设计方法进行周
期性的调整,而设计方法应以混凝土和钢材的实际非弹性性质为基础的观点也就变得更加明
确了。因此,在1956年美国混凝土学会(ACI)的钢筋混凝土规范和已1957年英国规范均已承认
极限强度设计可以作为资用应力设计以外的另一种设计方法。这里提到的两种设计方法可以
概述如下:

\subsection{资用应力设计(弹性理论)}

这里的结构构件的截面是在假定应力——应变之间为线性关系和保证使用荷载下的钢筋和混凝
土中的应力都不超过容许的资用应力的情况下进行设计的。材料的资用应力是取它的极限强
度或者屈服强度的某个固定的百分比值;例如,对于弯曲受压可以假定为混凝土圆柱体强度
的0.45倍。而作用在超静定结构中的弯矩和力则是按照线弹性性能的假定计算出来的。

\subsection{极限强度设计}

这里结构构件的截面是按极限荷载作用于结构时,在考虑材料的非弹性应变的情况下使它达
到极限(最大)强大(即混凝土达到最大强度而钢筋通常达到屈服)的原则进行设计的。而这一
极限荷载值是各项使用荷载乘以各自的荷载系数的总和。实际使用的典型的荷载系数对于恒
荷载为1.4,对于活荷载为1.7。在极限荷载下作用于超静定结构中的弯矩和力是假定结构直
到极限荷载为止均处于弹性状态来进行计算的。弯矩和力也可以采用另一种方法,即在某种
程度上考虑由于荷载很大时构件中力和变形之间存在的非线性关系而产生的各种作用的重分
布来进行计算。

倾向于采用极限强度设计的这种趋势有以下几个原因:

1.钢筋混凝土截面在荷载较大时具有非弹性性质,弹性理论由于没有考虑非弹性应变而不能
可靠地预测构件的极限强度。因此,对于按资用应力法设计的结构,确切的荷载系数(极限荷
载/使用荷载)是未知的,而且是因结构而异的。

2.极限强度设计可以使荷载系数的选用更加合理。例如,对于比较确切的荷载,如恒荷载,
可以采用较小的荷载系数,而对于不太有把握的荷载,如活荷载,就采用较大的荷载系数。

3.混凝土的应力——应变曲线是非线性的,而且与时间有关。例如,在持续不变的应力下,混
凝土的徐变应变可能达到初始弹性应变的数倍。因此,在资用应力设计中所用的模量比(钢筋
的弹性模量与混凝土的弹性模量之比)值只是一个粗略的近似值。徐变应变能在钢筋混凝土截
面中引起相当可观的应力重分布,这意味着在使用荷载下实际存在的应力与设计应力之间往
往只有很少的联系。例如,在使用荷载持续作用期间,柱子的受压钢筋可能达到屈服强度,
尽管在正常的模量比建议值按资用应力法进行分析时这种现象并不明显。而极限强度设计却
并不需要直到模量比。

4.非弹性应变可以是应力形成一个效率较高的分布状态。极限强度设计能够利用由此产生的
强度储备,而且有时它还表明资用应力是比较保守的。例如,双筋梁的受压钢筋在极限荷载
下通常都达到屈服强度,但弹性理论却可能表明受压钢筋中的应力还比较低。

5.极限强度设计能更有效地利用高强度钢筋,而且可以使高度较小的梁不设受压钢筋也能使
用。

6.极限强度设计能使设计者估计出结构在弹性后范围内的延性。这在设计者按重力荷载进行
设计及按地震或爆炸荷载进行设计是考虑可能发生的弯矩重分布时是一个重要的方面。

\section{强度及使用性能设计}

钢筋混凝土的最理想的设计方法应该是综合极限强度设计和资用应力设计的最优特点。这一
点近来已经被人们认识清楚了。这是合乎需要的,因为如果只按极限强度要求来确定断面,
即使荷载系数取值满足强度设计要求,但使用荷载下的裂缝和挠度仍有可能过大的危险。当
钢筋应力偏高或布置不当时,裂缝便可能过大。而当采用按极限强度设计可能通得过的,高
度较小而且应力较高时,挠度便可能起控制作用。因此,为了保证设计让人满意,就必须验
算使用荷载下的挠度和裂缝宽度,以确保其值处于结构功能所要求的合理界限以内。这种验
算需要应用弹性理论。

1964年欧洲混凝土委员会提出了钢筋混凝土国际实用规范的建议。这个文献引入了极限设计
状态的概念,建议按照几个极限状态来进行结构设计。最重要的极限状态是:极限荷载下的
强度、使用荷载下的挠度及使用荷载下的裂缝宽度。这个方法正在得到许多国家的肯定。因
此,极限强度理论目前正在成为选择截面的主要方法,而弹性理论只是用来保证使用性能。
还值得注意的是,在苏联和其他一些欧洲国家,多年来就已经在利用极限强度理论来选择截
面了。看来采用极限强度设计的趋势可能继续下去,预计要不了多少年就可能效法欧洲混凝
土委员会的榜样,而资用应力法就会从钢筋混凝土规范中消失。

在1956年和1971年的美国混凝土学会(ACI)规范中,既允许采用资用应力设计,也允许采用极
限强度设计。1971年的ACI规范则突出了以极限强度为基础辅以使用性能验算的设计方法。不
过,1971年的规范同时也允许选用另一种设计方法,即以资用应力法设计受弯的梁,而用乘
以系数降低了极限强度的公式来设计所有处于其他作用下的构件。显然,把这种同时也可以
选用的方法保留下来只是一种试图把传统设计方法保持下来的努力。将来ACI规范可能完全取
消这种可供选用的方法。1971年ACI规范中的术语变化也是值得注意的。例如,“极限”这词
很少出现,而是把“极限强度”写成“强度”。

本书系采用1971年ACI规范考虑强度和使用性能的方法,因为看来它突出了钢筋混凝土的真实
性能,是一种更加合乎逻辑的设计方法。只要有可能,就在书中写出ACI条款的背景情况。凡
是有必要的地方都能根据能够得到的最新的研究资料对规范条款做了补充,并给出与其他规
范的对比结果。

\section{ACI的强度及使用性能设计方法}

\subsection{强度条款}
\label{sec:aci}

1971年的ACI规范把保证结构安全的强度条款分为两部分,即荷载系数和承载能力降低系数。

\subsubsection{荷载系数}

荷载系数是用来保证在使用荷载增加到超过设计规定的荷载时仍具有足够的安全性,使破坏
发生的可能性非常小。荷载系数也有助于保证使用荷载下的变形不致过大。用于恒荷载、活
荷载、土和液体的侧压力以及风和地震荷载的荷载取值是不相同的。对于不同类型的荷载其
荷载系数之所以不同,是因为例如与所规定的活荷载相比,一个结构的恒荷载就不容易被超
过。结构的极限荷载至少应等于乘以各自荷载系数后的各项使用荷载的总和。1971年的ACI规
范建议,为了承受恒荷载$D$和活荷载$L$,所需提供的强度$U$至少要等于
\begin{equation}
  \label{eq:1} U=1.4D+1.7L
\end{equation}当在设计中必须考虑风荷载$W$时,则所需提供的强度$U$应至少等于
\begin{equation}
  \label{eq:2} U=0.75\times(1.4D+1.7L+1.7W)
\end{equation}在这里,活荷载$L$是否参与组合的两种情况都要验算。而当$D$和$W$所引起
的作用符号相反时,还应满足
\begin{equation}
  \label{eq:3} U=0.9D+1.3W
\end{equation}如果要计入地震荷载$E$,则应遵循\cref{eq:2}和\cref{eq:3},只不过要用
$1.1E$来替代其中的$W$。其他荷载类型的强度要求已在规范中给出。

这里规定的荷载系数没有根据破坏后果的严重程度而变化。例如,人们可能期望医院建筑的
荷载系数比工业建筑的大一些。然而在规定的使用荷载中已经包括了破坏严重性的影响。尽
管如此,还是应该把规范规定的系数看作是最小值。当破坏后果特别严重或者不能合理估计
出使用荷载的大小时,把荷载系数增大一些可能也是合适的。

\subsubsection{承载能力降低系数}

承载能力降低系数$\varphi$的规定是为了考虑计算方法的近似性以及材料强度、工作质量和
尺寸的变异性。这当中的每个因素都可能是在容许界限之内,但它们组合起来就可能导致承
载能力的不足。如果用于截面的基本强度公式在科学上是正确的,材料的强度与规定值相同,
截面尺寸与图纸所示无异,那么这可以认为这个基本强度公式将给出截面的理想强度。在设
计计算中要采用的截面可靠强度或置信强度就等于理想强度乘以承载能力降低系数
$\varphi$,而$\varphi$的数值则取决于若干个变量的重要性。1971年的ACI规范建议的
$\varphi$值为:

弯曲(无论是否有轴向拉力),轴心受拉:$\varphi=0.9$

弯曲(同时作用轴压),轴心受压:

当配置了螺旋筋时,$\varphi=0.75$

其他情况下,$\varphi=0.70$

剪切和扭转:$\varphi=0.85$

在规定承载能力降低系数时所考虑的另外一些变量中还包括构件破坏后果对于整个结构的严
重性,以及在破坏模式含有的事先警告程度。梁所取的$\varphi$值最高,因为它总是设计成
以受拉钢筋屈服的延性方式破坏,而这种破坏一般都会通过可观察的裂缝和较大的挠度事先
发出警告;而且还因为钢筋强度的变异性比混凝土小,梁的抗弯强度就能比较准确地加以预
测。柱子取的$\varphi$值最低,这是因为如果混凝土强度是起控制作用的因素,柱子就会以
脆性方式破坏。此外,柱子的破坏还可能意味着整个结构的倒塌,而且柱子的修复又很难进
行。螺旋配筋的柱子比箍筋柱好,因而就给它规定了一个较大的$\varphi$值。而用于受剪和
受钮的$\varphi$是个中间值,这是因为混凝土对构件强度所起的作用不如在受压构件中那样
突出,而预测这类构件强度的理论却又不如受弯构件那样准确。

在设计中是根据可靠强度来计算极限荷载的。一个承受恒荷载及活荷载的结构,当以理想强
度为基础时,它的总安全系数是
\begin{equation}
  \label{eq:4} \frac{1.4D+1.7L}{D+L}\frac{1}{\varphi}
\end{equation}据此,在弯曲(无论是否有轴向拉力)的情况下,相对于截面理想强度的总安
全系数就从$L/D=0$时的1.56上升到$L/D=4$时的1.82。其中较大的值适用于活荷载较大的情
况。对于同时承受弯曲和轴压作用的构件,当$L/D$从0增长到4时,总安全系数在2.0到2.34
之间变化。这样对于关键的建筑构件,赋予较大的总安全系数。

理想强度是用规范规定的混凝土和钢筋的强度来计算的。由于在实际结构中这些强度一般都
超过了规定值,因此就存在着附加的强度储备。

\subsection{使用性能条款}

如果构件的截面是根据所需要的强度来选定的,那么,估算这个结构在使用荷载下的性能就
是一个极为重要的问题。这是因为截面较小的构件和受压钢筋少的截面虽然能够满足强度要
求,但是在使用荷载下却会导致过大的变形及应力。因此,必须验证使用荷载下的变形尚在
容许的界限之内。为了外观和耐久性的需要,控制开裂也是很重要的。因此在使用荷载裂缝
宽度也不应超过规定的界限。规定挠度和裂缝宽度的容许界限是比较困难的,但是1971年的
ACI规范还是为此给出了建议值。

\subsection{延性条款}

除强度及使用性能之外,还必须考虑延性。重要的是要保证结构即使在极端情况下被加载到
破坏时,它也将具有延性性能。这意味着结构不以无警告的脆性方式破坏,但是在接近最大
承载能力时将产生很大的变形。接近最大荷载时的大变形能够发出充分的破坏警告,而且通
过保持承载能力可以预防整体倒塌和挽救生命。此外,构件的延性还使我们能够在设计中使
用那种考虑了弯矩重分布的弯矩分布图。

在需要按地震荷载进行设计的地区,延性是一个极为重要的问题。因为现行地震规范的基本
原则是用来设计只能在弹性范围内承担中等地震的结构。在强烈地震作用下,则是依靠屈服
后可供利用的足够延性使得结构能够保存下来而不致倒塌。所以,只要当结构在经受几个完
全进入屈服范围的加载循环时具有足够的延性,从而在弹性后的变形过程中吸收和耗散能量
时,才能证明对地震荷载所做的建议是有道理的。

为了保证延性性能,设计者应对诸如纵向钢筋的含量、钢筋的锚固和对受压混凝土的约束等
细部构造给予特别注意。以确保避免一切脆性破坏(例如剪切破坏)。1971年的ACI规范对能形
成延性截面的纵向钢筋含量提出了建议,并允许弯矩在弹性弯矩图的基础上进行适当的重分
布。这个规范还第一次包括了一个对抗震设计作出专门规定的附录。

\section{对构件强度的考虑}

\subsection{构件强度的发挥}

在设计中常常需要估计结构部件可能具有的强度上限和下限。要在一个加载到破坏的结构中
保证各个构件达到强度的先后符合所期望的顺序,就属于这种情况。例如,在一个连续框架
的梁柱节点处,若想避免柱子破坏及其可能带来的灾难性后果,就总是希望梁比柱子先达到
强度。抗震设计的一个特点就是避免出现各种模式的非延性破坏。因此,重要的是要了解结
构构件的强度可能发生的变化。

真实的结构在钢筋和混凝土的强度方面都有偏离规定值的情况,而且还有施工引起的不可避
免的对规定尺寸的偏差。此外,在推导强度公式时还采取了一些假定。因此要确切地计算结
构的真实强度是困难的;但却有可能给出各种构件的具有的不同强度的定义,以便在不同类
型的设计计算中使用。下文将给出理想强度、可靠强度、可期强度和超常强度的定义。

\subsection{理想强度$S_i$}构件截面的理想强度或名义强度$S_i$是根据预测这个截面的破
坏性能的理论并采用的截面几何尺寸及规定的材料强度为基础求出来的。本书的主要篇幅就
是从事理想强度的推导,而其他强度可以很方便地和它联系起来。

\subsection{可靠强度$S_d$}

在\cref{sec:aci}所述的强度条款中已经概述了承载能力降低系数$\varphi$的意义。它使可
靠强度或置信强度$S_d$与理想强度$S_i$的关系可以表示为:
\begin{equation}
  \label{eq:5} S_d=\varphi S_i
\end{equation}式中$\varphi$为承载能力降低系数,其值小于1。

\subsection{可期强度$S_p$}

可期强度$S_p$考虑了材料强度一般都大于它的规定强度的这个事实。例如,钢筋的屈服强度
可能比它的规定值高20\%;混凝土的强度可能比它的规定值高30\%,而且当龄期较长或三向
受压时甚至更高一些。通过一般在结构施工期间的常规试验可以得到材料的可期强度。如果
在设计阶段就需要这种资料,那就必须以在这之前对这些材料的经验为依据。通过下式可以
表示出可期强度与理想强度的关系:
\begin{equation}
  \label{eq:6} S_p=\varphi_p S_i
\end{equation}式中$\varphi_p$为考虑材料强度高于规定值的可期强度系数,其值大于1。

\subsection{超常强度$S_o$}
\label{sec:s_o}

超常强度$S_o$考虑了所用可能引起强度增大的因素。这些因素包括:钢筋的强度高于规定的
屈服强度,再加上变形较大时由应变硬化提供的附加强度;混凝土的强度高于规定的强度;
截面尺寸大于假定的尺寸;由侧向约束在受弯构件中引起的轴向压力以及为构造设置的或在
计算中未予考虑的附加钢筋。超常强度与理想强度之间的关系可以由下式表示:
\begin{equation}
  \label{eq:7} S_o=\varphi_o S_i
\end{equation}式中$\varphi_o$为考虑材料强度高于规定值的可期强度系数,其值大于1。

\subsection{不同强度之间的关系}

保证部件$A$承受来自部件$B$的荷载而不在部件$B$的强度被充分发挥出来之前发生破坏的最
高保护水平是当的可靠强度超过部件的超常强度,即$S_{dA}\ge S_{oB}$时达到的。由
$S_{iA}\ge S_{oB}$给出的是一个低一些的保护水平,而由$S_{pA}\ge S_{oB}$给出的则是
一个更低的保护水平。用两个部件可期强度的比值$S_{pA}/S_{pB}$,可以最好的表达出这些
情况所提供的保护程度。上述几种由高到低的保护水平可以由\cref{eq:5}和\cref{eq:7}求
出:
\begin{subequations}
  \begin{align} \frac{S_{dA}}{S_{oB}} &= \frac{\varphi_A
S_{pA}/\varphi_{pA}}{\varphi_{oB} S_{pB}/\varphi_{pB}}\ge 1&
\Rightarrow\frac{S_{pA}}{S_{pB}} &\ge \frac{\varphi_{pA}
\varphi_{oB}}{\varphi_{pB} \varphi_{A}}\label{eq:a}\\ 
\frac{S_{iA}}{S_{oB}} &=
\frac{S_{pA}/\varphi_{pA}}{\varphi_{oB} S_{pB}/\varphi_{pB}}\ge 1 &
\Rightarrow\frac{S_{pA}}{S_{pB}} &\ge \frac{\varphi_{pA}
\varphi_{oB}}{\varphi_{pB}}\label{eq:b}\\ 
\frac{S_{pA}}{S_{oB}} &=
\frac{S_{pA}}{\varphi_{oB} S_{pB}/\varphi_{pB}}\ge 1 &
\Rightarrow\frac{S_{pA}}{S_{pB}} &\ge
\frac{\varphi_{oB}}{\varphi_{pB}}\label{eq:c}
  \end{align}
\end{subequations}

例如,假若$\varphi_A=0.9$,$\varphi_{pA}=\varphi_{pB}=1.1$,而$\varphi_{oB}=0.9$,
则为保证部件$A$不致破坏,所需要的按\cref{eq:a}、\cref{eq:b}和\cref{eq:c}求得的部
件$A$的可期强度与部件$B$的可期强度比值分别为$(1.1\times1.3)/(1.1\times0.9)=1.44$、
$(1.1\times1.3)/1.1=1.3$和$1.3/1.1=1.18$,这就表示了部件$A$的不同的保护水平。

\chapter{混凝土与钢筋的应力——应变关系}

\section{混凝土} \sisetup{math-rm=\mathrm,per-mode = symbol}

\subsection{单轴受力性能}

在实际工程中混凝土很少只沿一个方向受力(单轴受力),因为在大多数结构部位中它都是沿
几个方向同时受力的。尽管如此,在很多情况下假定它处于单轴受力状态可认为是合理的。

\subsubsection{受压性能}

混凝土的抗压强度一般是用高度与直径之比为2的圆柱体求得的。对圆柱体是沿纵向以缓慢的
应变速度加荷,使它在2到3分钟内达到最大应力。标准圆柱体的高度是
\SI{12}{in}(\SI{305}{\milli\meter}),直径是\SI{6}{in}(\SI{152}{\milli\meter})。它
28天时达到的抗压强度通常是在2000到\SI{8000}{psi}(13.8到
\SI{55.2}{\newton\per\square\milli\meter})之间。小尺寸的圆柱体或立方体也在应用,
特别是用在生产控制方面。这些试件的抗压强度要高一些,但借助于由试验求得的换算系数,
就可以把由这些试件得到的结果换算成标准圆柱体强度值。

表示由单轴受压的混凝土圆柱体在持续几分钟的试验中得出的典型的应力——应变曲线。这些
曲线直到大约为抗压强度的一半时几乎是线性的。高强混凝土曲线的顶部相对较尖,而低强
度混凝土曲线的顶部则比较平缓。最大应力处的应变约为0.002。在达到了最大应力之后的更
大应变时,即使在试件内已经形成了与加载方向平行的可见裂缝,混凝土仍然能承受应力。
在柔性试验机上试验的混凝土有时会发生爆发性的破坏,这是因为在最大应力之后,当荷载
下降时混凝土不能吸收试验机释放的应变能。要把应力——应变曲线下降段的整个范围探测出
来,就需要刚性试验机。

混凝土的弹性模量$E_c$可以取为:
\begin{equation}
  \label{eq:8} E_c=w^{1.5}\:33\sqrt{f'_c}\ \mathrm{psi}
\end{equation} (\SI{1}{psi}=\SI{0.00689}{\newton\per\square\milli\meter}),其中
$w$为混凝土的容重,单位是
\SI{1}{lb/ft^3}(\SI{1}{lb/ft^3}=\SI{16.02}{\kg\per\cubic\meter}),$f'_c$为圆柱体
的抗压强度,单位是psi。\cref{eq:8}是Pauw根据短期荷载试验求得的,它适用于90到
\SI{155}{lb/ft^3}之间的$w$值。它给出的是当应力接近$0.5f'_c$时的割线模量。对一般容
重的混凝土可以认为$E_c$为$5700\sqrt{f'_c}$\ psi或$4730\sqrt{f'_c}$\
\si{\newton\per\square\milli\meter}。

Rusch的试验表明,混凝土在最大应力之前,其应力应变关系与其强度有关。然而,在最大应
力之前广泛采用的应力——应变曲线是二次抛物线。例如,经常引用的Hosgnestad的应力——应
变曲线如所示。其中$f''_c$是混凝土所达到的最大应力。这里所采用的下降段的延伸长度要
取决于假定的能够利用的混凝土极限应变值。这个问题还要在第三章和第六章关于构件抗弯
强度和极限变形的计算中进一步讨论。受弯构件混凝土所达到的最大压应力$f''_c$可能与圆
柱体强度$f'_c$不同,这是因为混凝土受压区的形状和尺寸不同。在第三章中还要用较长的
篇幅来讨论受弯构件中混凝土的强度问题。

\begin{figure}[htbp] \centering
  \begin{tikzpicture}[line width=1pt,>=angle 90,scale=1.4] 
\draw [->] (0,0)--(8.5,0) node [blue,below] {应变/$\epsilon_c$}; 
\draw [->] (0,0)--(0,5) node[blue,left] {应力/$f_c$}; 
\draw [red] (0,0) parabola[bend at end] (4,4); 
\draw [red] (4,4)--(7,3.5); 
\draw [gray,dashed] (0,4) node[blue,left]{$f''_c$}--(7.8,4); 
\draw [gray,dashed] (4,4)--(4,0) node[blue,below]{$\epsilon_o=2f''_c/E_c$}; 
\draw [gray,dashed] (7,3.5)--(7,0) node[blue,below]{0.0038}; 
\draw [<->] (1,0) arc (0:64:1cm);
\draw [thin] (7.2,3.5)--(7.7,3.5); 
\draw [<->,semithick] (7.6,4)--(7.6,3.5);
\draw (3.3,2) node[blue,fill=white] {$f_c=f''_c
      \left[\frac{2\epsilon_c}{\epsilon_o}-\left(\frac{\epsilon_c}{\epsilon_o}\right)^2\right]$};
\draw (1.7,0.6) node[blue] {$E_c=\tan \alpha$}; 
\draw (0.5,0.3) node[blue] {$\alpha$};
\draw (7.8,3.8) node[blue,right] {$0.15f''_c$};
\draw [->] (5.5,3.3) node[blue,left] {直线} parabola (6.0,3.65); 
\draw [->] (1.8,2) parabola (1.5,2.4);
  \end{tikzpicture}
  \caption{混凝土单轴受压时的理想化应力——应变曲线}
  \label{fig:1}
\end{figure}

当以较快的应变速度施加荷载时,混凝土的弹性模量和强度都将增大。例如,已有报告指出,
当应变速度为0.01/sec时,混凝土的强度可能提高17\%。

\subsubsection{受拉性能}

混凝土的抗拉强度一般低于抗压强度的20\%。它可以直接用受拉试件测得。然而,由于夹持
试件实现轴向受拉的困难以及由夹持装置所引起的次应力的不确定性,这种直接拉力试验即
使为了研究目的也是很少采用的。

混凝土的抗拉强度也可以根据水平放置在试验机上并沿直径加载的圆柱体在劈裂时所算得的
拉应力来间接测定。现将这一试验方法和根据弹性理论求得的沿加载直径引起的应力示于。
劈裂时横过直径的拉应力可以按关系式$2P/(\pi hd)$求得,其中$P$为劈裂时作用的荷
载,$h$为圆柱体强度,$d$ 为圆柱体直径。

混凝土的抗拉强度也可以用素混凝土梁通过弯曲试验来估量。梁的截面一般
为\SI{6}{in}(\SI{150}{\milli\meter})见方。弯曲抗拉强度,也就是熟知的抗折模
量$f_r$ ,是由弯曲公式$M/Z$ 算得的。式中$M$ 为试件的破坏弯矩,$Z$ 为截面模量。圆
柱体劈裂抗拉强度通常是在抗折模量的50\%到75\%之间变化。其差别主要是由于临近破坏时
受弯构件混凝土中的应力分布是非线性的。用于抗折模量的一个近似关系式为
\begin{equation}
  \label{eq:32}
  f_r=K\sqrt{f'_c} \quad \mathrm{psi}
\end{equation}
式中$f'_c$为圆柱体抗压强度,单位为psi。砂——卵石混凝土的$K$值在7到13之间变化。通
常是把$K$的下限值取为$K=7.5$。显然,抗折模量并不随抗压强度的增长而按比例增长。

由于混凝土的抗拉强度低,所以在钢筋混凝土构件强度计算中通常不考虑受拉的混凝土。如
果要考虑,则受拉应力——应变曲线可以理想化为直至抗拉强度为止的一条直线。在这个范围
内受拉的弹性模量可以取成与受压时相同。

\subsubsection{泊松比}

横向应变与单轴加载方向的应变之比称为泊松比。混凝土的泊松比一般是在0.15到0.20的范
围内。而已经测得的值则是在0.1到0.3之间。目前看来还没有关于泊松比随混凝土性质而变
化的可靠资料,但一般都认为高强混凝土的泊松比值较低。

\subsection{复合受力性能}

在许多结构部位混凝土都承受着作用于几个方向的正应力和剪应力。但是如果考虑到作用于
混凝土单元体上各力的平衡条件,就能证明任何一个组合应力状态都能换算成作用在三个相
互垂直的平面上的正应力。这三个正应力都是主应力,作用在这三个平面上的剪应力均等于
零。

尽管已经做了广泛的研究工作,但仍然未能针对复合三向应力这种一般情况提出一个可靠的
混凝土破坏强度理论。人们也曾尝试修改传统的材料强度理论,但没有哪一种理论可以准确
地适用于所有的情况。不过在很多实用场合下,个别比较简单的破坏理论还是能给出足够的
准确性的。

\subsubsection{双轴受力性能}

当主应力只在两个方向作用时,就出现双轴应力状态,即应力都作用在一个平面内,而第三
个主应力为零。表示的是由Kupfer、Hilsdorf和Rusch求得的导致混凝土破坏的两个方向的
正应力的组合。这些研究者得出的结论是,双轴受压混凝土的强度能比单轴强度高到27\%。
对于双向压应力相等的情况,强度的增长接近16\%。而双轴受拉的强度则近似等于单轴抗拉
强度。但应注意,受拉和受压相组合将使破坏时的拉应力和压应力都降低。

由Pister得出的关于单元体在单项正应力与剪应力组合作用下的破坏曲线如所示。该曲线表
明在有剪应力作用的情况下混凝土的抗压强度有所降低。举例来说,如果梁和柱内由剪应力
存在,这种作用就能影响梁、柱受压区混凝土的强度。

\subsection{受钢筋约束的混凝土}

在实际工程中可以用通常做成密排螺旋筋或箍筋形式的横向钢筋来约束混凝土。在这种情况
下,当混凝土的应力较低时,横向钢筋几乎并不受力,所以混凝土是不受约束的。当混凝土
中的应力接近单轴强度时,由于内部开裂的不断发展,使得横向应变变得很大,混凝土向外
挤压横向钢筋,横向钢筋就对混凝土作用一个约束反作用力,是混凝土变成受约束的。因此
可以说横向钢筋提供的是被动约束作用。很多研究者的试验都已证明横向钢筋的约束作用可
以显著地改善混凝土在大应变时的应力——应变特性。

试验表明,环形螺旋筋比矩形或方形箍筋更能有效地约束混凝土。环形螺旋筋由于它的形状
而处于环向轴心受拉状态,并形成沿圆周的连续约束压力。这在大横向应变时近似于液压的
约束作用。然而,方形箍筋通常只能在其拐角附加产生约束反作用,原因是混凝土对箍筋直
边的压力会使箍筋的直边向外弯曲,因此,混凝土横截面的相当一部分可能是不受约束的。
由于两个拐角之间的内部拱作用,混凝土只是在拐角处和截面的中心区域受到有效的约束。
不过,方形箍筋确实引起了延性的显著增长,而且许多研究者还观察到强度有所提高。

直到接近混凝土单轴强度时,横向钢筋的约束作用对应力——应变曲线的影响都是很小的。应
力——应变曲线在大应变区的形状则是一个多变量函数,其中主要有以下变量:
\begin{enumerate}
\item 横向钢筋的体积与混凝土核心体积之比。因为高的横向钢筋含量产生大的横向约束压
  力。
\item 横向钢筋的屈服强度。因为它给出了约束压力的上限。
\item 横向钢筋间距与混凝土核心尺寸之比。间距越小,约束效果越好。混凝土是被各根横
  向钢筋之间混凝土的起拱作用所约束的;如果间距偏大,显然就有较大的混凝土体积不能
  受到约束,从而可能剥落。
\item 在采用矩形或方形箍筋的情况下横向钢筋直径与其无支长度之比。因为钢筋直径越大,
  产生的约束效果就越好。小直径的横向钢筋在拐角之间只能起连杆作用,这时因为它们的
  抗弯刚度小,要向外凸出,而不能在箍筋拐角之间的区域内有效地约束混凝土。随着横向
  钢筋直径与其无支长度之比的增大,被有效约束的混凝土面积也就增大了,这时由于箍筋
  各边的侧向抗弯刚度较大。如果采用环形螺旋筋,这个变量就没有意义,因为螺旋筋由于
  它的既定形状而承受轴向拉力并对混凝土施加均匀的径向压力。
\item 纵向钢筋的含量和尺寸。因为这种钢筋也将约束混凝土。纵向钢筋一般直径较大,其
  直径与无支长度之比通常能使它有效地约束混凝土。不过,纵向钢筋必须安放成紧贴横向
  钢筋。这时因为横向钢筋要对纵向钢筋提供形成约束的反作用力。如果纵向钢筋必须移动
  后才能与横向钢筋有效地紧密接触,它的约束效果就会降低。
\item 混凝土的强度。因为低强混凝土比高强混凝土具有更大的延性。
\item 加载速度。因为混凝土的应力——应变特征与时间有关。
\end{enumerate}

横向钢筋外面的混凝土是不受约束的。可以预料这部分保护层混凝土或外壳混凝土与横向钢
筋以内的混凝土具有不同的应力——应变特征。保护层混凝土通常在达到无约束强度时就开始
剥落,特别是当横向钢筋含量偏大的时候。这是因为大量横向钢筋的存在使得混凝土核心与
保护层之间形成了一个薄弱面,并因而加速了剥落现象的发生。因此,若横向钢筋含量偏大,
就应在大应变时忽略保护层混凝土的作用。保护层混凝土或许可以假定成直到一个假定的剥
落应变为止都具有无约束混凝土的特征。当大于这个应变时,就不在考虑它的作用。如果横
向钢筋含量较低,保护层混凝土就不会剥落得那么快,并能与被约束的核心一起更多地发挥
作用。在这种情况下,也许可以适当考虑保护层混凝土在大应变下的作用。

\subsubsection{受环形螺旋筋约束的混凝土}

假定螺旋筋的螺距小到能对混凝土作用一个近似均匀的压力,就可以根据螺旋筋的环向拉力
把约束压力计算出来。表示半个螺旋圈的脱离体。当螺旋筋达到屈服强度$f_y$时,作用在
混凝土上的侧压强$f_l$便达到了最大值。若$d_s$为螺旋圈直径, $A_{sp}$为螺旋筋截面
积, $s$为螺距,则作用在所示半圈螺旋筋上各力的平衡条件要求:
\begin{equation}
  \label{eq:33}
  2f_yA_{sp}=d_ssf_l \qquad \qquad f_l=\frac{2f_yA_{sp}}{d_ss}
\end{equation}

将代入,我们就可以看出受螺旋筋约束的混凝土的轴心抗压强度为
\begin{equation}
  \label{eq:34}
  f'_{cc}=f'_c+8.2\frac{f_yA_{sp}}{d_ss}
\end{equation}

受螺旋筋约束的混凝土在延性方面的增长也是显著的。

\subsection{混凝土的徐变}

混凝土的应力——应变关系是时间的函数。由于混凝土的徐变变形,混凝土在应力作用下的应
变是随时间而逐渐增大的。最终的徐变应变可能是初始弹性应变的好几倍。通常,徐变对结
构的强度几乎是没有什么影响的;但是,它将引起钢筋混凝土构件在使用荷载下的应力重分
布并导致使用荷载下挠度的增加。在有些情况下徐变变形是有利的。例如,徐变可以降低结
构不均匀沉降所引起的混凝土应力,受拉徐变可以延缓混凝土的收缩开裂。关于由徐变引起
的应力和应变的计算方法将在第十章中进行探讨。

徐变发展的速度系随时间而减慢的。如果移去荷载,弹性应变就立即恢复。不过由于弹性模
量随龄期增大,这时恢复的弹性应变比初始弹性应变为小。徐变的恢复紧随在弹性恢复之后,
但恢复的徐变只是总徐变的一小部分。

徐变应变的大小取决于混凝土的组成、环境和应力——时间史。

混凝土的组成可以按骨料的品种和级配、水泥的品种和用量、水灰比及掺合料来描述。由于
骨料的吸水性和弹性性质各不相同,所以骨料的品种能对徐变产生明显的影响。例如,使用
砂岩骨料时可能产生两倍于使用石灰岩骨料时的徐变。骨料在体积上比水泥胶体更稳定,因
此增加骨料用量可以导致徐变的降低。增大水灰比和增加水泥用量都会像大量使用加气剂那
样增大徐变。

环境则可以按湿度和构件的外形及尺寸来描述。周围的相对湿度对徐变有显著影响。相对湿
度大,徐变应变就小。当构件失水受到限制时,徐变也要减小,因此构件的尺寸和外形就影
响着发生徐变的大小。此外,由于大构件的外部限制着内部水分的丧失,大构件的徐变就降
低了。混凝土尺寸和外形的影响也可以用体积与表面积之比或用有效厚度来表示。

应力——时间史可以用初次加荷时的龄期以及加荷持续的时间来描述。在龄期短的时候加荷产
生的徐变应变大。随着初次加荷时龄期的增加,徐变应变也就明显下降。徐变应变还随着加
荷时间的延长而增大。

\subsection{混凝土的收缩}

当混凝土通过蒸发而失去水分时就将产生收缩。收缩应变与混凝土中的应力状态无关。如果
混凝土受到约束,收缩应变就会引起开裂,而且还使结构构件的挠度随时间而增加。

收缩在很大程度上是一种可逆现象。如果混凝土收缩后在浸水到饱和,它就将几乎膨胀到原
有的体积。因此干燥与潮湿状态的交替就将引起混凝土体积的交替变化。这种现象是暴露在
每年季节性变化的气候条件中的结构(如混凝土桥梁)其挠度波动的一部分原因。

一般地说,表现出高徐变的混凝土也表现出高收缩。因此,与前面对徐变所讨论的情况大致
相同,收缩徐变的大小也取决于混凝土的组成以及环境。

\section{钢筋}

\subsection{钢筋的外形及尺寸}

钢筋的截面一般是圆形的。为了阻止钢筋相对于周围混凝土的纵向移动,就把肋条或突起物
轧制在钢筋的表面上。这种肋条或突起物统称为表面变形。表面变形的最低要求(间距、高
度及沿圆周覆盖的范围)已经通过试验研究确定下来并且已经由钢筋规范做了规定。美国试
验即材料学会规程要求这种表面变形的平均间距不超过钢筋名义直径的0.7倍,高度至少是
钢筋名义直径的0.04到0.05倍,而且它们至少要环绕名义周长的75\%。表面变形应布置成与
钢筋轴线的交角不小于45。在钢筋表面一般还轧有纵肋。

变形钢筋是按3号到18号的尺寸来生产的。这里,钢筋的号数表示它的名义直径中所包含的八
分之一吋的个数($1/8 in=3.18 mm$)。变形钢筋的名义尺寸就等于每延米与它具有同样重量
的光圆钢筋的尺寸。

\subsection{一次受力性能}

在钢筋混凝土结构中使用的钢筋的典型应力——应变曲线是由钢筋一次受拉求得的。这条曲线
有一个初起的线弹性部分、一个屈服台阶(即一个屈服点,在这个屈服点以后的应变是在应
力几乎不增加或者完全不增加的情况下增长的)、一个应力又随应变而增加的应变硬化段以
及应力下降直至发生断裂的最后一段。

钢筋的弹性模量是用应力——应变曲线的线弹性部分的斜率来定义的。钢筋的弹性模量通常是
取为。

屈服点处的应力,通常称为屈服强度,是钢筋的一个非常重要的特性。有时屈服可能伴随着
应力的突然降低,而使应力——应变曲线具有如所示的形状。在这种情况下,A点和B点的应力
分别称为上、下屈服强度。上屈服点的位置与试验速度、截面形状和试件形状有关。通常认
为下屈服强度才是材料的真实特性,并且把它简称为屈服强度。对于没有明显屈服台阶的钢
筋,屈服强度一般是取为对应于某个特定应变的应力。

屈服台阶的长度通常是钢材强度函数。高强高碳钢材的屈服台阶通常比低强低碳钢材的短得
多。同样,钢筋的冷加工也能使屈服台阶缩短到在开始屈服后就立即出现应变硬化的程度。
高强钢材在断裂前的延伸率也比低强钢材为小。

规定的屈服强度一般指的是具有保证的最小值。钢筋的实际屈服强度通常都比这个规定值高
一些。在某些情况下(例如在估计构件的抗震强度时)并不希望屈服强度过多地啊多数设计中
考虑的强度。举例来说,这时因为得到提高的构件抗弯强度在极限荷载时将会导致增大了的
剪力作用在构件上,从而就可能造成构件的脆性剪坏而不是延性弯坏。因此,用于地震区结
构的钢筋规程还应要求某个给定级别的钢筋不应超过某一个确定的屈服强度。

钢筋的受拉和受压应力——应变曲线通常假定是相同的。试验表明这是一个合理的假定。

在设计中需要把应力——应变曲线的形状理想化。通常是通过把这条曲线理想化两段直线来进
行简化,如所示。它忽略了上屈服强度和由应变硬化所引起的应力增长。这就是ACI规范所
假定的钢筋应力——应变曲线。如果屈服后在近似不变的应力下所产生的塑性应变比屈服时的
弹性伸长大得多,这条假定的曲线就能给出非常好的准确度。这种简化对低屈服强度的钢筋
特别准确。若钢筋在屈服开始后不久就发生应变硬化,这条假定曲线就会低估钢筋在高应变
时的应力。在某些情况下,为了更准确地估计构件在大变形时的强度,就可能需要求出钢筋
在大于屈服应变时的应力值。这在抗震设计中是一个特别实际的问题,因为在这里对延性的
要求可能意味着要考虑应变值将达到相当于屈服应变很多倍的可能性。给出了更加准确的、
可用的理想化应力——应变曲线。为了使用这样的理想化曲线,就需要明确钢筋在开始屈服时、
应变硬化时以及达到抗拉强度时的应力及应变值。这几个点的位置可以根据试验得出来的应
力——应变曲线来确定。

\subsection{交变受力性能}

如果在钢材试件的屈服范围之内对它施加交变的(拉——压)轴心荷载,就可以得到一条具有所
示形式的应力——应变曲线。这个图形表示除了包兴格效应。由于这个效应,在交变荷载下的
应力——应变曲线在应力远低于初始屈服强度时就变成非线性的。钢材的这种性能受到了在这
以前应变历程的强烈影响;时间和温度也起了一定作用。在这里,卸载的路线遵循初始的弹
性斜率。经常用于交变荷载的弹性——全塑性理想化曲线只是一种近似曲线。在考虑强烈地震
荷载对构件的效应时交变加载曲线是很重要的。

\begin{figure}[htbp] \centering
  \begin{tikzpicture}[line width=1pt,>=angle 90,scale=1.4] 
\draw [->] (-1,0)--(8,0) node [blue,right]{$f$}; 
\draw [->] (0,0) --(0,3) node [blue,above]{$\nu$}; 
\draw [red] (-0.5,0) .. controls (1,2) and (6.5,2.5)..(7,0);
  \end{tikzpicture}
\end{figure}

\chapter{抗弯强度理论的基本假定}

\section{基本性能的假定}

在推导钢筋混凝土截面抗弯强度的一般理论时采取了四项基本假定:
\begin{enumerate}
\item 弯曲前的平截面在弯曲后 仍保持平面。
\item 钢筋的应力——应变曲线是已知的。
\item 混凝土的抗拉强度可以忽略不计。
\item 确定混凝土压应力大小及分布的应力——应变曲线是已知的。
\end{enumerate}

第一项假定,即伯努利定律,它的含义是截面上各点的混凝土和钢筋的纵向应变与该点到中
性轴的距离成正比。大量的钢筋混凝土构件试验表明,只要混凝土和钢筋之间保持着良好的
粘结,则在直至弯曲破坏为止的各个加载阶段中,这项假定都是很接近于正确的。在混凝土
受压区,这项假定肯定是准确的。在混凝土受拉区出现一条裂缝意味着钢筋与周围混凝土之
间产生了一些滑移,而这也表示这项假定对裂缝附近的混凝土不是完全适用的。不过,如果
用包含有几条裂缝的标距长度来测定混凝土的应变,就会发现伯努利定律对这样得出的平均
拉应变是适用的。表示在钢筋混凝土柱的破坏区域附近所测得的不同荷载增量下应变沿截面
的分布状况。柱的截面或为\SI{10}{in}(\SI{254}{mm})见方,或为直径
\SI{12}{in}(\SI{305}{mm})的圆形。钢筋的应变是用\SI{1}{in}{\SI{25}{mm}}标距测定的,
而混凝土的应变则是用\SI{6}{in}(\SI{150}{mm})标距测定的。由于在单个应变量测量值中
存在的微小不准确性以及各个量测线位置的少量误差,稍稍偏离直线的情况是必然存在的。
从可以明显看出这些实测的应变分布是相当直的。因此,对于设计来说,平截面保持平面的
假定无疑是足够准确的。但这个假定在深梁或高剪力区内就不适用了。

第二项假定意味着钢筋的应力——应变性质是意义明确的。这个应力——应变曲线通常是假定为
双折线,从而忽略了应变硬化现象。在钢筋规程中未对应变硬化的起点作出规定,因此就难
于把它考虑进来。在一般情况下,要利用应变硬化所造成的任何强度增长都可能是不明智的,
因为这样构件就会伴随产生很大的极限变形。而当强度的增大可能造成不利情况时(例如在抗
震设计中引起脆性的剪切破坏而不是延性的弯曲破坏时),设计者便可参照钢筋的实际应力——应
变曲线来考虑应变硬化所引起的附加强度。

第三项假定几乎是完全正确的。因为在紧靠中和轴下面的混凝土总存在的任何拉应力都是很
小的,而且内力臂也非常小。

第四项假定是估计截面的真实性能所必需的。鉴于受压混凝土各点的应变是与该点到中性轴
的距离成正比的,\cref{fig:1}的应力——应变曲线的形状就表示了不同加载阶段压应力图形
的形状。是表示梁的一个截面在弯矩增长时其应力图形不断变化的形状。当混凝土上的总压
力乘以内力臂 $jd$达到最大值时,这个截面就达到了它的抗弯强度(最大抵抗弯矩)。如所示,
最大弯矩截面的压应力图形特征可以用参数$k_1$、$k_2$和$k_3$来确定。对于宽度为b、有
效高度为d的矩形截面,混凝土的总压力为$k_1k_3f'_cbc$,其内力臂即为$d-k_2c$,此处c
为受压边缘到中和轴的距离。不少人为了确定无约束混凝土的这些参数曾经作了大量的工作。
最著名的成果是由Hognestad等人在波特兰水泥协会(PCA)所做的短期试验和Rusch的试验。在
PCA试验中所采用的试件与所示的相类似。对试件的测试区段是通过不断增大两个侧压力
$P_1$和$P_2$来施加偏心荷载的。这两个侧压力$P_1$和$P_2$各自独立变化以保持中和轴(即
应变为零的纤维)在整个试验中始终位于试件的底面。这样就模拟出了受弯构件受压区的应力
分布状态。这时按照内、外力相等及内、外弯矩相等的条件,即可直接计算出$k_1$、$k_2$
和$k_3$的数值,同样也就的到了试件中混凝土的应力——应变曲线。与此同时也用轴心加载的
圆柱体测得了混凝土的应力——应变曲线,并已发现与上述试件中混凝土的应力——应变曲线相
似。但是当混凝土强度较高时,试件在达到抗弯强度所具有的最大应力$k_3f'_c$却略小于圆
柱体的强度。这些试验还测定了在达到抗弯强度时边缘受压纤维的混凝土应变$\epsilon_c$。
这里对砂——卵石混凝土求得的应力图形参数值是随圆柱体强度$f'_c$而变化的。这些数值列
于中。表中的数值对应于由各次试验求得的$k_1k_3$的最大值。

PCA的试验结果,即轴心受压混凝土的应力——应变曲线与由偏心加载试件得出的结果极为相似
的这个结论,不时为人们所怀疑。例如Sturman、Shah和Winter曾对偏心和轴压加载试件做过
试验,并且断定偏心试件曲线的顶峰是出现在应力比轴心试件高20\%和应变高50\%的时候。
应变梯度的存在可能不会有明显的影响,而如果有的话就会导致压应力图形的改善。还值得
注意的是,应变梯度的存在推迟了受压区纵向裂缝的出现。

\section{等效矩形应力图形}

许多研究者,例如Whitney,曾建议用等效矩形压应力图形来代替混凝土压应力图形的真实形
状以作为一种简化手段。为了求出抗弯强度,只需要知道混凝土压力的大小($k_1k_3$)和位
置($k_2$)。而用等效矩形应力图形就可以实现这个目的并使计算大为简化。以ACI规范为代
表的美国实际作法已经用等效矩形图形代替了真实的应力图形。这个矩形的平均应力为
$0.85f'_c$,高度为$a$。其中当$f'_c\le
40000\mathrm{psi}$(\SI{27.6}{\newton\per\square\milli\meter})时,
$a/c=\beta_1=0.85$;而当强度超过4000psi时,则每增加\SI{1000}{psi}
(\SI{6.89}{\newton\per\square\milli\meter}),$\beta_1$值就减小0.05。降低高强混凝
土的$\beta_1$值主要是因为这种混凝土的应力——应变曲线的形状不太有利。

为了使所示的实际应力图形的压力合力与等效应力图形的压力合力具有相同的大小和作用线,
各参数值必须满足:
\begin{equation}
  \label{eq:9}
  C=k_1k_3f'_cbc=0.85f'_cba \qquad k_1k_3=0.85\frac{a}{c}=0.85\beta_1
\end{equation}
和
\begin{equation}
  \label{eq:10}
  k_2c=0.5a \qquad k_2=0.5\frac{a}{c}=0.5\beta_1
\end{equation}

将ACI建议的$\beta_1$代入\cref{eq:9}和\cref{eq:10}从而求得
的$k_1k_3$和$k_2$值与PCA和Rusch由无约束混试件的试验得出的实际值进行对比,即如所示。
可以看出,对矩形应力图形的性质所建议的数值与试验值颇为一致。试验值的分散情况清楚
的表明:对矩形应力图采用更复杂的参数值是不必要的。此外,在只要很少几个试验结果是
关于圆柱体大于\SI{8000}{psi}的。但是由图中试验结果的趋势来看,ACI的应力图形参数对
圆柱体强度大于\SI{8000}{psi}的情况是偏于保守的。甚至可以认为ACI的参数在混凝土强度
较高是过于保守了。

\section{达到抗弯强度时的混凝土应变}

美国的实际作法(ACI318-71)是建议在截面达到抗弯强度时最外边缘混凝土受压纤维的最大可
用应变为0.003。很多研究者都曾测定过当$k_1k_3$最大时或截面抵抗弯矩最大时的应变。该
图表明0.003是一个相当保守的数值。尽管这个应变大于最大应力相对应的应变;但是当达到
这个应变时,受弯构件的受压混凝土一般都不会出现可见的开裂或剥落。当一个轴心加载的
圆柱体超过了最大应力所对应的应变时,通常裂缝就会相当明显,但在受弯试验中,直至达
到更大的应变时都不会见到裂缝。这也许是在距中和轴附近有应变较小的材料存在的缘故。

钢筋混凝土梁的计算抗弯刚度通常对所假定的最大混凝土应变值并不十分敏感。引
自Blume、Newmark和Corning的把这一点表示得很清楚。在图中对一个单筋混凝土梁的截面画
出了在两种不同受拉钢筋含量的情况下由混凝土的应力——应变曲线计算出的抵抗弯矩与
按ACI规范计算出抵抗弯矩的比值相对于边缘受压纤维应变的关系曲线。在前一项计算中采用
的应力——应变曲线是取自强度约为\SI{3600}{psi}的圆柱体。先对不同的应变值确定应
力——应变曲线下面的面积及其形心,从而定出对应于一系列边缘纤维应
变$\epsilon_c$的$k_1k_3$和$k_2$值。然后再按两种钢筋含量计算出不同的边缘纤维应变值
相对应的截面抗弯刚度,并与ACI规范计算出的抗弯强度进行比较。当边缘纤维的压应变
为0.007时,截面抵抗弯矩的降低幅度在$=0.005$时尚不足1\%,在$=0.025$时也不足6\%。由
此可见混凝土边缘纤维应变的取值在很大范围内对梁的抗弯强度几乎都没有影响。但是对于
受压破坏的偏心受压柱来说,边缘纤维应变增大时所发生的应力图形参数值的变化却可将使
抗弯强度随应变而产生较大的变化。

可是很明显,一个截面的曲率在很大程度尚取决于边缘纤维的取值。为了计算极限曲率,取
用大于0.003的应变看来可能是合理的。Blume、Newmark和Corning建议在计算无约束混凝土
的极限曲率时把应变值取为0.004。

\section{非矩形受压区}

对矩形受压区中等效矩形应力图所建议的参数值严格来说不适用于混凝土截面受压区不是矩
形的构件,例如中性轴通过肋部的T形和L形梁以及承受双向弯矩的梁和柱。这是因为等效矩
形应力图的平均应力和高度对于各种形状的受压区将是不同的,而且在最大弯矩时边缘纤维
混凝土的应变也将是不同的。给出了Rusch对几种典型截面计算出的混凝土边缘纤维在最大弯
矩时的压应变。图中的曲线表示混凝土的应力——应变曲线及截面压应力图形的形状。也考虑
了中和轴位置的两种数学上的极端情况。实心圆点表示中和轴通过受拉钢筋形心的情况,空
心圆圈表示中和轴位于受压边缘纤维处的情况。大多数构件的实际情况将处于这两种极端之
间。清楚地展示了当构件达到抗弯强度时,受压区形状对边缘纤维应变的影响。例如在双向
受弯的柱子中出现的三角形受压区,它在最大弯矩时的应变可能是T形截面的两倍。这种差别
之所以产生是因为对三角形受压区来说,受压面积的较大部分是位于中和轴附近的,因而最
大弯矩便发生在边缘纤维应变相对较大的情况,而T形截面的情况却恰好相反。

Rusch和Stockl的进一步研究已为非矩形受压区提出了应力图形参数。但是从这项成果以
及Mattock和Kirz的研究成果中可以明显地看出,除非截面严重超筋,采用对矩形受压区得出
的应力图形参数和边缘纤维应变还是能相当准确地估计出受压区不是矩形的梁的抗弯强度的,
因为力臂和内力均为明显受到影响。而对于具有非矩形受压区的柱,采用基于矩形受压区的
参数却无法获得能够接受的准确性。这是因为压力更大一些,而且混凝土压应力的分布对截
面的抗弯强度有更加明显的影响。例如,对于承受双向弯曲的柱截面就可能需要根据基本原
理按混凝土应力——应变曲线所导出的更加准确的参数。所以,为矩形受压区导出的参数在梁
的设计中将能得出足够的准确性,但是用于具有非矩形受压区的柱子时则应谨慎。

\section{为了确定截面受弯兼受轴向荷载时的强度所做的建议摘要}

可以把为了确定截面受弯兼轴向荷载时的强度而作的假定概括如下:
\begin{enumerate}
\item 弯曲前的平截面在弯曲后仍保持平面。
\item 混凝土的应力分布可以取为矩形,其平均应力为$0.85f'_c$,其高度自受压边缘算起
  为$\beta_1c$,$c$为截面受压边缘到中和轴的高度。当$f'_c \ge
  4000\mathrm{psi}$时,$\beta_1$值为0.85,当强度超过\SI{4000}{psi}时,每超
  过\SI{1000}{psi},$\beta_1$就降低0.05。
\item 混凝土的抗拉强度可以忽略不计。
\item 当构件达到抗弯强度时,边缘受压纤维处的混凝土应变可以取为0.003。
\item 钢筋在低于屈服强度时的应力可以取为钢筋的应变乘以弹性模量\SI{29e8}{psi}。当
  其应变大于屈服强度所对应的应变时,钢筋的应力可以认为保持在屈服强度的水平。
\item 上述混凝土压应变及压应力的分布可以用于具有非矩形受压区的梁,但对于具有非矩
  形受压区的柱,则有可能必要采用混凝土的应力——应变曲线为依据的更为准确的参数。
\item 持续加载的影响可以忽略不计。
\end{enumerate}
混凝土压应力的分布可以取成任何一种形状,只要它能可靠地预测构件的抗弯强度。前面用
过的几种可供选择的混凝土压应力与压应变之间的关系曲线有双直线、抛物线和抛物线直线
组合曲线。由CEB-FIP建议的,而且在欧洲通用的曲线是一条到应变为0.002为止的二次抛物
线和接在它后面的一段直到应变为0.0035的水平直线所组成的。而由ACI规范建议的矩形压应
力分布图形(见上述第2条假定)则能把混凝土强度对应力图形的影响考虑进去,并使抗弯强度
方程式的推导得以简化。

ACI与欧洲通用作法的另一个区别在于所建议的可用的最大钢筋应变值。例如:ECB-FIP的建
议是把构件达到抗弯强度时的钢筋最大拉应变限制在0.01以内,而ACI规范则对构件达到抗弯
强度时钢筋拉应变的大小不作任何限制(见第5条假定)。关于钢筋应变的这种限制几乎没有对
计算处的抗弯强度的大小造成什么差异,但却能限制一个构件所计算出的可资利用的极限变
形。鉴于钢筋的极限应变远远大于0.01,就难以看出这种限制的必要性。

\chapter{受弯构件的强度}

梁时承受横向外荷载的结构构件。荷载沿梁长引起弯矩和剪力。本章研究梁截面的抗弯强
度。

\section{矩形截面}

\subsection{单筋截面的分析}

表示了一个达到抗弯强度的单筋混凝土截面。其总内拉力为:
\begin{equation}
  \label{eq:11}
  T=A_sf_s
\end{equation}
其中$A_s$为钢筋截面积,$f_s$为钢筋应力。

由于钢筋所占的高度与截面高度相比是很小的,就假设整个钢筋截面上的应力是均匀的并等
于钢筋截面形心处的应力。

总内压力为:
\begin{equation}
  \label{eq:12}
  C=0.85f'_cab
\end{equation}
其中$a$为等效应力图形的高度;$b$为截面宽度;$f'_c$为混凝土的圆柱体抗压强度。

两个合内力的距离通称内力臂,可用下式给出:
\begin{equation}
  \label{eq:13}
  jd=d-0.5a
\end{equation}
式中$d$为受压纤维到钢筋截面形心之间的距离,称为截面的有效高度。

因此截面的抵抗弯矩为
\begin{equation}
  \label{eq:14}
  M_u=Tjd=Cjd
\end{equation}

下面讨论几种可能的弯曲破坏形式(受拉破坏、受压破坏及平衡破坏)和截面的理想抗弯强
度。

\subsubsection{受拉破坏}

如果截面的钢筋含量较少,钢筋将在混凝土达到最大抗压能力之前达到它的屈服强度$f_y$。
在进一步加载的情况下,钢筋的拉力仍将保持为$A_sf_y$。少量的追加荷载将使穿过弯曲裂
缝的钢筋产生很大的塑性伸长,并导致裂缝加宽以及混凝土边缘受压纤维处应变的大幅度增
长。随着应变的这种增长,混凝土压应力的分布就变成明显的非线性,致使压应力图形的平
均应力增大;由于必须保持内力平衡,而使中和轴深度减小。这个距离的减小引起内力臂长
度的少量增加,因而使抵抗弯矩也稍稍增大。正如在第3.3节讨论过的,随着应变的进一步增
长,抵抗弯矩最终还要减小,而且混凝土受压区开始压碎。已经表示出了当加荷直至达到抵
抗弯矩时混凝土压应力图形形状的变化。表示的是在梁柱节点处一根梁试验到破坏以后的情
况。这根梁发生的是由正弯矩引起的弯曲破坏(位于梁两侧的短销能使应变量测得以进行)。
由于这类破坏始于受拉钢筋反而屈服,因而把它叫做”主要由受拉引起的破坏“可能更适合
些。但是为了说起来简单就使用了"受拉破坏"这个名称。值得注意的是,在截面达到抗弯强
度时,除非钢筋含量太少,钢筋是不会被拉断的。导致钢筋拉断所需达到的很高拉应变是与
特别小的中和轴深度相关的。

对于受拉破坏,$f_s=f_y$,其中$f_y$是钢筋的屈服强度。为了满足平衡条件,
$C=T$,因此,从\cref{eq:11}和\cref{eq:12}得出:
\begin{equation}
  \label{eq:15}
  0.85f'_cab = A_sf_y \qquad a= \frac{A_sf_y}{0.85f'_cb}
\end{equation}
所以由\cref{eq:13}和\cref{eq:14}可以写出下列公式:
\begin{subequations}
  \begin{align}
    M_u &= A_sf_y(d-0.5a)\\
        &= A_sf_y
        \left(
          d-0.59\frac{A_sf_y}{f'_cb}
        \right) \label{eq:6000}\\
        &= \rho bd^2f_y
        \left(
          1-0.59\frac{\rho f_y}{f'_c}
        \right) \label{eq:28}\\
        &= bd^2f'_c\omega(1-0.59\omega)
  \end{align}
\end{subequations}
其中$\rho=\frac{A_s}{bd}\qquad \omega=\frac{\rho f_y}{f'_c}$

\subsubsection{受压破坏}

如果截面的钢筋含量较大,混凝土就可能在钢筋屈服之前达到最大抗压能力。在这种情况下
中和轴的深度便显著增大,从而使压力相应增大。可是这种增大由于内力臂的减小而稍微受
到一些抵消。但仍然是当混凝土边缘受压纤维的应变约为0.003时达到的截面的抗弯强度,而
截面便突然以脆性方式发生破坏。在这里,由于钢筋应力低,破坏截面混凝土受拉区的弯曲
裂缝宽度小,就几乎没有什么明显的破坏警告。

对于受压破坏,$f_s\le f_y$,即钢筋仍处于弹性范围以内。利用中应变图形的相似三角形
关系,就可以通过中和轴深度来确定钢筋应力。
\begin{equation}
  \label{eq:17}
  \frac{\epsilon_s}{0.003}=\frac{d-c}{c}\qquad \epsilon_s=0.003\frac{d-c}{c}
\end{equation}
\begin{subequations}
  \begin{equation}
    f_s=\epsilon_sE_s=0.003\frac{d-c}{c}E_s
  \end{equation}
由于$a=\beta_1c$,
\begin{equation}
 f_s=0.003\frac{\beta_1d-a}{a}E_s
\end{equation}
\end{subequations}
为了满足平衡条件,$C=T$,故由\cref{eq:11}和\cref{eq:12}可得
\begin{equation*}
  0.85f'_cab=A_sf_s=0.003\frac{\beta_1d-a}{a}E_sA_s
\end{equation*}
\begin{equation}
  \label{eq:20}
  \left(
\frac{0.85f'_c}{0.003E_s\rho}  \right)a^2+ad-\beta_1d^2=0
\end{equation}
解此方程即可得$a$,并由\cref{eq:13}和\cref{eq:14}得出
\begin{equation}
  \label{eq:21}
  M_u=0.85f'_cab(d-0.5a)
\end{equation}

\subsubsection{平衡破坏}

在一个特定的钢筋含量情况下,钢筋达到屈服强度与混凝土边缘纤维压应变达到0.003同时发
生。

这是$\epsilon_s=f_y/E_s$,于是根据应变图形的相似三角形关系,我们就能写出:
\begin{equation*}
  \frac{f_y/E_s}{0.003}=\frac{d-c_b}{c_b}
\end{equation*}
式中$c_b$为平衡破坏时中和轴的深度。
\begin{equation}
  \label{eq:22}
  c_b=\frac{0.003E_s}{0.003E_s+f_y}d
\end{equation}
或
\begin{equation}
  \label{eq:23}
  a_b=\frac{0.003E_s}{0.003E_s+f_y}\beta_1 d
\end{equation}
式中$a_b$为平衡破坏时等效矩形应力图形的高度。

为了满足平衡条件,$C=T$;因此我们得到
\begin{equation*}
  0.85f'_ca_bb=A_sf_y=\rho_bbdf_y
\end{equation*}
式中
\begin{equation*}
  \rho_b=\frac{A_s}{bd}
\end{equation*}
对于平衡破坏
\begin{equation}
  \label{eq:25}
  \rho_b=\frac{0.85f'_ca_b}{f_yd}
\end{equation}
把\cref{eq:23}代入\cref{eq:25}得
\begin{equation}
  \label{eq:26}
  \rho_b=\frac{0.85f'_c\beta_1}{f_y}\frac{0.003E_s}{0.003E_s+f_y}
\end{equation}

在一般情况下,当截面的$\rho$不同于$\rho_b$时,其破坏类型将取决于$\rho$是小
于$\rho_b$还是大于$\rho_b$。表示三种不同钢筋含量的截面在达到抗弯强度时的应变分布。
正如\cref{eq:15}和\cref{eq:20}所表示的那样,中和轴的深度取决于钢筋含量。对进行观
察可以看出,如果截面的$\rho$小于$\rho_b$,则$c<c_b$,$\epsilon_s>f_y/E_s$,因此发
生受拉破坏。同样,如果$\rho>\rho_b$,$c>c_b$,$\epsilon_s<f_y/E_s$,从而出现受压
破坏。

当$\rho<\rho_b$时,发生受拉破坏;

当$\rho>\rho_b$时,发生受压破坏。

注意,在假定计算公式在科学上时正确的,材料恰好具有所规定的强度,而且截面尺寸也与
假定相一致,就可以认为由上述抗弯强度公式所给出的就是截面的理想抗弯强度。

\begin{example}
  一个单筋矩形截面的宽度为\SI{10}{in}(\SI{254}{mm}),有效高度
  为\SI{18}{in}(\SI{457}{mm})。混凝土的圆柱体抗压强度
  为\SI{3000}{psi}(\SI{20.7}{\newton\per\square\milli\meter}),钢筋的弹性模量
  为\SI{29e6}{psi}(\SI{0.20e6}{\newton\per\square\milli\meter}),屈服强度
  为\SI{40000}{psi}(\SI{275.8}{\newton\per\square\milli\meter})。试计算当钢筋截面
  积为下列数值时截面的理想抗弯强度:(1)\SI{4}{in^2};(2)\SI{8}{in^2};(3)平衡破坏
  时的数值。
\end{example}

\textbf{解}

根据\cref{eq:26}我们可以算出
\begin{equation*}
  \rho_b=\frac{0.85\times3000\times0.85}{40000}\frac{0.003\times29\times10^6}{0.003\times29\times10^6+40000}=0.0371
\end{equation*}

1.\quad $A_s=$\SI{4}{in^2}
\begin{equation*}
  \rho=\frac{A_s}{bd}=\frac{4}{10\times18}=0.0222<\rho_b
\end{equation*}
因此发生受拉破坏,并由\cref{eq:6000}得
\begin{align*}
    M_u &= 4\times40000
    \left(18-0.59\frac{4\times40000}{3000\times10}    \right)\\
        &= 2.37\times10^6\mathrm{lb\cdot in}(\SI{268}{\kilo\newton\meter})
\end{align*}

2.\quad $A_s=$\SI{8}{in^2}
\begin{equation*}
  \rho=\frac{8}{10\times18}=0.0444>\rho_b
\end{equation*}
因此发生受压破坏,并由\cref{eq:20}得:
\begin{equation*}
  \frac{0.85\times3000}{0.003\times29\times10^6\times0.0444}a^2+18a-0.85\times18^2=0
\end{equation*}
\begin{equation*}
  a^2+27.27a-417.3=0
\end{equation*}
这个二次方程的解给出$a=10.93$in(\SI{278}{mm})(方程的另一个根为负数)。于是
由\cref{eq:21}得:
\begin{align*}
  M_u &= 0.85\times3000\times10.93\times10(18-0.5\times10.93)\\
  &= 3.49\times10^6\mathrm{lb\cdot in}(\SI{394}{\kilo\newton\meter})
\end{align*}

3.\quad $\rho=\rho_b=0.0371$

从\cref{eq:28}可以得到:
\begin{align*}
  M_u &=0.0371\times10\times18^2\times40000
  \left(
1-0.59\frac{0.0371\times40000}{3000}
  \right)\\
&=3.41\times10^6\mathrm{lb\cdot in}(\SI{384}{\kilo\newton\meter})
\end{align*}

中的曲线展示了上例截面的抗弯强度随同钢筋面积的变化情况。这条曲线是用例4.1中的公式
按其变化幅度远远进入受压破坏范围的一系列钢筋面积确定出来的。很明显,在受拉破坏范
围内,抵抗弯矩并不与钢筋面积成比例增长。这是因为虽然钢筋应力按比例增长,但是随着
钢筋含量的增加内力臂有所减小。如在上例中,内力臂系数从钢筋面积为零时的1.00减小到
平衡破坏时的0.71。在受压破坏范围内,随着钢筋面积的增加,抵抗弯矩增长极少,因为在
这个范围内钢筋应力和内力臂两者都随钢筋面积的增加而减少。因此,如果把钢筋面积增长
到超过平衡破坏时的钢筋用量,就几乎不可能获得额外的抗弯强度了。

值得注意的是,Whitney在1937年就提出了下列抗弯强度公式:

Whitney是把这些公式建立在矩形混凝土应力图的基础上的。他导出的这个矩形应力图形与今
天使用的完全相同。Whitney的受拉破坏公式与目前使用的是一样的。Whitney当时就找到了
平衡配筋率的数值,这个配筋率是根据一些梁的试验确定的。超出这个配筋率进一步增加钢
筋并未引起抗弯强度的明显增长。就是这个平衡配筋率,而Whitney的受压破坏公式就是极限
抵抗弯矩。虽然上述公式得自经验,但值都是相当准确的。

\subsection{单筋截面的设计}

在1.3节中已经讨论了在设计中使用强度公式要引用荷载系数和承载能力降低系数以保证结构
安全的问题。

在实际工程中受压破坏是危险的,因为这种破坏是突然发生的,几乎没有明显的预告,而且
是脆性的。而受拉破坏具有延性特征,在破坏之前混凝土开裂很宽。为了保证所有的梁在即
将破坏时不仅具有所希望的能够发生明显警告的特征,而且在破坏时还具有适当的延性,建
议在单筋梁中受拉钢筋的面积不要超过平衡破坏时钢筋面积的0.75倍。把钢筋面积限制为平
衡配筋面积的一个份额是必要的,因为正如所表示的,若钢筋的屈服强度较高活混凝土强度
较低,一根加载到抗弯强度的梁就有可能发生受压破坏。

一般来说,若要把截面设计成高度最小,则所需的钢筋含量就将是最大容许钢筋含量。从可
以明显地看出,这样的设计需要很高的钢筋含量。如果不是必须谋求一个很小的截面高度,
采用就不经济,故最好采用配筋较少的高度偏大的截面。此外,具有最小可能高度的梁,它
的挠度也可能过大,因而就可能需要进行挠度验算。在ACI规范中列出的跨高比是一个合理选
择构件尺寸的良好指南。如果超过了这个跨高比,构件的挠度就需要验算。

按强度设计法可能会比按采用弹性理论和容许应力的另一种也可以采用的设计法设计出更加
矮的单筋梁(另一种设计法在10.2.5节叙述)。例如假定,这时可按ACI318-71规范中的另一种
也可以采用的设计法设计的梁,若使它在使用荷载产生的弯矩作用下混凝土达到容许应力时
钢筋也同时达到容许应力,就需要钢筋含量为。然而,强度设计所得出的却是,因此就能采
用低的多的截面。这样,在强度设计中选择单筋截面的尺寸时就有了很大的灵活性。

这里请注意,虽然为了避免受压破坏的可能性已经把取成,但在采用“超强”钢筋时仍然存
在危险。例如,具有最大容许钢筋含量且钢筋设计屈服强度为的单筋梁,如果钢筋的实际屈
服强度大于,也将产生受压破坏。因此,高于规定强度的屈服强度或许也能导致脆性破坏,
即或破坏时的弯矩较大。同样,如果混凝土的实际强度比规定值低,也有可能在较小的弯矩
下导致受压破坏。

规定一个在任何情况下都必须被超过的最小配筋率也是有道理的。之所以需要这样做是因为
如果配筋率过小,则按钢筋混凝土截面计算出的抗弯强度就会变得比引起截面开裂的弯矩还
小,而且只要一开裂,破坏就是突然的和脆性的。为了防止出现这种情况,建议当以单位时,
梁内的不小于,而当以为单位时,不小于。这个值是根据把按素混凝土截面的抗折模量算得
的截面开裂弯矩与按钢筋混凝土截面算得的抗弯强度取成相等的条件而求得的。

在上例的第二种情况下,为了对给定了尺寸的截面求出它的配筋率(或钢筋面积),必须求解
二次方程。二次方程有两个实根,而设计中取用的值总是较小的根。这个道理可以用画出单
筋截面的关系曲线的来说明。这条曲线只是当时有效,但是二次方程的解却给出了另外一个
根,这个根就是曲线的下降段降到设计弯矩处的那一点所对应的值。

把这个例题的计算结果与利用第10.2.5节所叙述的ACI318-71规范的另一种设计方法所得出的
计算结果进行对比是有意义的。对于这个例题中指定的钢筋和混凝土的强度来说,这两者的
弹性模量比理应为9,钢筋的容许应力理应为,混凝土的容许应力理应为。这时按使用荷载下
的钢筋和混凝土同时都达到容许应力I而作出的设计所需要的,这一结果可以与例题中这两种
方法所需要的钢筋含量相差很显著是由于钢筋的容许应力较低。对于有效高度小于的情况,
如果按照基于资用应力的另一种设计方法进行设计,就将需要双筋截面和比按强度设计法可
以设计出的单筋截面的用钢量多得多的钢筋。

\subsection{双筋截面的分析}

在双筋梁中与单筋梁中一样,受拉和受压破坏都可能发生。当受拉破坏时,受拉钢筋屈服;
但当受压破坏时,受拉钢筋却仍处于弹性阶段。在这两类破坏形式中,受压钢筋都可能屈服
或者没有屈服。在实际的梁中,受拉钢筋总是会屈服的,而且在绝大多数情况下,受压钢筋
水平处的应变也较大,足以使受压钢筋也达到屈服强度。值越大及和越小,受压钢筋越可能
达到屈服。我们最好根据基本原理,对各种不同情况分别进行数值计算,而不是去推导适用
于所有情况的一般公式。如果需要,在Mattock、Kriz和Hognestad的论文中就可以找到这个
一般公式。下面的例子时演示数值法。

\subsection{双筋截面的设计}

由于下述种种原因,在设计中有可能需要受压钢筋。

1.当采用高度较小的梁时,用求得的抗弯强度仍然有可能不满足要求。通过配置受压钢筋和附
加的受拉钢筋可以提高抗弯能力。这种情况在设计中是少见的。因为强度设计法所允许的值
比按另一种设计法(资用应力法)设计的梁的平衡配筋率大得多。例如,对于的梁,按强度设
计所得的为0.0278,而按资用应力设计所得的平衡配筋率为0.0128。因此,尽管在资用应力
设计法中受压钢筋常常是必须的,但是强度设计法中需要用受压钢筋来提高强度的情况是少
见的。

2.在设计中可以用受压钢筋来提高截面在达到抗弯强度时的延性。显然,如果在截面内设有受
压钢筋,受压区高度就会减少,这是因为内压力是由混凝土和受压钢筋一起分担的。因此,
配有受压钢筋的截面其极限曲率将比较大。

3.受压钢筋可以用来减小梁在使用荷载下的挠度。在使用荷载作用下,含有的单筋梁的混凝土
应力比较高。例如在例题4.2中,当时,按高度为最小的要求,用忽略徐变的弹性理论所设计
的单筋梁在使用荷载下其混凝土的最大应力已达,尽管这时钢筋的应力大致只有屈服强度的
一半。由于实际的应力图形是曲线型的,混凝土的实际应力会更小一些;但无疑的是混凝土
的应变将会比较大,因而挠度也就可能比较大。降低由混凝土所受的应力可能使挠度减小,
而这可以通过在截面内配置受压钢筋来实现。

受压钢筋还可以减小梁在使用荷载下的长期挠度,这是因为当混凝土开始徐变时,梁内的压
力有由混凝土向钢筋转移的倾向。这样,混凝土的应力就降低了,而由徐变引起的挠度也就
大大减小了。由混凝土收缩造成的曲率也由于受压钢筋的存在而减小了。

4.对于外荷载可能出现的各种组合所做的考虑往往都表明弯矩可能变号。这对于在垂直和水平
荷载作用下的连续框架的横梁来说是常见的情况。这样的构件都需要上下两面配筋以便承受
可能出现的拉力,因而相当于双筋构件。在截面抗弯强度计算中忽略受压钢筋的存在总是偏
于安全的。然而,在某些情况下,也可能需要计入受压钢筋的作用来精确计算截面的抗弯强
度。

\section{T形及工字形截面}

\subsection{T形截面梁的有效宽度}

当钢筋混凝土肋型楼盖为整体浇注时,梁和板将整体工作。

当梁承受正弯矩时,一部分板就相当于梁的翼缘,要承受纵向压力去平衡肋内钢筋的拉力。
如果梁与梁之间的距离较大,一半的弯曲理论显然就不能适用于这种情况。这是因为翼缘在
梁肋上方处较之它在末端处处于较高的应力状态,翼缘内的纵向压应力将随距离梁肋的远近
II变化。所示翼缘压应力的这种变化之所以发生,是由于翼缘内存在着剪切变形(剪切滞后),
这种剪切变形使得纵向压应变随着离开肋部距离的增大而减小。

用弹性理论可以计算出在弹性范围内梁中压应力的实际分布状态。它和截面与跨度的相对尺
寸以及荷载的类型有关。当构件达到抗弯强度时,纵向压应力沿其翼缘的分布状态将比弹性
理论所表明的更为均匀。这是因为混凝土的应力——应变曲线在接近最大应力时显示出应力随
着应变产生的变化比较小。而且,除此之外,由于支承在两根梁之间的荷载作用,这块板通
常都将在梁的横向受弯,而这就能在翼缘与肋连接处的翼缘顶面引起平行于梁的裂缝。这时
板内与梁横向相交的钢筋以及沿裂缝的剪切摩擦虽然还能使纵向压力传入翼缘,但仍有足够
的理由促使我们采用一个偏保守的、较小的翼缘有效高度。

在设计中为了考虑压应力沿翼缘的变化情况而采用一个有效的翼缘宽度是比较方便的。这个
宽度可能比实际的宽度小,但是可以考虑为均匀受力的。目前,规范所规定的一些有效宽度
值是以近似于弹性理论的方法为基础的偏于保守的估计值。对于对称T型梁,ACI318-71建议
所采用的翼缘有效宽度不得超过梁净跨的1/4,而且翼缘由肋部向每侧挑出的宽度不得超过8
倍板厚或到相邻的一根梁的一半净距。对于仅一侧有翼缘的梁,挑出翼缘的有效宽度不得超
过梁跨度的2/3,或6倍板厚,或到相邻的一根梁的一半净距。

当梁承受负弯矩时,翼缘内的一些纵向钢筋无疑将与肋上部的主筋一起作为受拉钢筋发挥作
用。这种钢筋的拉力是通过翼缘的剪应变由翼缘传入肋部的,就像在正弯曲情况下压力的传
递一样。规范没有规定某个有效宽度范围内的板中钢筋可以考虑为起受拉钢筋的作用。但是,
很明显,在实事求是地估算梁对负弯矩的强度时,理应计入板中钢筋的作用。作为一个近似
办法,似乎可以把梁肋每侧4倍板厚宽度的板中钢筋计入梁的受拉钢筋。

\section{梁的侧向失稳}

当采用窄而长的梁时,发生在抗弯强度充分发挥之前的失稳可能成为破坏的原因。失稳破坏
是伴随着有扭转的侧屈形式出现的。如果梁在弯曲平面内的抗弯刚度远大于它的横向刚度而
又处于缺乏侧向支承的状态,这种失稳就会是严重的。由于绝大多数设计者都自觉地选用比
较敦实的截面,这类问题是相当少见的。但在安装预制混凝土结构期间,危险的情况有可能
出现在给部件提供恰当的横向约束之前。

如果试图符合实际地估计钢筋混凝土受力性能的所有特点,侧向失稳问题的分析方法就会变
得复杂起来。由于现在还没有足够的实验资料赖以令人信服地对于从理论上推导出临界荷载
加以验证,这里就不试图用数量表示相关的参数。因此,这类失稳问题只能利用一些尚有争
论的假定来进行分析。

\chapter{构件承受弯曲及轴向荷载的强度}

\section{引言}

柱子是主要用来承受压力的结构构件。短柱是指它在一定偏心距影响下的极限荷载只受材料
强度和截面尺寸控制的柱子。而细长柱则是指它的极限荷载还要受细长度影响的柱子;这时,
由于柱的横向变形,细长度引起了附加弯曲。

混凝土柱是用纵向和横向钢筋来配筋的。横向钢筋一般是做成箍筋或间距较密的螺旋筋的形
式。

\section{轴心受压短柱}

混凝土的收缩和徐变对使用荷载下钢筋混凝土轴心受压柱中混凝土和钢筋的应力具有很大的
影响,它往往起着提高纵筋应力和降低混凝土应力的作用。对于配筋率很高、初始荷载很大
而随后又大量卸载的柱子,甚至可能在混凝土中引起拉力,在钢筋中引起压力。因此,要用
弹性理论和容许应力去估计钢筋混凝土柱的安全性是及其困难的。

另一方面,柱的极限荷载并不明显地随着加载的历程而变化。当荷载增加时,钢筋一般将在
混凝土达到最大强度之前达到它的屈服强度。但在这个阶段柱子还没有达到它的极限荷载。
柱子能进一步承担荷载,是因为在钢筋保持屈服应力的同时柱子的变形和荷载还在增长,直
至混凝土达到它的最大强度。表明了这种受力性能。另一种情况是,如果在钢筋屈服之前混
凝土的应力已趋近它的强度,正如在使用屈服强度很高的钢筋将会出现的那样,这时混凝土
进一步增大的变形会使钢筋达到屈服强度。因此,轴心受压钢筋混凝土柱的极限荷载(也许更
应称之为屈服荷载)就等于钢筋的屈服强度与混凝土的强度之和。人们已经发现轴心受压柱中
混凝土强度大约是,其中是圆柱体的抗压强度。柱中混凝土的强度稍低于圆柱体强度是因为
试件形状和尺寸的差异以及由于柱子垂直浇注而导致柱子顶部范围内的混凝土的离析和水分
增多。于是,轴心受压柱的极限荷载可以表示为:

直到荷载为止,箍筋柱和螺旋筋柱的表现几乎都是一样的,横向钢筋对柱子强度增长所起的
作用是非常小的。一旦达到荷载,箍筋配置得不密的柱子就会立即破坏,伴随着混凝土的压
碎和箍筋之间的纵筋压屈。这是因为箍筋的间距通常都太大了,不能防止一般混凝土的破坏
和钢筋的压屈。

螺旋筋柱在达到荷载后,其螺旋筋外面的混凝土外壳将开裂或破碎。柱的承载能力由于混凝
土面积的减小而降低。但是螺旋筋的间距一般都小到足以防止纵向钢筋压屈的地步,所以,
纵筋能继续承受荷载。随之而来的是大量增加的变形,使核心混凝土(由于内部破裂而倾向于
增大体积)向外挤压在螺旋筋上,从而导致螺旋筋对核心混凝土施加约束反作用力。由此产生
的径向压应力提高了核心混凝土的承载能力,以至尽管失去了混凝土外壳,配有大量螺旋筋
的柱子的极限荷载仍然能提高到比还大的程度。混凝土的强度由于螺旋筋的约束而提高已在
2.1.3节中讨论过了。

其中为量至螺旋筋外径的核心面积,对比表明,ACI的要求将会保证柱子外壳剥落之后的极
限荷载超过剥落前的荷载。螺旋筋柱所具有的巨大的延性是相当令人感兴趣的。即虽然箍筋
间距不密的轴心受压箍筋柱表现为脆性破坏,但螺旋筋柱却具有很大的塑性变形能力。

试验已经表明,密排的矩形箍筋也能提高受约束混凝土的强度和延性,虽然它不如环形螺旋
筋那么有效。这时由于来自混凝土的横向推力将使矩形箍筋的各边产生水平方向的弯曲,箍
筋就仅能在靠近截面的拐角处产生约束压力;而环形螺旋筋由于它的形状却能沿它的周边产
生均匀的约束压力。

\section{单向弯曲的偏心受荷短柱}

\subsection{引言}

轴心受荷短柱在实际应用中是很少出现的,因为几乎总有一些弯曲存在,这种存在已经从下
列现象中得到证实,其中有:轻微的初始弯曲、梁和板对柱子施加荷载的方式以及由连续结
构引起的弯矩等。

在实际应用中,从强度观点来说,箍筋柱和螺旋筋柱两者的设计都是假定混凝土不受约束,
但是由于螺旋筋柱的韧性较大,ACI规范对螺旋筋规定了一个与箍筋柱的承载能力降低系数
相比不那么太严格的承载能力降低系数。

在以下的几段中将导出假定混凝土无约束时偏心受压柱的强度公式。在极限荷载下,混凝土
达到了它的最大承载能力,但纵向钢筋却可能达到也可能达不到屈服强度。在推导强度公式
时采用了的假定。在这一节中只考虑了绕截面一个主轴的弯曲(即单向弯曲)。

\subsection{单筋或双筋矩形截面的分析}

一个偏心加载到极限荷载的双筋矩形截面示于中。这里是考虑中和轴的深度小于整个截面的
高度。正如梁的情况以下,也可能出现受拉破坏或受压破坏,它取决于受拉钢筋是否达到屈
服强度。可是,不同于梁的是,不能用限制钢筋面积来防止出现受压破坏,因为破坏类型与
轴向荷载的高低有关。在极限荷载下,偏心受压柱的受压钢筋一般都能达到屈服强度,除非
荷载较小、或使用了高强钢筋、或柱截面尺寸较小以致使尺寸相对较大。通常是假定受压钢
筋屈服,然后再验算它是否已经达到了屈服应变。

能导致一个给定的柱截面发生破坏的各种组合最好是用一条相互作用曲线来表示。就是用于
一根典型偏心受压柱的这样一条曲线。凡是能给出AB线上任意一点的一组荷载和偏心距的组
合都将引起受压破坏,而BC线上任意一点所对应的组合则将引起受拉破坏,破坏时受拉钢筋
的屈服先于受压混凝土的压碎。平衡破坏发生在B点。凡是能由像话作用曲线以内的面积上任
意一点表示的荷载和偏心距的组合都能被承受而不致产生破坏;而在这个面积以外的点所表
示的组合就承受不起了。应当注意到适当大小的压力荷载能够提高截面的极限抵抗弯矩。当
时,前面导出的全然不能应用,因为这时中和轴已位于截面以外,应力图形的形状发生了变
化。这个观点可以用来说明。在这个图中给出了在极限荷载下的一个截面中与不同的中和轴
位置相对应的一系列应变分布图形。

此外,在这些公式中没有考虑减去由受压钢筋所占用的那一部分混凝土面积。这样造成的轻
微误差可以通过由受压钢筋的实际应力减去来克服,用以考虑这里曾经认为有混凝土承受这
个应力的事实。这也就是说,受压钢筋中的应力要取为,但若钢筋达到屈服。

\subsection{单筋或双筋矩形截面的设计}

在实际工程中所有的柱子都受有一定的、由初始弯曲和非对称荷载所造成的弯矩,所以轴心
受压柱不是一种实际情况;于是就建议不应把压力荷载作用的偏心距估量成小于某个最小值
(例如,对于箍筋柱不小于0.1h,对于螺旋筋柱不小于0.05h)。实际上,人们能够证明给所
有的柱子加上一个附加的偏心距来考虑那些不可预见的、可能使偏心荷载距增大的影响因素
是有道理的。

在柱子设计中常常不可能用限制截面尺寸的方法来排除受压破坏。所以,受拉破坏和受压破
坏的设计公式都是需要的。用于分析的公式在设计承载能力降低系数加以修正之后即可用作
设计公式。ACI318-71规定的柱子承载能力降低系数已在列出。这里要注意,对于在受拉破
坏范围内可以降低到零的较小轴向荷载来说,承载能力降低系数可以按线性关系提高,即当
极限荷载由大约下降到零时,对于螺旋筋柱可以从提高到;对于箍筋柱可以从0.7提高到0.9。
这里是柱截面的总面积。

\subsection{沿四周配筋的矩形截面}

当一个截面配有沿所有各边分布的钢筋时,由于整个截面上的钢筋可能处于不同的应力水平,
设计公式和分析公式的推到都变得困难了。对这样一个截面的分析可以用变形协调和平衡的
要求来完成。

\section{细长柱}

\subsection{细长柱的性能}

柱子的细长度可能由于柱子受弯所引起的侧向挠度而导致其极限荷载的降低。这种效应在得
到了具体说明。在这个图中所表示的是一根原来笔直的柱子由于荷载P以相等的偏心距e作用
于两端而使它产生单曲率弯曲的特定情况。柱子的弯曲变形使荷载在临界截面处的偏心距变
为$e+$,其中是在该截面处由侧向挠度造成的附加偏心距。因此最大的弯矩就增加到。这就
是通常的效应。由弯曲造成的侧向挠度的重要性取决于柱上加载的形式和柱的端部条件。在
这里称之为弯矩有时也叫做次弯矩,但是这个术语往往会意味着这个弯矩不太重要,其实在
某些情况下它可能是相当重要的。

短柱的定义是它的极限荷载不因弯曲变形而降低,这是因为附加偏心距可以忽略不计或者发
生在远离临界截面的地方。长柱的定义是它的极限荷载将由于附加偏心距所引起的增大了的
弯矩而降低。

所给出的柱子临界截面的相互作用曲线描述了在增长的荷载下的柱子的受力性能。如果附加
偏心距可以忽略不计,则各个荷载阶段的最大弯矩将都等于而且将随荷载的增大沿着一条线
性的轨迹发展。这就是短柱的受力性能,而且当它的轨迹到达相互作用曲线时就最终发生截
面的材料破坏。如果是一根细长柱,则最大弯矩将等于,而且由于在较高的荷载水平时增长
较快,它的轨迹就将发生弯曲。这时可能出现两种类型的细长柱受力性能。第一种是柱子在
具有侧向挠度时可能是稳定的,但因这时它的轨迹正好达到相互作用曲线,从而产生了截面
的材料破坏。这种类型的破坏通常是发生在无侧移的建筑所用的柱子中。第二种是如果柱子
非常细长,它就可能在它的轨迹到达相互作用曲线之前变成不稳定的构件。没有侧向支撑的
柱子就可能发生这种失稳破坏。

用细长柱的相互作用曲线还可以进一步说明在特定的荷载和端部条件下的细长柱的受力性能。
展示了由MacGregor等人所描述的这种曲线的作法。为所示那类的柱子的临界截面的相互作
用曲线,它描述了短柱和细长柱的受力性能。细长柱的无支长度与截面厚度之比为。细长柱
的破坏是发生在荷载和增大了的弯矩作用下的点,它破坏时的荷载和原始弯矩可以用A点来
表示。而与一系列值相对应的A点都是能够确定出来的,从而可以绘出所示的一簇曲线。这
些曲线给出了造成柱子破坏的荷载和原始弯矩。这些曲线可以用来表明在不同加载情况下由
于细长度引起的强度降低情况。

所表示的是由荷载引起了单曲率和双曲率弯曲的两端铰支无侧移柱。在这两种加载情况下,
弯曲变形都引起了附加弯矩,但是这里的附加弯矩并没有增大发生在柱端的最大原始弯曲。
不过,如果附加弯矩很大,最大的弯矩就可能由柱端移到柱的高度范围内。显然,在单曲率
情况下附加弯矩使最大弯矩增大的可能性比在双曲率情况下要大一些,因为在前一种情况下
侧向位移较大,而且在很大一部分柱长内原始弯矩都接近于最大弯矩。我们从中可以看出,
当两端偏心距相等而符号又相同时,极限荷载的降低最多;当两端偏心距相等但符号相反时,
极限荷载的降低最少。

若柱端侧移不受约束,则将在柱子两端引起最大的附加弯矩,而最大弯矩的增长就可能是很
显著的。弯矩在一根两端固定的有侧移柱中的增长情况示于。很明显,如果柱端的转动未受
到完全约束,而是弹性约束,就会产生一定的柱端转动;这时由于柔度的增加,侧向位移以
及由此而引起的附加弯矩就将增大。

框架柱的端弯矩取决于柱和梁的相对刚度。在加载过程中,梁和柱的刚度将由于混凝土的开
裂以及后来的非弹性变形而降低。柱的刚度还会由于柱的侧向挠度引起的附加弯矩而降低。
于是,柱中弯矩在加载过程中将由于挠度引起的附加弯矩和相对刚度的变化而发生变化。柱
中弯矩可能增大也可能减小。例如,对于无侧移框架中的一根短柱来说,由于刚度降低所造
成的柱端弯矩的降低量可能比由于挠度引起的弯矩增大量还大一些;于是柱的最大弯矩就将
减小,并导致承载能力的增加。但是对于无侧移框架中的一根细长柱来说,由挠度引起的弯
矩增长速度往往比约束弯矩更快一些,因而最大弯矩就将增大,并导致承载能力的降低。

在一榀能够产生侧移的框架中,如果梁是相当柔韧的,柱子就势必像刚体一样作用,而框架
便主要是由于梁的弯曲而产生侧向挠度。如果梁很刚劲,框架的侧移量就将更多地取决于柱
子的弯曲。在能够自由产生侧移的框架中,通过增大梁的刚度来提高对柱端转动的约束程度
也将能提高柱子的强度。但是,如果梁达到屈服,因而不能再起约束柱子侧移的作用,就将
形成一个不稳定机构。

上面对柱子受力性能所作的简单分析表明,影响细长柱强度的主要变量有以下几个:
\begin{enumerate}
\item 无支长度与截面高度的比值、柱端偏心比和两端偏心距的比值及符号。
\item 对柱端转动的约束程度。与柱相连接的梁系的刚度越大,柱子的强度也就愈大。
\item 侧向约束的程度。一根对柱端侧移不加约束的柱要比有不动水平支点的柱弱得多。
\item 配筋率和材料强度。这些因素影响着柱截面的强度和抗弯刚度。
\item 荷载作用时间的长短。在持续加载过程中混凝土的徐变会增大柱的挠度,因而通常都
  会降低细长柱的强度
\end{enumerate}

\subsection{细长柱的“精确”设计法}

受压构件的设计可以把从结构的二阶分析中求出的弯矩和轴力作为依据,在这个二阶分析中
应考虑构件的实际强度、挠度对弯矩和轴力的影响以及加载时间长短的影响。截面就可以按
能抵抗这些作用力来选定而不必加以修正,因为柱子细长度的影响已经在确定这些作用力时
考虑进去了。

在二阶分析中应该包括的主要因素是结构中柱子的侧向挠度所引起的弯矩。MacGregor已经
总结了进行这种分析的方法。结构可以理想化为由直杆组成的平面框架。这里必须采用实际
的弯矩——曲率关系来为挠度和附加弯矩提供精确的数值,而且还要考虑轴向力对受压构件转
动刚度的影响。这样确定出的最大弯矩中将包括框架的挠度和转角的影响在内。

采用这种分析方法来确定设计截面时所需要的柱中各项作用力是最合理的,但是由于它的复
杂性,这种分析需要编制好了备用的与此相适应的电算程序。

\subsection{细长柱的近似设计法:弯矩增大系数法}

如果是采用传统的以近似相对刚度和忽略构件侧向挠度影响为基础的一阶结构分析方法来确
定框架中的弯矩和轴力,就必须对这样求得的作用力进行调整以考虑二阶效应。于是截面就
要按能承受调整后的作用力来选定。在ACI318-71中为这个目的所规定的设计方法就是弯矩
增大系数法,它与美国钢结构学会规范中所采用的方法是类似的。

\subsubsection{方法}

弯矩增大系数法可以用的相互作用曲线来说明。设通过一阶弹性框架分析所求得的必须承受
的柱中极限荷载和弯矩为和,则设计截面时所应取用的轴力和弯矩即为。其中为弯矩增大系
数。

\subsubsection{受压构件的有效长度}

在设计中使用的有效长度系数中必须考虑柱端侧移受约束和转动受约束的程度。表示了柱端
无侧移或有侧移时的有效长度。

描绘了有侧移支承和无侧移支承框架的屈曲模式。在有抗侧移支撑的框架中,柱子的值范围
在0.5和1.0之间。在能够侧移的框架中,值总是要超过1.0。由于有支撑和无支撑框架的受
力性能是如此的不同,值通常就是按这两种类型的框架分别给出的,因而设计者也就必须判
断框架是属于有支撑的抑或是无支撑的情况。实际上完全无侧移和完全无侧向支撑的框架在
实际工程中都是难得出现的。ACI318-71的说明建议,如果在框架的某一层中包含有剪力墙
或其他类型的侧向支撑结构而其抗侧移的总刚度至少为该层所有各柱抗侧移刚度之和的六倍
时,则该层所有的柱子都可以认为是有抗侧移支撑的柱子。所以,这时设计者就必须运用他
的判断能力。


\chapter{受弯构件的极限变形及延性}

\section{引言}

钢筋混凝土构件在达到和超过极限荷载时的荷载——挠度性能如所示。图中对比了脆性和延性
这两种性质。由于下列一些原因,有必要考虑构件的荷载——变形特性:

1. 不应使构件产生脆性破坏。如果万一结构要受荷到破坏,它在接近最大承载能力的条件下
应能经受很大的挠度。这样就可能通过发出破坏警告并阻止完全倒塌来避免伤亡。

2. 在超静定结构设计中可能被采用的弯矩、剪力和轴向力的合理分布图形取决于构件在临界
截面处的延性。如果能够发生弯矩重分布,就可以得到一个不同于从线弹性结构分析中得出
的弯矩分布图形。这就是说,当接近极限荷载时,某几个截面可能在其他截面之前先达到所
能抵抗的极限弯矩;但如果这些部位能在保持极限弯矩不变的同时发生塑性转动,则在其他
各处弯矩也增大到极限值的过程中结构还能承受更多的荷载。结构的极限荷载是在产生了足
够多的塑性绞之后形成破坏机构的时候达到的。现在大多数规范都允许依据各处截面的延性
在设计中考虑一些弯矩重分布。利用弯矩重分布可以带来一些好处,因为它能降低连续构件
支座处钢筋的拥挤程度,而且还能减小弯矩包络图中的峰值弯矩。

3. 在有地震作用的地区,一个很重要的设计依据是结构在承受地震型荷载时的延性。这是因
为当今的抗震设计原则是依靠弹性后的变形来吸收和耗散能量,以求在大震中幸存下来。因
此,若要防止倒塌,那些不能以延性方式起作用的结构就必须按高得多的地震力来设计。

\section{弯矩——曲率关系}

在这一章中我们要讨论受弯构件在屈服时和在极限弯矩时的荷载——变形特性。这种特性主要
取决于截面的弯矩——曲率特性曲线,因为一般形状和尺寸的构件其大部分变形是由与弯矩相
联系的应变所引起的。由剪切或扭转引起的比较显著的附加变形将在第七章和第八章中讨论。

表示一个钢筋混凝土构件中的一个原来笔直的受有等量端弯矩和轴向力的单元。曲率半径量
至中和轴。由于在裂缝之间混凝土还要承受一些拉力,因此曲率半径、中和轴的深度、边缘
受压纤维的混凝土应变和受拉钢筋的应变都将是沿着构件变化的。若只考虑一个长度隗的很
短的单元并采用中的符号,则该单元两个端面之间的相对转角即可由下式给出:

曲率实际上是沿着构件长度变化的,这是因为中和轴是上下起伏的,而裂缝之间的应变也是
波动的。如单元长度小并跨过一条裂缝,则该处曲率即可按开裂截面处的应变和由计算得出。

如果在一根钢筋混凝土梁的弯矩增长到破坏的过程中在一个短的标距上测出了它的临界截面
处的应变,就可以按计算出那个截面的曲率,并从而得以求出该截面的弯矩——曲率关系。显
示了在受拉破坏和受压破坏的两根单筋梁上通过实测得到的这样两条曲线。在初始阶段这两
条曲线都是线性的,因而这时弯矩和曲率之间的关系即可由经典弹性方程给出,其中为截面
的抗弯刚度。随着弯矩的增大,混凝土的开裂降低了截面的抗弯刚度,而这时轻度配筋截面
的刚度降低程度比配筋量较大的截面要大一些。截面开裂后的性能主要取决于钢筋含量。对
轻度配筋钢筋得出的是一条到钢筋屈服那一点为止的实际上是线性的曲线。当钢筋屈服时,
曲率在弯矩几乎保持不变的情况下出现了很大的增长,弯矩由于内力臂的增大也缓慢上升到
最大值,然后下降。反之高配筋截面的曲线在混凝土进入应力——应变关系的非弹性部分时就
变成了非线性的,而这时除非混凝土受到密布封闭箍筋的约束,否则破坏就可能是相当脆性
的。如果混凝土没有受到约束,它就会在钢筋屈服之前当曲率还相对较小时被压碎,致使弯
曲承载能力立即下降。为了确保梁在实际应用中的延性性能,所采用的钢筋含量总要小于平
衡设计时的数值。

一根受拉钢筋屈服的实际的梁,其弯矩——曲率关系可以理想化为所示的三折线关系。第一段
到开裂为止,第二段到受拉钢筋屈服,第三段到混凝土可以应变的极限值。在很多情况下把
曲线进一步理想化为所示的、具有递减的近似程度的任何一种双折线关系也是足够准确的。
是一条表现了一次加荷性能的理想化原始曲线,一旦形成了裂缝,正如大多数梁在使用荷载
下将会出现的情况那样,从荷载为零到钢筋开始屈服时这一段关系曲线就接近于线性。的双
折线对于起初就已经开裂的梁是一种准确的近似曲线。

\subsection{理论上的弯矩——曲率关系的确定}

承受弯矩及轴向荷载作用的钢筋混凝土截面的理论上的弯矩——曲率曲线可以根据确定抗弯强
度时所用的同样假定推导出来。这就是假定受弯前的平截面在弯曲后依然保持平面,并假定
混凝土和钢筋的应力——应变曲线是已知的。利用这些假定并根据应变协调和力的平衡这两个
要求即可确定出一系列弯矩和轴向荷载相关联的曲率。

表示钢筋即混凝土的典型应力——应变曲线,其中为钢材的屈服强度,为构件中的混凝土强度。

与一个给定的轴向荷载水平相对应的理论上的弯矩——曲率关系可以用递增边缘受压纤维混凝
土应变的方法来确定。与每一个值相对应的、满足力的平衡条件的中和轴深度是通过调整给
定的值来决定的。这里要注意的是,在只是受弯矩作用的情况下,。然后就用这样求得的内
力和中和轴深度根据来去顶与那个给定的值对应的弯矩。通过对一系列递增的值进行上述计
算,即可画出弯矩——曲率曲线。这样一个计算过程是冗长的,如有必要最好用计算机来完成。

对于一个给定的最大混凝土应变来说,单筋截面的延性是随受拉钢筋含量的增加而降低的,
而受压钢筋的存在则显著提高了延性。

\section{无约束梁截面的延性}

\subsection{屈服弯矩和曲率以及极限弯矩和曲率}

在极限设计和抗震设计中,构件的延性通常都是表达为极限变形和开始屈服时的变形的比值。
下面就来估量弯矩和曲率在受拉钢筋开始屈服时以及在混凝土达到极限应变时这两种情况下
的相对应的数值。这时构件的受压混凝土将考虑为不受约束的。虽然在实际条件下无约束的
混凝土很少见到,但除非采取积极措施用适度密排的横向钢筋对它加以约束,否则一般还是
把它视为无约束的。

表现了一个双筋矩形截面在受拉钢筋开始屈服时和在极限混凝土应变时的一般情况。受拉钢
筋开始屈服时的曲率可以按钢筋屈服时的应变根据求得。对于所考虑的钢筋含量来说,当受
拉钢筋首次到达屈服强度时,混凝土边缘纤维中的应力可能还比圆柱体强度小一个相当的幅
度。混凝土的应力——应变曲线在达到之前是近似线性的,所以如果在钢筋达到屈服强度时混
凝土的应力还没有超过这个值,则到中和轴的深度就可以用阿生导出的弹性(线性)理论公式
来计算。一旦确定了中和轴深度系数,则钢筋与混凝土的压力值及其压力重心也就能求出来
了。所以根据,规定在开始屈服时的弯矩和曲率的几个公式即为:

若混凝土边缘受压纤维的应力大于大约,则受拉钢筋开始屈服时的中和轴深度就要用实际上
是弯曲的混凝土应力——应变曲线来计算了(该曲线可相当近似地取为抛物线)。不过,即使计
算出的应力和一样高,还是可以用直线公式得出一个估计值。






\chapter{受剪构件的强度及变形}

\section{引言}

对钢筋混凝土受弯构件受力性能所做的广泛研究已经把弯曲破坏机理澄清到了如此程度,以
致一些被充分理解的结论目前已经纳入了许多国家的设计规范。可是在理解和定量估计受弯
兼受剪构件性能方面的进展却不太引人注目。数以百计的公开出版物,其中大多数是在近十
五年内发表的,都谈到了这个问题的复杂性。

绝大多数钢筋混凝土结构构件都无法避免必须抗剪的问题。剪力很少单独作用于结构构件,
更多的是与弯矩、轴向力或许还有扭矩共同作用。因此,除了要确定剪力单独作用时的效应
之外,尚需讨论它与结构上的其他作用力之间可能存在的相互影响。特别是在受弯构件中,
抗剪机理与混凝土和埋入钢筋之间的粘结力以及钢筋的锚固都是相互紧密联系着的。

钢筋混凝土梁中的剪力传递在很大程度上是依赖于混凝土的抗拉和抗压强度的。那么剪切破
坏通常都是非延性的这一情况也就不足为奇了。因此就必须防止这种破坏。特别是在抗震结
构中要像其它有关章节中所阐述的那样把着重点放在延性这个问题上;为此设计者就必须保
证绝不发生剪切破坏。这就意味着当延性是必不可少时,构件的抗剪强度就应略高于构件可
能发挥出的最大抗弯强度。

在处理钢筋混凝土构件时使用各向同性匀质弹性体中剪应力的古典概念任然是合适的。只要
适当加以修正,这种弹性理论方法就能对裂缝形成和强度给出令人满意的预测。可是随着裂
缝的发展就产生了极为复杂的应力格局,而目前许多常用的公式却与这一时刻构件的实际性
能几乎没有关系。不过,广泛的试验工作,尤其是近几年来的试验工作,已经把识别各种抗
剪机理的理论知识大大推进了一步。本章将对此进行比较详细的讨论。

\section{剪应力的概念}

作用在结构构件任一横截面的横向力或剪力都能由平衡条件求出。用"剪力图"来表示这种力
的大小是比较方便的。作用在这样一个横截面中的剪应力之和自然应该与作用于该截面处的
外剪力相平衡。通过对构件内一个微元体建立平衡关系,即可明显看出每个单元体上的垂直
及水平剪应力的大小应该相同。

在一根各向同性的、匀质的、尚未开裂的梁内,沿任一纤维的水平剪应力可以很容易地从弯
曲应力的内部平衡条件求出。

\section{无腹筋钢筋混凝土梁的抗剪机理}

\subsection{斜裂缝的形成}

在钢筋混凝土构件内,弯曲和剪切联合造成了一个双向应力状态。这样引起的主应力。当主
拉应力超过混凝土的抗拉强度时,就会形成裂缝。在弯矩大的区域内,构件受拉边缘纤维处
的主拉应力最大,它是造成垂直于构件轴线的弯曲裂缝的原因。在剪力较大的部位也有可能
在大致与梁轴成的方向形成相当显著的也称为斜拉力的主拉应力。它可能导致斜向(斜拉)裂
缝。这些斜裂缝都是由弯曲裂缝延伸而成的,极少例外。只有在相当特殊的情况下,例如在
有翼缘梁的腹板中,斜向裂缝才是由中和轴附近开始出现的。在评价开裂后的受力性能时如
果不把开裂后混凝土中复杂的应力分布加以考虑,主应力的概念就几乎没有什么价值了。钢
筋混凝土受弯构件在斜裂缝形成之后不是立即破坏,就是在已开裂的梁中形成一个能承受更
多荷载的全新的抗剪机构。

由弯曲和剪切所引起的斜向开裂荷载通常比按主应力分析和混凝土抗拉强度所预料的要小得
多,这主要是由于收缩应力的存在、剪应力在弯曲裂缝之间的重分布以及横向钢筋对横截面
的局部削弱,这种削弱造成了沿梁长方向有规律的不连续性。

在钢筋混凝土设计的早期阶段,斜向开裂曾被认为是不合乎需要的。可是现在却已公认在使
用荷载下斜向开裂是可以容许的,只要裂缝宽度保持在与受弯时所容许的相同的限值以内。

\subsection{在梁剪跨内的平衡}

展示了一根简支梁的一部分,在它上面作用的剪力是个常量。这里可以辨认出对这个以斜裂
缝为界脱离体保持平衡的内力和外力。由图可见,外部作用的总横向力是由
\begin{enumerate}
\item 受压区的剪力;
\item 由抗弯钢筋跨过裂缝传递的销栓力;
\item 由骨料颗粒的咬合作用沿整个斜裂缝传递的斜向剪应力的垂直分力
\end{enumerate}
这三者所共同抵抗的。

为了简化对平衡的叙述,我们假定可以把骨料咬合作用传递的剪应力汇总成一个单个力,它
的作用线通过该截面上两个特殊的点。由于这一简化,的力多边形就体现了脱离体的平衡条
件。这一条件可以表示为下列形式:

重要的是要注意到中相互关联的弯矩和拉力并非发生在梁的同一个截面处。可以看出,在据
支座处抗弯钢筋中的拉力是由据梁支座处的弯矩所决定的。钢筋应力的增加显然是取决于理
想化的斜裂缝的倾角。这一情况在确定抗弯钢筋的截断点时是必须考虑的。对这种拉力分布
的移位现象还要在考虑附加的作用时进行更详细的探讨。

\subsection{主要的抗剪机理}

如果把给出的外弯矩与内抵抗弯矩之间的关系和众所周知的剪力与弯矩沿梁长的变化率之间
的关系联系起来,就可以得出下列形式的内抵抗剪力:

适中这一项表达了理想的等截面受弯构件的性能。在这类构件中以不变的内力臂作用的内拉
力是沿着梁长逐点变化的,以便能与外弯矩的大小恰好保持平衡。内拉力沿梁长的变化率这
一项被称为作用在单位梁长抗弯钢筋上的粘结力。如果内力臂保持不变(这是等截面受弯构
件弹性理论分析中一个通常被承认的假定),因而,则即可得出理想的"梁作用"方程为:

与此相同的结果是曾经按的形式得到过的,在那里是把在抗弯钢筋及其以上的各个水平处的
构件单位长度上的粘结力称为剪力流。很明显,只有当剪力流或粘结力能在抗弯钢筋与它周
围的混凝土之间有效地传递时,才可能对构件性能作这种简化。这样就产生了将要在下一章
中探讨的粘结现象。半个多世纪以来一般都相信在没有腹筋时就是通过''梁作用''这样的方式
来抗剪的。

当由于某种原因使钢筋与混凝土之间的粘结沿剪跨全长遭到破坏时,拉力就不能产生变化,
因而。在这种情况下,外剪力只能由内部的斜压力来承担。这一极端情况可以称为"拱作用"。
它的抵抗剪力就由右端第二项来表达,即

这里用内压力替换内拉力,是要表示平衡外剪力的就是一个倾角不变的压力的垂直分力。

在通常的钢筋混凝土梁中,当由于滑移、开裂以及其他原因而使”梁“所需的粘结力不能全
部发挥出来,将由所表示的两种机构联合提供对剪力的抗力。其中每一种机构在不同水平的
外荷载强度下所提供的抵抗剪力值则取决于伴随着这两种作用而产生的变形的协调性。

\subsubsection{在梁跨内的梁作用}

由荷载在一根简支梁上产生的裂缝将把受拉区分成许多块体。每一个块体都可以看做是象悬
臂一样起作用。悬臂的根部连在混凝土受压区,而自由端刚好在弯曲受拉钢筋之外,由于相
似的缘故,就把这些块体称作"混凝土悬臂"。

在已经表明,为了使理想的梁作用得以实现,全部粘结力都必须有效地承担起来。此外,还
需要看一看混凝土悬臂怎样才能满足这个要求。只要我们首先验明了一个典型悬臂所受的全
部作用力,就可以对这种抵抗作用进行更详细的分析。悬臂上的作用力有以下组成部分:
\begin{enumerate}
\item 由抗弯钢筋的拉力在两条相邻裂缝之间的增量所产生的粘结力。
\item 如果裂缝的两个面发生剪切位移,就可以通过骨料咬合作用产生剪应力。
\item 同样的剪切位移还可以引起通过抗弯钢筋截面的销栓力。
\item 在悬臂的“嵌固”端促成了一个轴向力、一个横向剪力和一个弯矩来平衡作用在悬臂
  上的各个力。
\end{enumerate}

这里可以看出,由粘结力施加于悬臂的弯矩除了由混凝土的弯曲抗力抵抗之外,销栓力和骨
料咬合力也起抵抗的作用。通过试验以使我们有可能定量地比较这三种形式的悬臂抗力。混
凝土的弯曲抗力在很大程度上取决于混凝土的抗拉强度和由这三个作用力产生的应力分布图
形以及关键性的悬臂截面高度。高度通常相当小,尤其是在裂缝充分发展的阶段。试验表明,
在常用尺寸的梁内可以由混凝土悬臂“嵌固端”的抗弯来抵抗的粘结力最多为20\%。

当沿斜裂缝发生剪切位移时,一定数量的剪力就将以抗弯钢筋销栓作用的方式传递。在钢筋
抵压混凝土保护层的部位,销栓力的作用要受到混凝土抗拉强度的限制。一旦产生了劈裂裂
缝,销栓的刚度以及它的有效性都将急剧降低。这种劈裂还对钢筋的粘结性能产生不利影响。
而混凝土的抗劈裂强度又将取决于同一层的各根钢筋之间赖以承受拉力的那部分混凝土有效
面积的大小。特别重要的钢筋是浇筑混凝土时的相对位置。由于顶部钢筋下面的沉积效应较
大,积水较多,所以它就需要由比梁的底部钢筋大得多的剪切位移才能提供相同的销栓抗力。

试验表明,在无腹筋梁内的销栓力所起的作用不超过悬臂总抗力的25\%。但在使用箍筋时,
销栓作用将更加显著。这是因为抗弯钢筋能更有效地挤压在紧紧绕过它的箍筋上。尽管如此,
在箍筋能支承销栓力的作用之前,裂缝还是将在与抗弯钢筋大致平行的方向上产生。销栓机
构的刚度很大程度上取决于裂缝和它相邻的、将能支承销栓力的箍筋之间的相对位置。对无
腹筋试验梁的实测表明,作用在混凝土悬臂上粘结力由50\%到70\%是由骨料咬合机构来承受
的。

在邻近破坏时,梁作用的这三种机构(销栓作用、骨料咬合和悬臂嵌固端的抗弯强度)的最大
承载能力不一定成叠加关系。斜裂缝向受压区的伸展使得悬臂“嵌固端”显著减小。这样就
导致了很大的转动,尤其是在悬臂的“自由端”。这意味着销栓承载能力将被耗尽。销栓裂
缝和靠近钢筋的次生斜裂缝的形成将对此时承担着大部分荷载的骨料咬合作用产生影响。悬
臂一侧骨料咬合力的突然减小会引起不平衡,除非能在悬臂根部产生一个相应的拉力。而这
样的拉力通常将导致裂缝的进一步发展,这在细长的梁中是制止不住的。这样的破坏叫做斜
拉破坏。由于它的发生通常都非常突然,因此就特别不合乎需要。

我们通常所谓梁的受压区的抗剪强度,是以假定骨料咬合和销栓作用是不可靠的抗剪方式为
前提的。然而近来的经验再一次表明,这个假定是不符合事实的。泰勒曾对斜裂缝上面的混
凝土受压区作过研究,并发现这块面积上承担的剪力只是缓慢地增长到梁接近破坏时那个截
面上总剪力值的25\%到40\%。因此,剪力的其余部分一定是由梁的中和轴以下的受拉区来承
担的。在骨料咬合及销栓机构破坏之后,受压区除了由弯矩产生的压力之外,一般已无力承
担继续增长的剪力,于是梁便破坏了。

\subsubsection{在剪跨内的拱作用}

如所示,的第二项表示剪力能够由梁中的斜压力来承担。拱作用需要在支座处有一个相当大
的水平反力,在简支梁中这个水平反力是由抗弯钢筋来提供的。这就对钢筋的锚固提出了严
格要求,而且它的确是造成拱最常见的破坏形式的原因。在理想化的梁内,假定有充分的锚
固。于是在底部钢筋中就能沿全长建立起所需要的不变的拉力。途中画阴影线的面积表示受
压混凝土的范围,在它以外之处就可能形成裂缝。通过考虑应变协调要求以及假定应变沿整
个混凝土截面按线性分布即可确定出压力线的唯一位置。钢筋在两个锚固端之间的总伸长必
须等于在同一高度处纤维混凝土的总伸长。在混凝土开裂处,可以根据受压区的应变用线性
外插求得伸长值。在满足这些判别条件的情况下,就能确定出钢筋与其周围混凝土之间的相
对平移。展示了滑移沿剪跨的典型分布情况。

在这种理想化梁的研究中出现的以下三点是值得注意的:
\begin{enumerate}
\item 拱作用只能在出现滑移的情况下(即完全丧失粘结传递能力时)产生。
\item 完全的拱作用所需的相对平移值是朝荷载作用点的方向逐渐加大的,并达到了与剪跨
  内钢筋的总伸长大致相等的数值。
\item 在荷载作用点附近压力线要比按一般弯曲理论预测的位置高得多,因此中和轴也高得
  多。
\end{enumerate}

在实际的梁中,特别是当采用带肋钢筋时,钢筋和混凝土之间不会发生明显的滑移;平移位
移的产生主要是由于斜裂缝之间形成的混凝土悬臂的弯曲变形或破坏以及在这些裂缝顶点以
上的受压区发生弯曲的结果。此外,在实际的梁中,从梁作用到拱作用的转变是逐渐进行的,
而且只要在试验梁中观察到拉力沿钢筋不断发展并因而得知内力偶臂长的变化,就可以作出
这种判断。由于伴随拱作用和伴随梁作用而产生的变形之间显著的不协调性,所以这两种机
构所能充分发挥的强度是不能叠加的。

能够从拱作用获得的强度在很大程度上取决于能否提供这样的斜向压应力。当钢筋拉力和梁
宽给定时,斜向压应力的大小取决于压力线的倾角。剪跨和梁高之比是这个倾角的一个尺度。
它还可以用弯矩和剪力表示如下:

如不将丧失锚固的情况考虑在内,则拱作用的破坏可以归纳为三类:
\begin{enumerate}
\item 在梁作用破坏以后,斜裂缝的延伸使受压区大大减小。有一个裂缝倾斜面要延伸到当
  荷载作用点附近混凝土的可用面积小到不足以承受该处压力时的高度,而混凝土也就被压
  碎了。这就是通常所说的“剪压破坏”。
\item 压力线的偏心可能太大,以致在“受压区”发生弯曲受拉破坏。破坏是非常突然的。
\item 当压力线较陡时(即当小于2时),由于拱作用更为显著,就可能得到相当可观的富余
  强度。而破坏则可能最终是由于斜向压碎或者劈裂所造成的。这里的劈裂可以用标准圆柱
  体的横向劈裂试验来比拟。这时梁的弯曲承载能力常常都能达到,因为拱机构是足以承受
  相应的剪力。
\end{enumerate}

重要的是要注意只有当荷载作用在梁的受压区时,无腹筋梁才会出现拱作用。而当主梁是在
接近其下边缘处支承次梁时,受力情况就可能比较危险。很明显,当外剪力传到受拉区时,
在梁内就不能形成有效的拱作用。上述资料清楚了表明了在压区加载的深梁中,拱作用一定
是起支配作用的抗剪形式。

\subsection{剪切破坏机理}

承受前面所描述的那类集中荷载的简支梁,其剪切破坏机理可以按比值的三个大致范围进行
划分。

型————当时,在斜向开裂荷载作用时或作用后不久发生梁机构破坏。接着出现的拱作用也没
有能力承担开裂荷载。

超过斜向开裂荷载后的受压区剪切受压破坏或弯曲受拉破坏。当时,这通常是一种拱作用的
破坏。

由混凝土的压碎或劈裂造成的破坏。

若如前述考虑抗剪的梁作用,就很清楚,在两条相邻裂缝之间传递的粘结力的大小将受到裂
缝之间形成的悬臂块体强度的限制。假定在等截面梁跨内每一个悬臂的强度均相同,即,则
梁作用所能产生的最大弯矩即为:

其中为单位梁长上的最大粘结力;为裂缝间距;为最大弯矩截面到支座的距离。如果这个弯
矩小于截面的抗弯强度,则对应于梁作用的抗剪强度将对梁的承载能力起控制作用。从可以
明显看出,由剪跨内形成梁作用时的混凝土悬臂所承担的弯矩系随离支座的距离的增加而增
大的。梁作用还意味着受限制的抗剪强度保持不变,它与剪跨对梁高的比值无关。

可是,梁跨内的较高配筋率将意味着在给定的荷载下弯曲裂缝较窄,而这就将使骨料咬合作
用和销栓作用能承担较大的荷载。这种由于抗弯配筋率较大而导致的梁作用强度的提高也已
为试验所证实。

\subsection{无腹筋梁的抗剪设计}

在前几页中已经讨论了承受集中荷载的无腹筋简支梁抗剪的本质。可以看出,梁的剪切破坏
机理,特别是当时,在很大程度上取决于混凝土的抗拉强度。所以,由显然相同的构件得出
的试验数据离散型较大就不是意外的了。对于沿受压边承受均布荷载作用的梁,得出的结果
要稍好一些。但另一方面,比值在连续梁所反映的情况与简支梁中遇到的情况并不相同,因
为考虑的截面与有反力作用的支座并不重合。基于这一原因,ACI便在大量试验结果的基础
上采纳了一个比较简单的半经验设计公式。用这个公式预测梁的抗剪强度在大多数情况下是
偏于保守的。它还通过下列形式考虑了影响抗剪强度的一些主要因素。例如由参数度量的混
凝土抗拉强度、用表达的对裂缝的控制以及剪跨与梁高的比值等。

实践证明,在所以的梁内,不论名义剪应力可能小到什么程度,为了确保梁在可能沿斜向开
裂之后不致立即破坏,比较好的办法是像建议的那样设置最少数量的腹筋。这一点很重要,
当抗弯钢筋含量较小时,也可能不再是保守的了。而且,构件中难以预计的轴心拉力也会降
低。

像这样的试图预测梁作用和拱作用这两种如此不同的机构的强度的一般表达式,不可避免地
会有缺点。但是到目前为止仍然没有可能合理地考虑影响抗剪机理每一个组成部分的所有因
素以及这些因素的相互关系。

\section{由腹筋钢筋混凝土梁的抗剪机理}

\subsection{腹筋的作用}

梁内有了像箍筋这样的腹筋之后,不会从根本上改变前面所描述的抗剪机理。作为梁作用的
主要元件的混凝土悬臂这时将像被拉住的悬臂那样工作。除去由骨料咬合、销栓和悬臂的弯
曲作用所承担的粘结力之外,还有一部分粘结力能够由习惯上所说的“桁架作用”来承担。
在这个桁架中,悬臂起斜向压杆的作用。

箍筋的存在在其他许多方面对梁作用也是有益的,它通过以下方式对抗剪机构的强度发挥作
用:
\begin{enumerate}
\item 改进销栓作用。箍筋能够有效地支承在它附近与弯剪裂缝相交的纵向钢筋。
\item 通过桁架作用产生的斜压力,抵消悬臂块体内的弯曲拉应力。
\item 限制斜裂缝在弹性范围内的张开程度,从而使由骨料咬合作用传递的剪力得以维持和
  增强。
\item 当箍筋布置得足够密时,能对混凝土提供约束,从而特别是在受拱作用影响的部位提
  高其抗压强度。
\item 当在锚固区内的销栓力和锚固力的作用而形成的劈裂裂缝时,能防止粘结遭到破坏。
\end{enumerate}

总之,可以说,做了适当的细部设计的腹筋将能保持梁的整体性,并因而能够保持前面已经
详述的梁机构的强度,从而使更多的剪力能由桁架机构来承担。

\subsection{桁架机构}

一个平行弦桁架与一个有腹筋混凝土梁之间的抗剪作用方面的相似性是混凝土结构的一个古
老的概念。这种由Morsch在本世纪初所假定的相似性的含义是,等效桁架的腹杆是由起拉杆
作用的箍筋和走向平行于斜裂缝的、通常是与梁轴成角的混凝土压杆所组成的。混凝土的弯
曲受压区和弯曲受拉钢筋构成这种比拟铰接桁架的上弦和下弦。桁架的内力只需要平衡条件
就能确定。这个桁架的性能与前已详述的“理想梁作用”相似到了如此地步,以致它能承受
分散于弯曲受拉钢筋方向上的各个虚拟铰节点处的粘结力,从而以不变的内力偶臂长来抵抗
变化的外弯矩。

在梁内伴随着梁作用或拱作用以及桁架机构而出现的变形之间是不协调的。这种按惯例不予
考虑的应变不协调性在趋近极限(即塑性)状态时越来越不显著。

中的比拟桁架描绘了腹筋以角倾斜于水平线的一般情况。它也可以用来表示由桁架所承担的
外剪力与各内力之间的关系。承受压力的斜压杆与水平线的夹角为。

斜压杆的坡度按照惯例都是假设与梁轴成的。但是已经观测到,在压杆边界处的斜裂缝是沿
梁长变化的。以应变能原理为基础所作的研究表明,压杆的最优倾角大约为。从可以明显看
出,当斜压杆的倾角变为小于时,由于有较大的箍筋穿过坡度平缓的裂缝,腹筋的需用量就
会减小。实际情况往往都是这样的,因而按斜压杆倾角为的设计公式便是偏于保守的。而另
一方面,在荷载作用点附近的斜压杆坡度却比较陡。但是这些部位的局部拱作用可以增强其
他抗剪机构的承载能力。一般地说,在混凝土强度高以及腹筋含量少,从而相当于受拉体系
刚度较小的梁内,压杆的倾角小于。因而这里的箍筋就比在压杆倾斜的桁架内更为有效。反
之,当梁的腹筋含量多而混凝土强度较低时,混凝土上的荷载就会以有较多的箍筋参与出力
为代价而被减轻。

平缓的斜压杆和坡度较陡的钢筋意味着混凝土的压应力比较大。这表明腹筋的含量不能无限
制地增大。在这种梁内由斜压引起的腹杆压碎可以导致剪切破坏。在估计梁腹的抗压强度时,
需要考虑下列另外一些因素:
\begin{enumerate}
\item 斜压杆在参与梁作用时还要承受弯矩。由于桁架内没有真正的“铰节点”而引起了次
  弯矩。
\item 穿过斜压杆的箍筋通过粘结力把拉力传给斜压杆,以致斜压杆往往普遍处于双向应变
  状态。我们知道,当同时作用有横向拉应变时,混凝土的抗压承载能力会大为降低。
\item 压力是在比拟桁架的“节点”处引起的,而这些压力沿腹板远远不是均匀分布的。因
  此就可能出现偏心及横向拉应力。
\item 有些斜杆的倾斜角度可能比小得多,这将导致斜压应力的显著增加。
\end{enumerate}

这些观察结果表明,有必要把混凝土的斜压应力值限制在适当低于混凝土抗压强度值的范围
内。基于这一原因,ACI就把桁架机构对抗剪强度所能起的作用限制为一个非常保守的数值。
于是,按所取的大小得出梁内的绝对最大名义剪应力应为。

箍筋只有在它被充分锚固时才能发挥出给定的强度。箍筋在其长度上的任何一点都可能与斜
裂缝相交。由于斜裂缝有可能非常接近于构件的受拉或受压边缘,所以箍筋在它的整个长度
内都必须能发挥出它的屈服长度。因此,很重要的是必须把箍筋弯绕过较粗的纵向钢筋并在
向前伸出一个足够的发挥长度。各本规范都规定了能令人满意的各种锚固形式。为了使桁架
作用显著,箍筋必须把荷载交代到“铰节点”或其附近。像在第十三章中扼要叙述地那样,
精心作出的细部构造设计应能保证这种荷载传递。如未能保证箍筋与纵向弯曲受拉钢筋的良
好贴合,则在箍筋转角处由于荷载集中传递就可能导致混凝土的局部压碎。已在一些构件中
观测到了大到的箍筋滑移。如果是在浅梁中,这样大的滑移就能使斜裂缝宽度大为增加。

有时,与一条伸展的斜裂缝相交的一批箍筋都屈服了,那条裂缝就要开始展宽而不受约束,
因而作为抗剪强度的一个重要组成部分的骨料咬合作用这时就会失去效用。这样损失掉的抗
剪强度是不可能转移到销栓和桁架机构上去的,因为它们的能力也已消耗殆尽了。因此,当
变形稍有增加时就会接着破坏。为了防止这种非延性破坏,实践证明较好的方法是确保(这
在抗震设计中实际上是强制性的)在构件的抗弯承载能力全部耗尽之前箍筋不要屈服。

梁的桁架机构只有在斜裂缝形成之后(即在混凝土内的斜拉力消失以后)才会起作用。箍筋的
主要作用是横穿潜在的破坏性斜裂缝传递横向(垂直)剪力。有人不时主张在梁的腹板内使用
钢筋网。它虽然既能承担水平力,又能承担垂直力,但是在抗剪方面并不是更为有效的。这
是因为在普通梁的梁腹中水平钢筋除了能帮助控制裂缝和增强销栓作用外并不能对承担横向
(垂直)力发挥什么作用。水平腹筋能增强“混凝土提供的抗剪强度”,但却影响不到桁架机
构的抗剪强度。然而对于深梁,正如在第十三章将要讨论的,在支座区域内锚固良好的附加
水平钢筋却能使拱作用显著增强。

\subsection{有腹筋梁的抗剪设计}

前面已经指出,只要裂缝宽度不是过大,无腹筋梁的抗剪机构,特别是骨料咬合作用就将继
续起作用。故在腹筋情况下只要腹筋应变并不很大(即箍筋尚未屈服),“梁作用”还是要承
受剪力的。所以,在箍筋屈服之前或开始屈服时,可以把这两种作用的强度如此叠加起来:

应当记住,在ACI的规定中是假定斜压杆的倾角是的。另一种可以采纳的观点是认为混凝土
提供的强度可以忽略不计,而且混凝土斜压杆的倾角是小于的,因此更有的箍筋跨过可能出
现的裂缝。对于大多数梁来说,用这两种方法预测的抗剪强度是很接近的。

在每根梁内,无论剪力大小如何,都必须配置至少与相当的最少腹筋用量。同时,为了确保
可能出现的每一条裂缝有效地为箍筋所穿过,箍筋的间距还不应超过。在可能形成塑性铰之
处,箍筋的间距不应大于。当弯曲受拉钢筋屈服时,由弯曲裂缝延续而成的斜裂缝也就必将
有所增长。在这些部位由混凝土提供的那部分抗剪强度应该忽略不计,而腹筋应按全部剪力
配置。

在预计弯矩可能反向时,例如在地震荷载作用时,设置密排的封闭式箍筋就特别重要。这样
的箍筋在抗弯强度充分发挥的部位能为该处的受压混凝土提供侧向支点。由于斜向腹筋只在
一个方向有效,因此当荷载可能反向时就不应采用这类腹筋,除非是沿两个方向都设置斜向
腹筋。

在本章末尾给出了一个例题,用以具体说明上述原理的应用。

\section{弯曲和剪切的相互影响}

用配置了适当腹筋的普通钢筋混凝土梁进行的试验表明,剪力对弯曲承载能力没有看得出来
的影响。这使设计者得以忽略其相互作用,并对弯曲和剪切分别进行处理。可是前文的讨论
表明,在梁的剪跨内的弯曲、剪切、粘结和锚固之间的确存在着密切的关系。这一点通过沿
梁长对弯曲受拉钢筋的性能进行分析就能明显地看出来。当在极限弯矩作用下沿截面传递的
剪力很大时,混凝土和钢筋中的弯曲应变分布就可能受到影响。在深梁中剪力可能处在这样
一种支配地位,以致它能使按前几章所述的原理得出的弯曲承载能力不能充分发挥,从而控
制构件的强度。

\chapter{受扭构件的强度及变形}

\section{引言}

钢筋混凝土结构中的扭转现象常常是由于结构之间的连续性而产生的。由于这个原因,扭转
问题在本世纪前半叶相对来说就未曾受到足够的重视,而在设计中忽视扭转问题看来也没有
造成什么严重后果。在最近十至十五年内,研究活动的大幅度增长显著地提高了对这个问题
的理解。目前世界各地已经或正在对混凝土中扭转问题的很多方面进行探讨。

到目前为止,大多数关于扭转的规范参考文献都是以各向同性匀质弹性材料的性能中借用的
概念为依据的。现行的ACI规范第一次包含了关于扭转的设计建议。这些建议是以相当数量
的试验资料为基础的,但是在综合了更多的得自现代研究成果的资料之后,这些建议大概还
要进一步修改。

扭转现象可能是由于一阶效应或二阶效应的结果而产生的。一阶扭转情况是发生在外荷载不
能由扭转以外的另一种方式来承受的时候。在这种情况下,为了保持静力平衡而需要的扭矩
可以单独地确定出来。这种情况也可以称为平衡扭转。它主要是一个强度问题,因为如果抗
扭强度得不到满足,结果或其他部件就会破坏。

在超静定结构中扭转现象就可能是由于连续性的要求作为二阶效应而产生的。在设计中忽略
这种连续性就会导致过宽的裂缝,但不一定有更严重的后果。设计者常常都直觉地忽略这种
二阶扭转效应。支承板或次梁的框架边梁就是这种情况的典型。在具有刚性连接的空间结构
中几乎没有可能避免由于变形的协调性所引起的扭转现象。某些诸如

\chapter{使用荷载下的性能}

\section{使用荷载下的性能}

结构在使用荷载下的性能是设计中需要考虑的一个重要问题。若截面仅按强度要求进行选择,
就会存在一种危险,即对于预防倒塌虽将具有足够的安全度,但结构在使用荷载下的性能可
能不会令人满意。例如,在使用荷载下构件的挠度可能过大,或者混凝土的裂缝宽到不能接
受的地步。

按照欧洲的术语,结构应按几种极限状态进行设计,其中最重要的是极限荷载下的强度、使
用荷载下的挠度和使用荷载下的裂缝宽度。其他可能的极限状态还有使用荷载下的振动和疲
劳。设计的目的应是保证具有足够的安全储备以防止倒塌和防止在使用荷载下结构变得不适
于使用。

因此,为了作出满意的设计,就必须查明使用荷载下的挠度值和裂缝开展情况是否处于合理
的极限值之内。这种验算需要使用弹性理论。1971年的ACI规范强调了采用以强度为基础、同
时进行适用性验算的设计方法;但是规范也允许使用以弹性理论为基础的、对没有轴向荷载
的受弯构件在使用荷载下采用规定的容许应力的另一种方法。

本章研究的是使用荷载下性能,其中对用于确定构件在使用荷载下由于弯曲所引起的应力以
及用于另一种可供代换的设计方法的弹性理论进行了阐述,同时也扼要地叙述了关于使用荷
载下的挠度和裂缝宽度的计算方法。

\section{关于构件中弯曲应力的弹性理论}

\subsection{有效弹性模量}

在钢筋混凝土中应用弹性理论时遇到的一个比较大的困难是混凝土的非弹性性质。混凝土的
弹性模量取决于应力的水平和荷载作用的时间这两个因素。由公式表示的根据短期加荷试验
确定的弹性模量所给出的是在大约$0.5f'_c$时的割线模量。正如图所示。对于低速加荷的情
况,弹性模量将由于徐变而降低。

图表明混凝土在不变的轴向压应力作用下的徐变变形情况。当施加的应力不超过大约
$0.5f'_c$时,在一个给定的时期内所发生的徐变几乎与施加的应力成比例,所以在不变的混
凝土压应力$f_c$作用之下的徐变应变便可以表示为
\begin{equation}
  \label{eq:16} \epsilon_{cr}=C_t\frac{f_c}{E_c}
\end{equation}

式中$E_c$为在加荷的瞬间混凝土的割线弹性模量;$C_t$为混凝土的徐变系数。这个系数正
如第节所述,是一个与加荷时混凝土的龄期、加荷持续时间、混凝土的配合比、构件的厚度
以及湿度等有关的经验系数。注意,这里由于线性假定,$C_t$与应力的大小无关,故总应变
可由下式给出:
\begin{equation}
  \label{eq:18}
\epsilon_{tot}=\frac{f_c}{E_c}+C_t\frac{f_c}{E_c}=\frac{f_c}{E_c}(1+C_t)
\end{equation}而包括徐变影响的混凝土有效割线弹性模量极为
$\frac{f_c}{\epsilon_{tot}}$,并可由表示为
\begin{equation}
  \label{eq:19} E_{eff}=\frac{f_c}{\epsilon_{tot}}=\frac{E_c}{1+C_t}
\end{equation}当已知徐变系数时,通常均可在假定的应力不变的条件下用有效弹性模量来
建立应力和应变之间的关系。

在典型的设计条件下,经过很长时间加载后的徐变系数$C_t$的平均值为1.5到2.0,但也可能
出现较大的波动。第给出了一种把徐变系数作为对它有影响的许多变量的函数来计算的方法。

图表明,如果卸掉荷载,弹性应变就会立即恢复,而且有一部分徐变也会随着时间逐渐恢复。

\subsection{弹性理论的各项假定}

受弯构件的弹性理论采取了三项假定:
\begin{enumerate}
\item 受弯之前的平截面在受弯后仍保持平面;
\item 如裂缝已从边缘受拉纤维处开始出现,则混凝土的抗拉强度即可忽略不计;
\item 混凝土和钢筋的应力应变关系都是线弹性的。
\end{enumerate}

在强度理论中所作出的第一条假定已在作过讨论。

第二条假定的含义是,当边缘受拉纤维中的应力超过混凝土的抗折模量,混凝土就将向上裂
开到中和轴。这个假定是合理的,因为一旦开始开裂,受拉钢筋中以及受压混凝土中的应力
将由于应力重分布而显著增加,而且如果有的话也只剩下在裂缝尖端应力集中点与中和轴之
间的很少一部分混凝土来承担拉力。

第三条假定对于应力低于屈服强度的钢筋来说是正确的,对于在使用荷载应力下的混凝土来
说也是合理的。通过利用混凝土的有效弹性模量还可以把徐变的影响考虑进来;但是在确定
有效弹性模量时曾经采取过的徐变应变与作用力之间成正比的关系只有在混凝土的压应力不
超过圆柱体强度的一半左右时才是严格适用的。不过,对成正比关系的偏差只是在应力明显
较高的情况下才会变大。

在截面分析中,当中和轴的深度由于混凝土的徐变所引起的应力重分布而发生变化时,用含
有不变的徐变系数的混凝土有效弹性模量只能得出应力的近似值。这是因为有效弹性模量适
用于不变的应力状态,而整个横截面的应力大小和分布却是随时间变化的。不过,用有效弹
性模量还是会给出合理的应力近似值的。

如果需要一个考虑应力变化史影响的更加精确的方法,则在具备了徐变--时间数据及加荷史
资料的情况下采用徐变率法或叠加法。

\subsection{用内力偶法进行梁的分析}

\subsubsection{矩形截面}

描绘了一个在使用荷载阶段出现裂缝后的双筋矩形混凝土梁截面。混凝土截面的尺寸以及钢
筋的面积和位置作为已知量考虑。这个截面可以利用内力偶的概念以及应变协调和力的平衡
条件进行分析。

应变$\epsilon_c$、$\epsilon'_s$和$\epsilon_s$可以通过应力表达为如下的形式:
\begin{equation*}
\epsilon_c=\frac{f_c}{E_c}(1+C_t)\quad\epsilon'_s=\frac{f'_s}{E_s}\quad\epsilon_s=\frac{f_s}{E_s}
\end{equation*}式中$E_c$为混凝土(在加荷瞬间)的弹性模量;$E_s$为钢筋的弹性模量;
$C_t$为混凝土的徐变系数。从应变图形中的相似三角形关系可以得到
\begin{equation*}
\frac{\epsilon}{kd}=\frac{\epsilon'_s}{kd-d'}=\frac{\epsilon_s}{d-kd}
\end{equation*}通过代换公式中的$\epsilon_c$、$\epsilon'_s$和$\epsilon_s$,即可求
得下列钢筋应力的计算公式:
\begin{equation}
  \label{eq:27} f'_s=\frac{kd-d'}{kd}nf_c
\end{equation}
\begin{equation}
  \label{eq:24} f_s=\frac{1-k}{k}nf_c
\end{equation}其中
\begin{equation}
  \label{eq:28} n=\frac{E_s}{E_c}(1+C_t)
\end{equation}这个比值$n$通常称为模量比。

当把受压钢筋占去的微小混凝土面积的影响忽略不计时,混凝土中的总压力即为
$0.5f_cbkd$,而内力的平衡方程便可以写成
\begin{equation}
  \label{eq:29} C_c+C_s=T\qquad\qquad 0.5f_cbkd+f'_sA'_s=f_sA_s
\end{equation}
把表示的钢筋应力代入就得出
\begin{equation*} 
0.5f_cbkd+\frac{kd-d'}{kd}nf_cA'_s=\frac{1-k}{k}nf_cA_s\\
0=k^2+2k(\rho+\rho')n-2(\rho+\rho'\frac{d'}{d})n
\end{equation*}
式中
\begin{equation*} \rho=\frac{A_s}{bd}\qquad \rho'=\frac{A'_s}{bd}
\end{equation*}解$k$的二次方程得
\begin{equation}
  \label{eq:30}
k=[(\rho+\rho')^2n^2+2(\rho+\rho'\frac{d'}{d})n]^{\frac{1}{2}}-(\rho+\rho')n
\end{equation}混凝土三角形应力图形的形心到边缘受压纤维的距离为$kd/3$;因此,混凝
土的合力对受拉钢筋的内力臂为$d(1-\frac{k}{3})$,其中$k$由给出。

用内压力对受拉钢筋取矩便可求得截面的抵抗力矩为:

\section{挠度控制}

\subsection{挠度控制的必要性}

由于高强度钢筋和混凝土的应用以及强度设计法的引用,现在已有可能设计出更加细长的结
构构件;而且现代建筑结构又往往不设实体墙或隔墙,非结构部位可能由于结构构件的变形
而易于损坏。因此,受弯构件在使用荷载下的挠度控制就呈现出较大的重要性。

ACI的435委员会关于允许挠度的一篇报告把挠度的影响划分为以下四大类。

\subsubsection{感觉上的可接受性}

感觉上的可接受性往往是一个因人而异的主观判断因素,而且在很大程度上取决于使用者的
社会经历和结构类型。在这一类中包括了象挠曲的梁和下垂的悬臂梁那样的视觉效果,象活
荷载或风的动力效应所引起的振动那样的触觉效果以及振动产生的噪音那样的听觉效果。要
根据感觉的可接受性来规定挠度的极限值时困难的,因为个人的见解纷杂不一。

\subsubsection{结构的适用性}

适用性的界限与结构的预定用途有关。属于这个范畴的例子有应能排水的屋面、应保持平面
的地板(如健身房)以及支承敏感的仪器设备的构件。适用性的挠度极限值是比较容易确定
的。

\subsubsection{对非结构部件的影响}

为了避免诸如墙壁、隔墙和天花板等非结构部件发生开裂、压碎、鼓出以及其他类型的损坏,
必须对挠度加以限制。挠度也不应妨碍可动部件如门、窗等的正常开启。在这里不仅时由于
重力和横向荷载引起的挠曲变形而且温度和收缩的效应也可能时重要的。而挠度的极限值则
取决于非结构部件的类型和安装的方法。

\subsubsection{对结构构件的影响}

为了使结构的性能不致出现与设计假定不同的情况,也有必要对挠度加以限制。在这个范畴
中包括了能造成失稳的挠度(例如在拱、壳体或长柱中);能造成应力体系变化的挠度(例
如由梁端转动引起的支承面积的变化)以及能造成应力增大的动力效应的挠度(例如由动荷
载引起的共振)。因此如果有可能就应在构件设计中考虑挠度对结构性能的影响。

\chapter{粘结与锚固}

\section{引言}

\subsection{需要考虑的基本问题}

由于外荷载很少直接作用于钢筋,钢筋就只能从它周围的混凝土获得它所承受的那一部分荷
载。“粘结应力”是给钢筋和混凝土界面上的剪应力所定的名称,它在钢筋与其周围混凝土
之间起着传递荷载的作用,从而使钢筋的应力发生变化。这种粘结在得到有效发挥时就能事
两种材料形成一种组合结构。结构构件中配筋细部设计的最主要目的就是要能获得良好的粘
结性能。

粘结力的大小是用钢筋中力的变化率来衡量的。在任意两个截面之间的钢筋应力如果没有变
化,粘结应力就不存在。粘结应力通常是定义为钢筋表面单位面积上的剪力,即用下式表示:

过去在只采用光面钢筋时,粘结强度曾经是一个更加严重的问题。而表面变形的钢筋却能提
供附加的粘结强度和安全成分。但是在另一方面,变形钢筋的性能,特别是当采用高强钢筋
和粗钢筋时,却又带了一些新的问题。这就需要对传统的粘结原理重新进行研究。

由于现行规范的要求全然是以实验为基础的,因此本章就不去讨论一系列设计规定的详尽背
景情况,可是设计者必须对那些能够对结构性能产生关键性影响的粘结锚固特性有所认识。
所以,对这类问题就要在一定的深度上加以探讨,为的是使设计者能以效果较好的配筋细部
设计。

钢筋混凝土构件中的粘结应力产生于性质不同的两种情况,即钢筋的锚固和由于弯矩沿构件
长度的变化而引起的钢筋内力钢筋应力沿其长度的变化。

\subsection{锚固粘结或使钢筋发挥效力的粘结}

在任一截面处需要承受一定内力的钢筋必须延伸至该截面以外距离为处。这个也就是能通过
粘结把钢筋的内力传给混凝土所需要的距离。如果规定了平均粘结力为,并假定它均匀分布
在这一长度上,则由平衡条件即可得出如下关系:

\subsection{抗弯粘结}

这一公式表明,当外弯矩的变化率(即剪力)较大时,弯曲粘结应力也将表现出较高的数值。
然而你,大体是过分地简化了这种情况,它甚至连近似地预测实际粘结应力的大小也未做到。
这时因为沿构件长度方向每隔一定距离就有混凝土裂缝存在,它导致由于裂缝之间的混凝土
承担拉力而引起的附加粘结应力。所以,即使当剪力为零时,也将形成粘结应力。但是已经
观察到,只要钢筋具备了足够的锚固长度,起源于抗弯粘结应力的破坏就不致发生。而对抗
弯粘结情况的考虑是要求在构件弯矩为零的部位(简支支座及反弯点处)验算锚固长度,因为
在这些部位受拉钢筋的面积可能较小而剪力则可能较大,从而导致较高的粘结应力。


\section{粘结抗力的本质}

\subsection{粘结抗力的基本特征}

通常都认为光面钢筋的粘结抗力就是砂浆膏与钢筋表面之间的化学附着力。可是,即使较低
的应力也会引起足够的滑移而使混凝土和钢筋之间的附着力遭到破坏。一旦发生滑移,进一
层的粘结抗力就只能靠摩擦和在钢筋与其周围混凝土之间已经移位的小砂粒的契固作用来产
生了。摩擦阻力取决于钢筋表面的状况。光面钢筋在经受标准加荷试验时,破坏是发生在附
着力和摩擦阻力被克服的时候,而且钢筋通常是被从它埋入的混凝土中拔出的。

带肋钢筋因其肋条与周围混凝土互相咬合而具有大为提高了的粘结能力。在钢筋两个肋条之
间,发挥出的粘结强度与下述各种应力有关:
\begin{enumerate}
\item 由于沿钢筋表面的附着力而产生的剪应力;
\item 对肋条侧面的压应力;
\item 作用在相邻两个肋条之间混凝土圆柱面上的剪应力。
\end{enumerate}

在两个肋条中点之间的一小段钢筋长度范围内,这些应力与要通过粘结传给混凝土的力之间
的关系可以根据简单的平衡条件得出如下:

当荷载增大时,沿钢筋表面的附着力必然要遭到破坏,而余下的摩擦抗剪强度与产生于肋旁
的承压强度相比又非常小;所以实际上就可以忽略不计。其余两个能产生粘结力的重要组成
部分之间的关系即可简化如下:

Rehm已经成功地把粘结问题的几个方面与几何参数联系起来。他发现,当左右时,埋入混凝
土中钢筋在一小段长度上表现出最令人满意的性能。当肋过高和相距过近时,剪应力就会控
制粘结性能,而钢筋将被拔出。当肋的间距约大约10倍肋高时,局部压碎的混凝土可能在肋
前形成一个楔块,因而破坏便通常是由周围混凝土的劈裂所引起的,由于肋条前面的混凝土
处于被约束状态,它能承受数倍于圆柱体压碎强度的挤压应力。这两种与肋条有关的破坏机
理已在显示出来。显然,带肋钢筋的几何形状必须使剪切拔出破坏不能发生。

粘结性能最重要的一个方面就是它对裂缝开展的影响。这一点与某一特定类型的钢筋在不同
情况下的粘结滑移特性有密切关系。一般来说,与可用的粘结力同时产生的滑移愈小,粘结
的质量也就愈好。

\subsection{周围混凝土中的应力状态}

在前面几节中我们探讨了在带肋钢筋一根肋条的紧邻部位经常出现的情况。为了使一根钢筋
的强度得以充分发挥,跨越许多根相邻肋条的那些粘结力都必须得到传递。因此,周围混凝
土中的应力状态是沿着埋入的钢筋而变化的,并且影响着粘结性能。这样就不能简单地根据
一定数量的单根肋条的粘结强度的总和来获得钢筋的粘结强度或锚固强度。

围绕于带肋钢筋周围的混凝土中的这种应力可以导致混凝土的裂缝和变形,如所示。这里传
给混凝土的粘结应力使混凝土保护层偏心受拉。这样形成的应力所造成的混凝土变形在主要
裂缝附近有把混凝土拉离这根钢筋的趋势。而当钢筋与砂浆之间的附着粘结抗拉强度到达之
时,周围的混凝土便与钢筋分离开来。这时还会形成许多可能扩展不到混凝土外表面的内部
次生裂缝。对于光面钢筋来说,在钢筋与混凝土发生分离之处,一颗预料粘结应力会完全消
失。而当为带肋钢筋时,这个区域的粘结力就必须全靠肋条的承压来传递。

当混凝土在一条主要裂缝处使它自己从钢筋周围分离开来时,原先与钢筋接触的混凝土表面
的周长就要增大,从而引起环向拉应力。这类应力会导致劈裂裂缝。

当接近于传递粘结力的极限能力时,肋条前面就会出现压碎现象。被压实了的混凝土粉末在
肋条前面延伸的距离直至达到肋高的三倍,并形成一个扁平的楔子。它往往会把混凝土进一
步从钢筋推开。因此就引起了附加的环向拉应力,而这种拉应力可能导致劈裂破坏。

围绕某一根个别钢筋的混凝土,由于参与了其他的结构作用,还可能受到除粘结作用产生的
应力以外的其他应力的作用。在建筑物框架横梁的交叉点处要引起横向作用于钢筋的压力或
拉力。同样地,在支承连续板的梁中,顶部钢筋周围的混凝土中也可能招致横向拉力。这样
的横向拉应力有可能导致沿主筋过早开裂,并对钢筋的粘结性能产生不利影响。相反地,横
向压力则可能对埋入混凝土中的钢筋提供有利的约束作用。

\subsection{劈裂破坏}


在大体积混凝土中具有足够埋入长度的地方,标准的变形钢筋是没有可能产生粘结破坏(拔出
破坏)的;相反,这根钢筋却会在它的加荷端发生断裂。可是在大多数结构部件中,围绕一根
钢筋或一组钢筋的混凝土面积却相对较小。在这种情况下通常的破坏形式就是劈裂,因为周
围的混凝土承受不了那里的环向拉应力。

特别严重的情况是发生在梁的剪跨内,在那里有可能由于下列现象的综合作用而沿抗弯钢筋
引起劈裂:
\begin{enumerate}
\item 在每条弯曲裂缝附近形成的环向拉应力;
\item 当由较大的粘结力需要传递时,由表面肋条的楔入作用和肋条前面的受压混凝土所引
起的环向或横向拉应力;
\item 由抗弯钢筋的销栓作用所产生的横向拉应力。这一现象与沿斜裂缝的剪切位移有关。
\end{enumerate}

\subsection{约束}

若能对围绕钢筋的混凝土加以约束,就能限制劈裂裂缝的开展。在某些部位,例如在梁的简
支端,一般都有来自支反力的横向压力。横向压力对钢筋的锚固是有利的。

箍筋,特别是密布的箍筋,可以阻止沿埋入的钢筋所形成的裂缝的开展,并使较大的粘结力
能已被传递。在很多情况下粘结力的传递只是当能通过骨料咬合作用把剪应力传过劈裂裂缝
时才是可能的。箍筋并不能防止当在梁中采用粗钢筋时(例如11号钢筋,直径35.8mm)总要形
成的劈裂裂缝;但正如第七章所述,它可以使摩擦力得以沿裂缝进行传递。箍筋看来并不能
改善对裂缝宽度的控制,但它可以保证发生的是更具有延性的粘结破坏形式。

当钢筋的接头位于构件的要害区段内时就可能需要螺旋筋或矩形箍筋这种形式的特殊配筋。
这种配筋的性能可以用界面剪力传递的概念来进行估计。

\subsection{重复加载及循环交变加载}

当钢筋拉力增大而且钢筋与混凝土之间的附着粘结受到破坏时,在肋条处的承压能力充分发
挥之前就发生一些摩擦滑移。这时若从钢筋上卸去荷载,就会产生反向摩擦阻力,并在钢筋
中引起一些残余拉力,同时在周围混凝土中引起相应的压力。而肋条附近的非弹性变形、混
凝土中的微裂缝以及收缩应变的释放会导致一些永久性滑移,其数值主要取决于先前作用的
荷载大小。基于这一原因,钢筋受拉时形成的裂缝在荷载卸去之后就不能完全闭合。在重复
加载的情况下,摩擦阻力逐步减小,导致粘结机构刚度的退化。

在地震过程中,像梁柱交界面这样的临界截面可能发生受拉和受压屈服的现象。粘结力的逐
渐衰减可能导致屈服发展到锚固区内,从而使可以用来承担钢筋屈服强度的有效发挥长度急
剧减小。

\section{可用粘结强度的确定}

各种钢筋埋入不同强度混凝土中的粘结性能传统上一直都是用拔出试验的方法来确定的。通
常所用的从周围混凝土中拔出钢筋的方法同时也引起了横向作用于钢筋的压力。这种横向压
力对粘结强度产生了有利的影响,因而就不是结构中遇到的典型情况。为此而提出了可以消
除横向压力的各种形式的试验试件。在这些试件中粘结强度是以拔出力在埋入钢筋的表面上
所引起的平均粘结应力来表示的。



\chapter{框架的强度和延性}

\section{引言}

为了确定能用于设计的弯矩、剪力及轴向力的可能的分布情况,就有必要研究钢筋混凝土框
架在极限荷载下以及在接近极限荷载时的性能。如果各个临界截面具有足够的延性,容许各
项作用力在接近极限荷载时产生重分布,就有可能采用与线弹性结构分析所给出的不相同的
弯矩和力的分布。在地震区还有一个更重要的设计方面,这就是结构在承受地震型荷载时的
延性,因为目前的抗震设计原理是在遇到较大地震时要依靠由于非弹性变形所导致的能量耗
散。

极限荷载下的性能在这两个方面都取决于构件的变形特性。对于框架来说,这些变形特性主
要取决于弯矩和曲率之间的关系。梁和柱截面的弯矩——曲率关系以及构件在极限和接近极限
荷载时的弯曲变形计算方法都在第六章中讨论过了。给出了一个截面的典型的弯矩——曲率曲
线,其中受拉钢筋在极限弯矩时达到了屈服强度。这条曲线标明了混凝土开始开裂时、受拉
钢筋开始屈服时以及混凝土开始剥裂和压碎时的各点。一个延性截面能够在超过开始屈服时
的曲率之后的一个较大曲率范围内维持接近于极限值的抗弯能力。

\section{弯矩重分布和塑性较}

很明显,如果结构受荷达到和超过了使用荷载的范围,钢筋混凝土截面的弯矩——曲率关系的
非线性性质就会引起弯矩相对值的少许调整。特别是由于一些截面的塑性转动,就有可能使
这些弯矩呈现出一种不同于由线弹性结构分析所导出的弯矩图形,而且所有正负弯矩的临界
截面在极限荷载下都达到了各自的极限抵抗弯矩。这样,弯矩的重分布就能对超静定结构的
极限荷载产生显著的影响。

现在以一根等截面的两跨连续梁作为例子;设$M'_u$为负弯矩截面的极限抵抗弯矩,$M_u$为
正弯矩截面的极限抵抗弯矩。我们假定截面配有足够的抗剪钢筋,能够达到极限弯矩而不致
发生剪切破坏。我们还假定这些截面的弯矩——曲率关系为所示的延性截面的理想双折线关系,
所有的截面直到极限弯矩为止都具有同样不变的抗弯刚度,而且弯矩在较高曲率时仍保持极
限值不变。当荷载较小时,由两个集中荷载引起的弯矩分布将与弹性理论分布一致。梁的恒
载在这里忽略不计。当作用荷载继续增加时,就会有一个临界截面,譬如说中间支座截面,
将先于其他截面达到极限抵抗弯矩。于是中间支座处的弯矩将为$M'_u$,如所示。这根梁能
够承受更多荷载的幅度是取决于中间支座处的塑性转动能力。如果截面是脆性的,则弯矩在
达到最大值后便将迅速减小,而梁就会在不再承受更多一点荷载的情况下突然破坏。如果截
面是延性的,梁就能承受更多的荷载,因为中间支座处的塑性铰在转动的同时还能保持抵抗
弯矩$M'_u$不变,而弯矩重分布就会发生,直至跨中最大正弯矩增加到$M_u$为止。这样就形
成了所示的破坏机构。描述了各个临界截面处的弯矩随着梁上荷载而变化的情况,这时假定
塑性铰是首先在中间支座处形成的(这个假定要求$M'_u/M_u \le M'/M=1.2$)。这里应该注意
到的是,在所有各个阶段中平衡条件都要求

而且如果直至跨中截面的弯矩增长到$M_u$时中间支座处的弯矩总是保持为$M'_u$不变,我们
就可以得到

因此,如果塑性铰具备足够的转动能力,极限荷载下的弯矩重分布就可能与用弹性理论计算
出来的差别很大,并将取决于截面的极限抵抗弯矩。在钢筋混凝土结构中,要形成第一批塑
性铰的延性可能还不足以实现每个临界截面都达到极限弯矩的完全弯矩重分布。所以,如果
要依靠弯矩重分布,就必须保证能在塑性铰处提供足够的延性。

作为一个例子,让我们现在来计算的两跨连续梁的中间支座处首先形成塑性铰所需要的塑性
转角。梁和曲率图表示在中间支座B处已经发生了足够的塑性转角并使得极限弯矩$M_u$恰好
在跨中发挥出来的阶段。所以,表明的就是刚好达到的阶段。这时是认为塑性曲率发生在临
界截面每侧的等效塑性铰长度$l_p$上。而沿构件长度的弹性曲率可以根据弯矩分布图和假定
不变的抗弯刚度$EI$计算出来。中间支座B处的塑性转角

在上述实例的计算中包含有几个假定,现讨论如下。

对所有的截面均假定直至极限弯矩为止都具有不变的抗弯刚度$EI$。这个假定只是在混凝土
开裂前的低荷载情况下才是准确的。当梁开裂之后,裂缝部位的抗弯刚度降低,而抗弯刚度
沿构件的变化导致了弯矩的分布发生变化,并使之不同于按不变的抗弯刚度用弹性理论计算
出来的弯矩分布。在进一步加载的情况下,开裂范围扩大,抗弯刚度的分布以及随之形成的
弯矩分布都将再次发生变化。当构件所含正、负弯矩钢筋的数量不同时,这种效应就特别显
著;而在T形截面梁中甚至还要更加明显,这是因为负弯矩区段翼缘的开裂较之正弯矩区段腹
板的开裂所导致的抗弯刚度降低程度要大得多。抗弯刚度沿着梁的这种变化将对在极限荷载
下实现完全的弯矩重分布所需要的塑性转角的大小产生影响。所以,严格地说,在确定极限
荷载下的塑性铰转角时必须考虑开裂对截面抗弯刚度$EI$的影响。

这里选用的弯矩——曲率关系是假定在超过屈服后有一个水平段,在这里弯矩保持为其极限值
不变。这一假定只是近似地反映了开始屈服以后的实际弯矩——曲率关系,因为这条曲线在受
拉钢筋开始屈服后有一个上升到极限弯矩的区段。所以,负弯矩和正弯矩区段的临界截面不
可能同时发挥出极限弯矩,因为这些截面处的曲率将处在弯矩——曲率曲线上的不同位置。很
明显,如果假定极限弯矩同时存在于所有的临界截面,就会对极限荷载得出一个偏于不安全
的数值。例如,如果开始屈服时的抗弯能力为$M_y=0.9M_u$,其中$M_u$为极限弯矩,则计算
出的极限荷载(假定所有的临界截面均处于极限弯矩之下)的误差就可能是5\%左右。很清楚,
如果把在最后形成的铰处达到屈服($M_y$)作为极限弯矩,而$M_y$又明显小于极限弯矩
$M_u$,则在极限荷载计算中(假定在所有的铰处都达到极限弯矩)的误差可能是显著的。

正如我们所看到的,要精确计算钢筋混凝土框架在极限荷载下形成完全的弯矩重分布时所需
要的塑性转角是困难的。不过,要想在设计中依靠弯矩重分布,我们就必须保证使各个临界
截面所能提供的延性超过根据象方才讨论过的那样一些理论上的考虑所计算出来的延性需要
量。

多年来早就知道,在荷载较大的钢筋混凝土结构中是会发生几分弯矩重分布的。

\section{框架的全过程分析}

如果截面的弯矩——曲率关系已知,则钢筋混凝土框架在从零到极限荷载的任一加载阶段的弯
矩、剪力、轴向力和挠度就可以利用静力平衡和几何协调这两个条件用解析方法来确定。但
是,弯矩——曲率关系的非线性造成了困难,因此通常就需要随着荷载一个增量一个增量地增
长而采取一个逐步求解的过程。此外,承受弯矩和轴向力的截面的弯矩——曲率关系不仅取决
于截面的几何性质和材料的性能,而且与轴向力的水平有关。以用于分析的刚度法为基础的
逐次线性逼近法可以用来一直追踪框架从零到极限荷载的性能。在这种方法中要把框架的各
个构件沿其长度划分为若干小单元。在每一个荷载水平下,与作用於每个单元上的特定弯矩
及轴向力相对应的抗弯刚度($EI=M/\Phi$)是从弯矩——曲率关系曲线上的对应点处得出来的。
各个构件在初始的若干荷载增量下均假定为没有开裂,而变形则是用未开裂截面的抗弯刚度
来确定的。在每个荷载增量下都要对各个单元进行检查来确定是否已经达到了开裂弯矩。如
果发现开裂弯矩已经达到,单元的抗弯刚度就要在开裂截面的基础上重新计算,而且框架中
作用力也要重新计算。在一个荷载水平下这个步骤要重复进行到所有的抗弯刚度都为正确时
为止。在较高的荷载下,当单元上的应力进入了非弹性范围时,每个单元的抗弯刚度都要调
整到与按这一组弯矩和轴向力的水平计算出来的弯矩——曲率曲线上的相应点相对应的值。最
后,随着荷载进一步增加,塑性铰将遍布整个框架,而当机构形成从而不能承受更多的荷载
时,就达到了极限荷载。

显然,对钢筋混凝土框架在所有加荷阶段中的性能进行全过程分析的方法时冗长的,而它只
有在借助大容量计算机的情况下才能成功地进行。对于这样一个通用的计算机程序来说,需
要输入的包括框架的几何尺寸、横截面的性质、材料的性质和加荷形式。输出的则可能是任
意荷载水平下的弯矩、剪力和轴向力的分布和挠度以及极限荷载值和塑性铰的位置。这样一
种确定极限荷载的分析方法避免了在常规的上限塑性分析中为了确定正确的破坏机构所需要
的试算过程。在常规的上限塑性分析中必须对所有可能出现的破坏机构进行探讨,以便确定
能给出最小极限荷载的那种机构。此外,通过全过程分析方法还能获得对极限荷载更准确的
预测,因为在极限荷载下临界截面的抵抗弯矩是已知的。预测,这种方法还避免由于假定抵
抗弯矩同时存在与所有临界截面而引起的误差。这种方法还能把轴向力对抗弯能力的影响准
确地包括进来,因为轴向力的大小是已知的,而在常规的上限塑性分析中,首先就要把轴向
力的水平估算出来。

由于这个全过程方法是一种解析方法,要把它用于设计看来就需要一个试算解。预测难以认
为这种方法是适合于设计应用的。然而,当有了更多的便于获得的计算机程序时,这种方法
还是会成为估计或复核结构在全部加载范围内的性能(其中包括使用荷载和极限荷载下的性
能)的一种强有力的分析工具的。在这种程序中也可以包括框架的几何形状在荷载作用下的变
化,以便考虑挠度对内力的影响。这样就能计及由于柱子的挠度所引起的弯矩增大,并且表
示出任何可能的失稳效应。

\section{确定用于设计的极限荷载下的弯矩、剪力和轴向力分布的方法}

现在我们来讨论一些能够在钢筋混凝土框架的强度设计中用来确定极限荷载下的弯矩、剪力
和轴向力分布的方法。在这些方法中由考虑某种弯矩重分布的或不考虑弯矩重分布的弹性弯
矩图法以及各种极限设计法。

\subsection{弹性弯矩图}

对于各种各样的荷载组合来说,结构在极限荷载下的弯矩和力都可以利用线弹性结构分析法
来计算。而各个截面则设计成具有至少等于从这种分析中得出的弯矩和力的极限承载能力。
这是ACI318-71和其他很多建筑规范所建议的方法。ACI规范允许在计算构件的相对抗弯刚度
和抗扭刚度时采取只要时合理的各种假定,条件是这些假定在整个分析中是相互协调的。

看来不合理之处在于,虽然截面是在考虑了混凝土和钢材的非弹性性质的情况下按强度法设
计出来的,但是极限荷载下的弯矩和力却是在假定构件具有线弹性性质的前提下计算出来的。
不过,这种近似方法是能够成立的,因为这样求得的弯矩和力的分布满足静力平衡条件和边
界条件。也就是说,这样的弯矩分布从静力学的角度来说是可以接受的。这样一种设计事实
上可以看作是能够成立的下限(极限设计)解。

把结构假定成具有线弹性性能有以下优点:它能保证在达到极限荷载之前只会发生少量的弯
矩重分布,因为各个临界截面都将倾向于同时达到各自的极限承载能力。因此,在各个临界
截面处所需要的塑性转角就比较小,各个截面的塑性转动能力也就不需要验算。不过,很明
显,总还是需要几分弯矩重分布的,这是因为一经开裂和开始产生非弹性应变之后,构件的
抗弯刚度就会发生变化;因此,如果用线弹性结构分析计算出的弯矩不是以抗弯刚度最终的
复杂分布为依据时,在所有的临界截面都能达到其抗弯强度之前就需要发生一定程度的弯矩
重分布。

把构件假定为具有线弹性性能至少还有另外两个优点:一是它能使钢筋和混凝土在使用荷载
下的应力保持在尽可能低的情况下,从而使混凝土的裂缝宽度减至最小;而是可以用相对简
单和充分确认的结构理论来求出设计弯矩和力。

通常,用于结构分析的抗弯刚度是以混凝土毛截面为依据的,不考虑混凝土开裂,钢筋也忽
略不计。看来这是一个粗略的近似,因为当构件开裂时,抗弯刚度会发生变化。例如,对一
个模量比为10的矩形截面,其开裂时的抗弯刚度从毛截面值降下来的幅度对于
$\rho=\rho'=0.01$的截面可能是30到60\%,对于$\rho=0.01,\rho'=0$的截面则可能是40到
60\%,这取决于钢筋在截面中的位置。但还必须记住弯矩的分布取决于各个构件抗弯刚度的
比值。有时构件开裂后的抗弯刚度比值仍然接近于最初假定的数值,因为在所有的截面中可
能发生的是类似的刚度变化;于是,在荷载较大时,为了形成假定的弯矩图形,只需要较小
的弯矩重分布。不过,由开裂引起的抗弯刚度比值的改变在有些情况下也可能是很显著的。
例如,在连续T形梁中,开裂会使负弯矩区段的抗弯刚度较之正弯矩区段产生较大程度的降低;
因此,在开裂之后,最大负弯矩与最大正弯矩的比值便将小于假定抗弯刚度一致时所得出的
比值。此外,在框架中柱子抗弯刚度的变化可能没有梁那么大,因为通常柱子的配筋比梁多,
而且在一般情况下都承受有轴向压力。所以对柱子来说,抗弯刚度从毛截面值到开裂截面值
的变化将不会很大。事实上,在很多框架中,两将在使用荷载阶段开裂,而此时柱子则将保
持不裂。已开裂梁的抗弯刚度的降低可能导致柱子的弯矩比按毛截面刚度计算时更大。但在
接近极限荷载时,柱子的抗弯刚度将会降低,这时弯矩又将重新分回梁中去。为了避免较大
的弯矩重分布,较好的办法可能是把梁的惯性矩趣味近似的换算开裂截面值(例如$0.5I_g$),
而柱的惯性矩则取毛截面值。抗弯刚度偏离毛截面值最多的情况可能是出现在既考虑构件的
抗钮刚度,又考虑构件的抗弯刚度的框架中,因为开裂导致抗钮刚度的降低要比抗弯刚度的
降低大得多。例如,开裂有可能使一个构件的抗钮刚度降低90\%。因此,抗钮刚度往往都可
以忽略不计。

上面的讨论再一次强调了,除去在设计中采用最终的复杂的刚度分布之外,就总是需要一定
的弯矩重分布,而且重分布的程度取决于设计者对抗弯强度的假定。所以,虽然线弹性结构
分析法给出了一个在强度设计中确定弯矩及力的分布的简便方法,但是应该考虑到的是,为
了达到设计极限荷载,各个临界截面都将需要一定的延性。因此,对构件的刚度就应采取相
当逼真的近似值。

\subsection{极限设计}

极限设计法允许采用在极限荷载下的任何一种弯矩分布,只要下列各项条件能够得到满足:

1.所取的弯矩分布从静力学的角度考虑应是可以接受的。这就是说,不论对于整个结构还是
对于它的任何一个构件,所选的弯矩图都不能违背平衡法则。这样一种弯矩分布,举假定足
够的反弯点位置以使结构成为静定、然后再根据静力平衡方程求例来说,可以通过先出这样
形成的弯矩和力的途径来确立。

2.塑性铰区的转动能力足以使这一假定分布在极限荷载下形成。

3.在使用荷载下的开裂和挠度不致过度。

上述的第一到第三项要求可以用极限平衡、转角相容和适用性来表示。这个方法把弹性弯矩
突发向前推进了一部并容许幅度较大的弯矩重分布;但它必须表明是能够达到的,而且它还
必须不损害结构的适用性。

是在单位长度上受有极限均布荷载$\omega_x$作用的一根连续梁中可能形成的一个极限弯矩
图。图中各跨当其两端不受转动约束时,由作用的外荷载所引起的自由(静定)弯矩的最大值
为$\omega_xl^2/8$。由于各支座处的梁端约束弯矩而形成的弯矩调整线可以选在自由弯矩图
中的任何位置。各个截面所需要的抵抗弯矩的大小可以根据各个塑性铰处的弯矩计算出来。
例如,如果把弯矩调整线的位置选成使中间支座的弯矩均为$\omega_xl^2/16$,则中间各跨
与两个端跨所需要的最大正弯矩即分别为$\omega_xl^2/16$和$0.0958\omega_xl^2$。这样我
们便有无限多个可供使用的弯矩调整线的位置,因为截面可以根据需要提供的极限抵抗弯矩
来配筋。这种情况可以同连续钢梁的塑性设计相对照:如果不使用钢盖板,则钢梁截面不论
对于正弯矩还是负弯矩都具有相同的塑性抵抗弯矩。因此在破坏弯矩图中弯矩调整线便只有
一个位置。而对于根据混凝土构件来说,在一般情况下弹性弯矩图只是能够采用的各种可能
的弯矩图中的一个。

由于作用于所有各跨的活荷载将导致最大的自由弯矩,因此它将给出正、负弯矩值的最不力
情况。可是为了计算负弯矩钢筋在跨中所需的延伸范围,活荷载却应从该跨去掉。例如,考
虑一根承受均布荷载的两跨连续梁,并把这里对作用的荷载和活荷载所选的极限弯矩如图。
为了决定AB跨内顶部钢筋所需的延伸范围,就把活荷载从该跨移开,剩下的只有恒载弯矩,
这样就显示出了在该跨内为了防止由这种加载形式引起的破坏而应配置钢筋的范围。

由极限设计带来的主要优点与由弯矩重分布带来的相类似,即可以对弯矩图进行选择以避免
在构件支座处钢筋的拥挤;按划分自由弯矩为正、负弯矩而得出的弯矩进行设计,而不按考
虑到不同加载位置的弹性理论弯矩包络图中的各个弯矩峰值进行设计也能获得相当大的经济
效益。这种方法还是设计者能够对结构的实际性能作出判断。不过,现在一般还没有把关于
塑性转动能力和适用性的知识形成用于极限设计的简单规则来包含在建筑规范中。承认极限
设计法的最大阻力一直是担心它不能满足关于开裂和挠度等使用荷载下的性能,以及一些现
有的极限设计方法的复杂性和缺乏关于构件塑性转动能力的准确的试验数据。

\section{考虑地震荷载的设计}

\subsection{基本概念}

在地震过程中地面将以随意的方式沿一切方向产生运动。把地面的水平和垂直加速度作为试
件的函数而进行的量测表明,地面的加速度可能是相当大的。例如1940年在El Centro地震过
程中记录下来的地面加速度峰值便是0.33g。一个极端的例子是1971年的San Fernando地震,
当时在Pacoima水坝坝址曾经测到的地面加速度峰值超过了1g。

当结构在一次地震中经受地面运动是,它是以振动的方式作出反应的。这是如果结构是按弹
性性能工作的,最大的反应加速度就将取决于结构的自振周期和存在的阻尼大小。在假定结
构结构对一些典型的地震记录作出弹性反应的情况下对一些结构所做的动力分析已经指出了
这些结构可能经受的反应加速度的等级。例如,就展示出了一个以承受地面振动的单自由度
振荡器的形式出现的简单结构。这种最大反应加速度$s_{abs}$是以结构自振周期和阻尼大小
的函数形式画出来的,而阻尼则是临界粘滞阻尼的一个百分率来表示的。这些曲线是从比较
不规则的实际曲线经理想化(“平滑化”)处理而成的;可以理解为最大的地面加速度。显然,
在一定的周期范围内,结构的最大反应加速度可能数倍于地面加速度。周期非常短的结构(也
就是刚度非常大的结构)的最大反应加速度将接近于最大地面加速度。振动周期长的结构的最
大反应加速度几乎不可能大于最大地面加速度。而在周期更长的情况下,最大反应加速度还
可能小于最大地面加速度。阻尼的增长总会导致反应加速度的降低。在地震过程中作用于简
单结构的最大惯性荷载用质量乘以加速度求得。

有一些建筑规范,例如加利福尼亚工程师协会规范和建筑官员国际会议规范所建议的设计地
震荷载都是以静力侧向荷载的形式出现的。多层结构的等效静力侧向荷载通常都是以近似三
角形分布的方式作用在结构上的,把最大的荷载加在顶部,以模拟图中第一振型的挠曲形状。
这些规范都是用静力设计荷载来确定结构要承受住由地震引起的动力荷载而必须具备的强度
的。但是这里建议的静力设计侧向荷载的水平一般都相当低。对一些能在强烈地震过程中记
录下来的地面运动下作出弹性反应的结构所做的动力分析已经表明,理论上的反应惯性荷载
可能远大于这些规范所建议的静力设计侧向荷载。虽然这个差别对于在设计中用安全系数进
行调节来说是过大了,但众所周知,按规范的的侧向荷载所设计的一些结构都经受住了一些
强烈地震。这种显然异常的现象主要是由延性结构通过弹性后的变形耗散能量的能力所引起
的,而且得到了诸如由于增大的阻尼和土壤于结构的交互作用而使反应减弱的这样一些因素
的帮助。构件的延性则可能是最重要的因素。

显然,要把结构设计成能够经受住可能最大的地震而不损坏将会是不经济的。为了提供能以
抵抗非常强烈的侧向振动的强度而需要的费用必须与结构的重要性和地震发生的概率相权衡。
加利福尼亚工程师协会规范对荷载的水平所给出的判断条件如下:建筑物应能抵抗较小的地
震而无损坏;应能抵抗中等的地震而无结构性的损坏,担忧一些非结构性的损坏;应能抵抗
较大的地震而不塌倒,但有一些结构性的和非结构性的损坏。所以,对损坏的可能性是认可
了的,但却不应有生命损失。规范的目的是要获得一种将在建筑物的使用寿命中预料不只发
生一次的地震下表现为弹性的结构;此外,这种结构还应能经受住在其使用寿命中可能发生
的较大地震而不致倒塌。为了防止在较大的地震下倒塌,它的构件就必须具有足够的延性以
通过弹性后的变形来吸收和耗散能量。这里所需要的延性量级可能伴随着非常大的永久变形。
因此,虽然结构应不倒塌,但这样造成的损坏可能已无法修复,而结构便可能成为一项全然
的经济损失。

\subsection{对位移延性的各项要求}

在抗震设计中首要考虑的问题显然是需要得到这样一种结构,它能在经受深入非弹性阶段的
若干次侧向加载循环时以延性方式变形。对于按规范设计的对典型地震运动作出反应的结构
所做的非线性动力分析已经表示处所需要的弹性后变形的量级。

参照一个单自由度振荡器可以看出非线性性能能给结构对强烈地震运动的反应带来的影响。
这样一个能作出弹性反应的振荡器将具有所示的荷载——挠度关系。其中点b为最大反应。曲线
下面的面积abc表示在最大挠度时所贮存的势能;而当质量返回到零位置时,这一能量就转化
为动能。如果振荡器没有足够的强度来承担弹性反应的全部惯性荷载,并且形成了一个具有
弹——塑性特性的塑性铰,它的荷载——挠度曲线如所示的那样。当达到了塑性铰的承载能力时,
挠度反应就沿着de线进行,而点e则表示最大反应。在这种情况下达到最大挠度时所贮存的势
能是用面积adef来表示的。注意,这时作用在结构上的力已经受到塑性铰承载能力的限制。
当质量返回到零位置时,转化为动能的能量是用小三角形面积efg来表示的,因为由面积
adeg所表示的能量已被塑性铰以热能或其他不能恢复的能量形式耗散掉了。这样,很明显,
在弹性结构贮存的全部能量在每次循环中都是作为周转能量被转化回来了,而在弹——塑性结
构中则只有一部分能量被转化回来。所以,在每次循环中贮存在弹——塑性结构中的势能不需
要像在弹性结构中那么大,而且弹——塑性结构的最大挠度也不需要比弹性结构的大许多。事
实上一系列动力分析已经表明,这两种结构所达到的最大挠度可能几乎相同。以最大挠度相
当的假定为基础的工作性能。

上面的考虑可以近似地适用于多层框架。很明显,按照规范的静力荷载设计的建筑物在强烈
地震下就可能需要相当大的延性;而如果按较大的静力侧向荷载进行设计,延性要求自然就
可以降低。通过利用设计频谱和降低系数,设计者就可以近似地把所要经受住的地震与静力
设计荷载和位移延性系数联系起来。对于一幢多层的建筑物,能够给定位移延性系数的位移
是在适当的部位(例如在屋顶水平处)来量测的。在一个多层框架中,总的趋势是要在整个框
架的各个临界截面处产生塑性铰,但所以的塑性铰不会在同一个荷载下产生。所以,侧向荷
载——挠度关系曲线将不是双折线,而是倾向于具有较为弯曲的形式,这是由于塑性铰在不同
的荷载水平下陆续产生时刚度逐渐降低的结果。在这种情况下,为了估计位移延性系数,就
可以近似采用双折线型的侧向荷载——挠度曲线,并把开始屈服时的挠度当作设计荷载在按弹
性性能工作的框架梁上所引起的挠度。不过,这种近似方法意味着在重要的情况下为了精确
估计将对较大的地震做出反应的结构所需要的延性就可能需要对这个结构进行非线性动力分
析。

需要强调的是,延性是与塑性变形相联系的,因此也就是与永久性的结构性损坏相联系的。
这就意味着一个用较小的荷载降低系数设计的结构在强烈程度较小的地震中容易遭受永久性
损坏。对于重要的建筑物,特别是那些需要在地震灾害之后继续发挥作用的建筑物,结构的
势能延性就不可能得到利用,因为对损坏的控制将是首要的设计准则。对于这样一类结构或
许就要用一个比较大一些的荷载降低系数,例如,即根据要求的。


钢筋混凝土截面的延性 可以用曲率延性之比来表达。其中为弹性后阶段末的曲率,为开始屈
服时的曲率。这是假定挠度变形占支配地位。在一些典型构建中得出的曲率延性系数的现成
数值已在第六章讨论过了。在很多情况下,可资利用的曲率延性系数是相当大的;但重要的
是要认识到在位移延性系数与曲率延性系数之间是有显著差别的。这是因为一旦屈服现象再
一个框架中开始出现,变形就集中在这些塑性铰的部位;所以当框架在弹性后阶段产生侧向
挠度时,在一个塑性铰处所需要的比值就有可能大于比值。

在一种简单的情况下,曲率延性与位移延性之间的关系可以参照所示的在柱顶承受一个侧向
荷载的悬臂柱来说明(图中也示出了在极限弯矩时的理想化曲率分布)。柱顶的侧向挠度可以
通过用曲率图形对柱顶取矩的方法来确定。关于根据曲率计算挠度的问题已在中进行过讨论。
柱顶在极限弯矩时的侧向挠度为

由于塑性铰的等效长度一般是在构件截面高度的0.5到1.0倍的范围内,因此,很明显,一根
悬臂柱所需要的比值通常会超过比值。而且若等效的塑性铰长度只占柱高的很小一个比例,
则所需的值就会特别大。另一方面,对于矮而宽的剪力墙,当其高度与其截面高度的比值使
得等效的塑性铰长度约等于0.3倍墙高时,曲率延性系数就会与位移延性系数具有相同的量级。

在多层框架这种复杂得多的情况下,至今还没有清楚地规定下来按照现行规范的地震荷载设
计所需要的比值。位移延性和曲率延性之间的关系是复杂的,因为对于大多数多层框架来说,
在各个临界截面处的屈服现象并不是发生在同一荷载之下,而且惯性荷载的分布比规范建议
的静力荷载的分布更为复杂。一些试图说明多层框架中比值和之间的关系尝试将在下面几节
中讨论。


为了使多层框架达到给定的位移延性系数而需要的曲率延性的量级可以根据静力倒塌机构和
一些简化假定来近似估计。我们现在来研究一个框架结构,它在对地震的加速度脉冲做出反
应同时承受重力荷载和地震荷载的作用。我们为此做出以下假定:

1.框架的各个截面将具有如所示的双折线弯矩——曲率特性,但不一定就具有相同的。对于梁
来说,这些量的数值将取决于截面的性质;但是对于柱子来说,开始屈服时的轴向力水平也
是有影响的。这个假定忽略了那些还没有开裂的构建的较大抗弯刚度以及裂缝之间的较大抗
弯刚度。因此就稍微偏高地估计了由弯曲引起的变形。

2.只考虑由于弯曲所引起的构件变形。这对于不含有截面高度与构件长度之比很大的构件的
框架结构以及对于没有广泛斜向受拉开裂的框架是一个合理的假定。忽略由剪切引起的变形
将在一定程度上补偿对弹性弯曲位移的偏高估计。

3. 当作用于框架上的地震荷载一直增加到发生屈服时,全部临界截面处的屈服现象都将在同
一荷载下开始,而且有足够多的截面发生屈服现象以形成一个机构。但是由于钢筋和混凝土
的实际强度的变化,由于规范规定的近似为三角形的地震荷载与由地震在结构中引起的惯性
荷载的实际分布之间的存在的差别,以及由于像在第一章中所讨论的那样一些影响构件强度
的因素,这种情况在实际结构中是很难发生的。不过,这个假定将使我们有可能获得一个合
理的简单解。

表示当侧向荷载增长到恰好在框架中导致屈服的地步,在一根典型柱子中的曲率分布。柱子
中的这种曲率分布是遵循弯矩图的形状的,因为这时各处的弯矩还依然处于弯矩——曲率关系
曲线的初始现行区段内。柱中的曲率从一层到另一层都将是不同的,这是因为截面的性质和
轴向力的水平都不相同。柱子在任一标高处相对于地面的侧向挠度都可以通过那个标高以下
的曲率图形对那个标高取矩的方法来计算。兹将框架的各层从底层起用1、2、3、来编号,并
注意到在柱中既有正曲率又有负曲率的情况,则当开始屈服时,第层的顶面相对于结构底面
的侧向挠度即为

当利用曲率图形取矩的方法来得出时,每一层的曲率分布图形都被看作是由一块矩形面积和
一块三角形面积所组成的,正如对第层所表示的那样。这就是说,这一层对挠度所起的作用
就等于矩形BCDE对第层顶面的面积减去三角形ACD对第层顶面的面积矩。

若假定底层柱的反弯点是出现在由底面算起的0.6倍柱高处,而所有其他各层都出现在高度中
点,则

屈服开始后发生的侧向位移是由塑性铰部位的转到造成的。可能形成两种类型的倒塌机构。

若在各根梁达到屈服曲率之前各根柱的临界截面已经开始进入屈服阶段,则。在这种情况下,
当侧向荷载保持不变是,由于各柱临界截面处塑性变形的发展,从而还会进一步产生挠度。
在最不利情况下,可能由于其他各层柱子均相对较强而仅在一层内形成柱型侧移机构来产生
这种进一步的挠度。表示了仅在第层形成的机构,其中只画出了塑性变形。这里发生的全部
塑性变形都是由关键性的这一层的柱子中的各个铰的塑性转动所造成的。在画出了当在第层
的一根典型柱的两个铰处都达到了极限曲率时的柱中曲率分布图形。


在已将位移延性系数取为相对于地面的极限位移与屈服位移的比值。如果对于顶层,则表明,
对于框架顶层以下的各层(但在塑性铰以上),其值均大于4,因为对于较低的各层值较小。如
果对于一个n层建筑物要求所有各层的平均值为4,而且如果塑性铰是出现在底层柱中,则应
将带入来求所需的曲率延性系数。这是因为建筑物的质量中心将在它的一半高度以下,而如
果是在大约为建筑物高度的一半之处得出所要求的等于4的值,就对作为一个整体的建筑物的
位移延性系数得出了一个合理的近似结果。但如果塑性铰发生在较高的各层柱中,则曲率延
性系数就不得不更大一些。

显然,如果框架有几层高,则由柱子截面提供使用的曲率延性系数一般都将不足以满足柱型
侧移机构的延性需要。

如果各根梁的临界截面在各柱达到屈服曲率之前就已经开始进入屈服阶段,进一步的侧向变
形就会在不变的侧向荷载下经由梁中塑性铰的塑性变形而产生。还需要在每根柱的柱脚处形
成一个塑性铰,但是柱子的其余部分可以保持弹性。展示了这样得出的梁型侧移机构。图中
只画了塑性变形。对于所示的变形状态进行的研究表明,在每根柱子底部的塑性转角为


这里导出的表明了,对于静力侧向荷载的作用来说,为了达到一个给定的位移延性系数而需
要的曲率延性系数的量级。只要给定一个框架的底部,就可以用这些公式来估计一个设计是
否适当。所示的柱型侧移机构是有危险的,因为塑性变形可能只发生在一层的各根柱子中;
在一个高层建筑中不太可能对这种机构提供能经受柱一次较大的地震的足够的曲率延性。很
明显,设法保证发生所示的梁型侧移机构要好得多,因为此处梁截面所需的延性比较小,而
且可以更容易地达到要求。为了保证发生梁型侧移机构,应使柱子具有足够的强度,以避免
塑性铰在此形成。对这种机构中的柱脚一定要谨慎地进行细部构造处理,采用横向箍筋或螺
旋筋,使它能产生所需要的塑性转角。

这里导出的各个公式都是以框架在所有的塑性铰截面处同时达到屈服这一简化假定作为依据
的。如果在所有的塑性铰出现之前需要进行弯矩重分布,那么在第一批要形成的塑性铰处所
需要的曲率延性就必须更大一些。此外,还假定弯矩图就像根据规范建议的等效静力侧向荷
载所要求的那样。这个所示的规范静力荷载主要是与反应的第一振型相对应的。对于高层建
筑来说,更高的振型就可能具有显著影响并导致在弯矩图形中发生根本性的变化。于是,在
实际动力情况下的梁内可以同时形成的塑性铰就比规范的静力荷载为基础的近似方法所建议
的要少一些。因此,上面刚刚介绍的静力近似方法只能作为一种入门。

尽管如此,我们的静力分析表明了对于强柱的客观需要,目的是如有可能借此来避免在柱中
形成塑性铰(因为通过梁内的塑性铰来耗散能量更为有效)。一般来说,在梁和柱脚的塑性铰
截面处所需要的曲率延性系数看来应能至少达到所需位移延性系数的四倍。

我们不能要求上述以静力倒塌机构为依据的近似方法能在对一次较大地震做出非线性反应的
多层框架这样一个复杂情况下对曲率延性系数给出准确的估计。人们也曾就若干对地震做出
非线性反应的多层框架做过动力分析,但很难得出概括性的结论。在多层结构的非线性反应
中所包含的变量是如此之多,以致目前能够做的还仅仅是定性的陈述。例如,地面运动的形
式有相当大的影响。一个被设计成对一种地面运动做出反应时在它的构件中可以发挥出近乎
一致的曲率延性的结构,在对另一种不用的地面运动做出反应时就可能局部地发挥出较高的
曲率延性。这就显得有必要对一系列多层框架进行更多的非线性动力分析,以便对所需要的
曲率延性系数的量级给出一个较好的指标。下面就讨论从现有的一些非现行动力分析中得出
的一些结果。

多层框架的分析表明,框架的位移延性系数是与和它具有相同的力——位移特征的、并和按弹
性反应荷载的同样份额进行设计的单自由度体系的位移延性系数属于相同量级的。从这种分
析中已经得出了构件的转角延性系数,其中为最大反应时构件端部的转角,为开始屈服时构
件端部的转角。所以,按照这种分析,所需要的曲率延性系数就可以根据含有假定的塑性铰
长度的曲率分布图来求得。例如,如果一根构件两端的弯矩相等但符号相反,而且构件沿其
长度不受横向荷载作用(即构件处于所示的对称双曲率状态),则弹性理论将指出此时端部的
转角为:

在所示的对称变形情况下,对尺寸调配得比较合适的框架来说,框架的动力分析得出的往往
都是构件的转角延性系数约为位移延性系数的两倍。可是,薄弱楼层的存在却将导致这些薄
弱层的转角延性要求比尺寸调配得比较合适的框架大许多倍。

由非线性动力分析法和由静力倒塌机构法这两者所表明的延性要求向结构的薄弱部分集中的
现象指出了一个在抗震设计中极为重要的原则。在通常按静力荷载所作的设计中,存在于结
构中的一些过强部分从来都不会降低那个结构的强度。但是在抗震设计中,当一个结构是依
靠通过塑性铰耗散能量而经受住地震时,结构中的一些过强部分的存在意味着对曲率延性的
要求将集中到结构的一些局部区段中去,并可能由于在这些区段内被迫产生的很高的非弹性
变形而导致倒塌。要注意的是,结构中薄弱部分起着象保险丝一样的作用。当结构的薄弱部
分达到了极限强度时(例如在柱型侧移机构的情况下),框架的其余部分还可能仍处于弹性阶
段。结构的薄弱部分可能是由于结构的这一部分设计得过弱和结构的其他部分设计得过强而
造成的。因此在抗震设计中存在着来自设计得过弱的构件和设计得过强的构件这两方面的危
险。通常造成过强现象的一个原因就是存在着结构反应中未予考虑的墙体。只在某些楼层中
设有墙体就可能会强制在其他各层中形成柱型侧移机构的。例如,假使一个建筑的底层是敞
开的,而它的上面各层确实由墙体包围起来的,破坏就可能集中在底层。为了避免产生这种
问题,在这类墙体和结构构件之间也许就改采用分隔缝。我们可以把1971年加利福尼亚San
Fernando地震后橄榄树景观医院看作是最近一个破坏主要集中于结构中一个楼层的例子。在
这个结构中上面四层都设有剪力墙和柱子,而在底层中仅由柱子来抵抗侧向荷载额作用;破
坏就主要集中在这一层的柱子上。结构在地震后差不多有0.6m大的永久性侧向位移就几乎全
是由底层的变形造成的。这一层的箍筋柱和螺旋筋柱的实例,而且明显的是,只有约束得特
别好的混凝土才有能力变形到这种程度并保持承载能力。还说明了关于地震破坏的另外一个
特点。柱子的修复要比梁的修复困难得多;因为要把结构扶直,而且在修复过程中要用支撑
加固。大范围的柱子损坏可能意味着结构不得不拆除,就象橄榄树景观医院那样。

非线性动力分析还表明,在多层框架柱中可能出现与按规范规定的静力侧向荷载求得的弯矩
分布相比没有预料到的弯矩分布情况。静力侧向荷载分析表明,除了接近框架顶和框架底的
楼层之外,只要梁不是比柱子柔得很多,反弯点通常就是位于接近柱高中点处。可是,非线
性动力分析却提出,在结构对地震的地面运动做出反应的过程中的某些时刻,在楼盖之间的
柱段内,反弯点的位置可能更接近于梁——柱节点,柱子有时甚至可能完成单曲率的。这个分
析方法是一个沿着地震时间历程的步进法,它直接对位移增量求解,并考虑了构件的弹——塑
性性能。柱子中在某些瞬间没有出现预料到的弯矩分布的原因是高振型的强烈影响,特别是
第二和第三振型。振动框架的挠曲形状近似地表示在中。规范规定的静力侧向荷载的分布通
常都是三角形的,它对应于从柱底处为零按线性变化到结构顶部处为最大值的侧向荷载,有
时还带有一个作用在顶点的附加集中荷载。这样的荷载分布是近似于对应基本(第一)振型的。
如果其他振型比较显著,则显然在某些时刻作用于结构上的惯性荷载的分布就可能与假定的
分布具有相当的差别,并导致所示的那种弯矩分布。在一些情况下,柱子的反弯点移到远离
柱高中点的地方,这就意味着在这种情况下引起的柱中弯矩可能要比有静力侧向荷载分析求
得的弯矩大得多,并可能导致在一些柱中形成塑性铰。

动力分析还表明,塑性铰通常并不是在同一时段内在所有的梁内普遍出现的。塑性铰的形成
往往是以一次包括几层的波浪形式沿框架向上发展的;但是在低层建筑中,却可能有全部的
梁都同时形成塑性铰的时刻。

当要计算或测定一个承受地震荷载的结构或结构部件的可供利用的延性时,极限变形的定义
往往会带来困难。可供利用的极限变形并不一定要是对英语最大承载能力的变形;因此,它
也不一定要是对应于一个特定的边缘混凝土压应变或特定的钢筋应变的变形。对极限变形采
用的定义看来似乎取决于能够容许承载能力降低多少和所能容许的结构损坏程度。若根本不
能容许损坏(例如有重要影响的结构或存贮危险化学品的结构),就可能需要保证结构具有完
全弹性的反应,而且设计强度也要相应较高。而对另一个极端来说,经受住地震而不倒塌就
可能是唯一的判别标准。在这种情况下可以容许在反应过程中出现非常大的应变,包括或许
无法修复的结构损坏。对于处于上述两者之间的情况就可以规定出所限制的应变或所限制的
变形。很多构件和结构都具有超过荷载——挠度曲线峰值的变形能力。如果判别标准是经受住
了地震而不倒塌,把极限变形定义为对应于最大承载能力的变形就太保守了。看来比较合理
的似乎是至少认可这种在最大荷载达到之后仍然具有的变形能力的一部分,并把可供利用的
极限变形定义为在最大荷载之后承载能力已经降低了某个数量时的变形。例如,在许多情况
下,最大承载能力降低10或20\%看来是可以允许的。但确切的数量则要取决于具体情况。在
重要的情况下可能要采用步进动力分析法来确定结构的某个荷载——位移滞后性能对于经受住
地震的作用来说是否合适。

1971年的ACI规范有一个包含抗震设计专门条款的附录。根据规范说明,这些条款“是准备应
用于处在发生建筑物较严重损坏的可能性很高的地震区内的钢筋混凝土结构,而这种结构是
在由于利用了由延性抗弯空间框架构成的抗侧力体系而把侧向总地震力降低得相当多的情况
下进行设计的,....。如果不利用对于侧向地震力的荷载降低系数,这些专门条款就不是
必须遵循的。这些条款适用于具有现浇梁柱接头的框架。因此,这些要求适用于那些“将为
最严重的地震作用所迫而进入到足以形成可逆塑性铰的侧向变形中去”的框架。这些条款是
以加利福尼亚结构工程师协会的规范的1967年和1968年版本以及其他技术情报和研究成果为
依据的。这些关于框架的条款划分成受弯构件、柱以及梁柱节点等节。现将所述的一些要点
概括如下:

对抗弯钢筋的配筋率定出了上线。的最大值不超过产生平衡破坏数值的0.5倍。参照可以表明
这个要求便是

还做出了规定来保证总是配有最少量的顶部与底部钢筋。顶部与底部钢筋两者都应沿构件全
长具有最少为的配筋率,其中钢筋的屈服强度。这里还提出了建议,保证配有足够的钢筋来
考虑反弯点不可预见的移位。在与柱子连接处,抵抗正弯矩的能力最少应为抵抗负弯矩的能
力的50\%,而且只要有可能,钢筋就应穿过柱子连续布置。在边柱处,梁内的钢筋应伸至柱
子的另一面为止,并用弯钩加上一段必要的附加延伸长度来保证锚固。

设计剪力是根据构件上的重力荷载和由侧向位移在构件端头所产生的塑性铰的抗弯能力来计
算的。这里采用现实的极限抗弯能力意味着由弯矩引起的剪力不会超过计算出的数值。要沿
构件全长布置不少于最低用量的腹筋,其间距在塑性铰区应不超过,在其余部位则不应超过,
其中为构件截面的有效高度。箍筋在绕过需要起受压钢筋作用的钢筋时以及在塑性铰区内应
该是封闭式的,其间距不应超过规定的数值。在受拉区或交变应力区,如果没有设置规定数
量的钢筋,受拉钢筋就不能采用搭接接头。

竖向钢筋的配筋率限制在0.01到0.06的范围之内。通常,在任一梁柱接头处,沿接头的每个
主平面的柱子的抗弯强度之和应大于梁的抗弯强度之和。当柱子的受约束核心截面的抗弯强
度之和足以承受设计荷载或者当其余的柱子和受弯构件能够由他们自己来承受那个水准的作
用荷载时,可以不满足这个要求。这个要求是用来保证塑性铰形成在梁内,而不是形成在柱
子内的。

在接头内布置的特殊横向钢筋应满足。接头还应具有至少等于在节点核心区处有梁内钢筋的
屈服力和柱子的剪力所引起的剪力那样大的足够的抗剪强度。这里要承受的剪力是作用在节
点核心每一个水平截面上的总剪力。规范对混凝土抗剪机构和横向钢筋建议采用相同于对柱
子所用的抗剪强度公式。在柱子的四边都有梁嵌入柱内并且覆盖了相当大的一部分柱子表面
的那种接头处,规范允许把需要的抗剪钢筋减少一半。

ACI的条款推荐采用一般的方法来进行抗剪钢筋设计。这种方法不考虑混凝土承担的剪力在高
强交变加载过程中可能降低的情况。弯矩在塑性铰区内交变的结果能够导致由混凝土横过受
压区承担的剪力以及由骨料咬合和销栓作用承担的剪力的降低。这是因为在某些阶段,在构
件的这类区域内将存在沿斜截面全高张开的裂缝, 其中的弯矩仅由上、下钢筋组成的力偶来
承受,而且还会有斜拉裂缝交替地张开和闭合。这表明有必要忽略混凝土的抗剪机构,并在
梁内抗弯钢筋有可能在循环加载下达到屈服的区域内设置腹筋来承受全部剪力。这相当于在
这些区域假定$\upsilon_c=0$。

\subsubsection{柱}

避免出现塑性铰

规范要求除另有专门规定之处以外,在接头处的每一个主平面内,柱子的抗弯强度之和应大
于梁的抗弯强度之和。其目的在于使塑性铰在梁内形成,而不是在柱内形成。可惜由于下述
两点理由,这个要求看来是不能阻止柱子出现塑性铰的。

我们从得知,非线性动力分析表明了反弯点在一次地震过程中的不同阶段可能发生在原理柱
高中点的地方。柱子有时甚至可能是单曲率弯曲。于是就可能出现如所示的那种弯矩分布图。
在一个典型节点处,由梁传入的总弯矩是由柱子的弯矩之和来对抗的,因此可得:
\begin{equation*} \sum M_b=M_{b1}+M_{b2}=M_{c1}+M_{c2}
\end{equation*}柱的最大弯矩可由下式给出
\begin{equation}
  \label{eq:31} M_{c1}=\sum M_{b}-M_{c2}
\end{equation}现如为梁处于极限抗弯能力时传入柱中的总弯矩,为柱1的极限抗弯能力,而
我们希望时防止在柱1形成塑性铰,那么这个要求就是

这时如果柱子仍然保持按双曲率弯曲,则极限情况就发生在时,于是就要求。如果柱子是处
在单曲率弯曲状态的,就要求。于是ACI关于在节点处柱的强度之和应大于梁的强度之和的要
求将不能防止在柱中出现塑性铰。为了确保塑性铰不在柱中形成,当反弯点可能位于层高的
任意一点时,我们就必须要求每根柱界面的抗弯强度至少等于在这个弯矩平面内的每根梁截
面的抗弯强度之和。如果反弯点位于层高之外,就得要求甚至更大的柱子抗弯能力。



在中曾对沿着一般性的水平方向作用在结构上的地震荷载的影响进行过讨论。很显然,这种
荷载如果不是作用在建筑的一个主方向内,就会由于双向弯曲而降低柱子的抗弯强度,并将
由于从梁接受的抗弯强度分量而增大由梁传入柱子的弯矩。举例来说,如果双向弯曲把柱子
的抗弯强度降低了30\%,同时把由梁传入的弯矩增大了40\%,柱子就需要有两倍于单轴受弯
的强度。这种情况在角柱处可能会特别严重,因为在那里除了双向弯曲效应之外,由地震引
起的从两个成直角的梁传入的轴向力还可能叠加起来。

很明显,ACI的简单规定将不能防止在柱中形成塑性铰。在一般情况下,如果要在柱中避免塑
性铰,则反弯点从柱高中点移开以及荷载不沿一个主轴方向作用的情况就将需要柱子的强度
显著地大于ACI的要求。要防止在柱中形成塑性铰是如此困难,以致必须考虑在某些柱子中出
现塑性铰的不可避免性。应当注意的是,若仅在一层柱的一端形成塑性铰,则只要未在另一
层柱中形成由反号弯矩引起的塑性铰,就将不会导致出现柱型侧移机构。

\subsubsection{抗剪强度}

ACI的条款要求横向钢筋要足以保证构件的抗剪能力至少等于在框架中形成塑性铰时的剪力。
横向钢筋假定能同时起抗剪钢筋和约束钢筋的作用。剪力是假定像在一般设计方法中那样由
混凝土抗剪机构来承受;所以这个规定没有考虑在交变加载情况下混凝土抗剪能力可能发生
退化的现象。正如在梁中的情况一样,看来由混凝土提供的那部分抗剪强度,即,在塑性铰
区应取为零,而剪力即全部由腹筋来承受。在塑性铰之间(即在离开柱端的部位)剪力可以考
虑由混凝土承受。在塑性铰处要求全部的抗剪能力都由腹筋来提供可能对于较高轴压力情况
是偏于保守了,而今后的试验资料可能会揭示出,如果轴向压力较高,混凝土是能够承受一
些剪力的。可能比较合理的是当时忽略混凝土承受的剪力,而当时就让混凝土承受一般的一
半。

条文中也没有提到当地震荷载沿一般性方向作用于结构上时在柱中可能引起的较高剪力。这
时传入柱中的弯矩增大了,它导致了更大的剪力。而这个提高了的剪力是由沿其对角线方向
受荷的截面来抵抗的。对矩形截面沿斜向受荷时的抗剪强度还没有进行过适当的研究。尽管
如此,还是可以通过对那些与斜拉裂缝相交的箍筋拉力的分量求和的办法来估计腹筋所提供
的那部分抗剪强度。

\subsubsection{梁柱节点}

条文指出,节点核心的抗剪强度可以利用与柱子所用的相同的抗剪强度公式,用混凝土抗剪
机构的抗剪能力与横向钢筋的抗剪能力相加的办法来计算。在第十三章所述的近期试验已经
表明,当高强循环(交变)荷载作用于节点时,这样一种方法是不能令人满意的,因为节点核
心中的混凝土已经破碎了。目前对钢筋混凝土节点核心的抗剪机理还没有完全弄清楚,但是
把节点核心的设计方法建立在由构件得出的试验结果上看来是错误的。在第十三章论述的近
期试验资料还表明,临界裂缝是从节点核心的一个角走向另一个角的,但与相交的构件轴线
并不成45;一个更好的设计方法似乎应该是提供足够的横向钢筋来承受横过对角裂缝的总剪
力。

此外,条文也没有提到当地震沿着一般性的方向作用于结构,并在建筑的两个轴线上引起同
时作用的荷载时,在一个节点核心中所引起的高剪力。在这种一般情况下的节点剪力要大于
地震荷载沿建筑的一个主轴作用时的剪力。这个同时导致两个方向的梁系达到屈服的一般加
载情况很可能意味着周围的梁对节点的约束并不像条文所含意的那样有效。

\subsection{计算柱子塑性铰区域内特殊横向约束钢筋的另一种方法}

\subsubsection{方法}

在前面中讨论过的一些考虑已经表明,在非常大的地震过程中,由于地震荷载同时作用在建
筑的两个主轴方向以及反弯点位于柱端附近,而将导致柱中产生很大的弯矩。因此就必须认
为在一些柱子中形成塑性铰是不可避免的。要防止在柱中出现塑性铰,就得要求柱子比梁强
得多。墙体的存在以及构件强度的无意识的变化也可能导致柱子出现塑性铰。因此柱中可能
出现塑性铰的区域就应具有延性性能。

正如前面指出的,现行规范关于确定特殊横向钢筋的方法是在保护层剥落之后保持柱子的轴
心荷载强度这样一个原则为依据的。这个方法没有把细部构造要求与偏心受荷柱截面所需要
的塑性转动能力联系起来。

弯矩——曲率关系提供了一个度量截面塑性转动能力的尺度,而一个以确保良好的弯矩——曲率
关系为出发点的方法则能成为按照延性要求来进行细部构造处理的合理依据。在根据弯矩——
曲率关系来估计可供利用的曲率延性系数方面存在的一个问题是如何定义极限曲率。很多截
面在超过了弯矩——曲率曲线的顶点之后显然都还保持着相当大的塑性转动能力;而且,承认
这一点并把定义为截面抗弯能力降到最大弯矩的80\%到90\%时的曲率想必是合理的。此外,
我们还根据前面的研究得知,柱子在其柱脚处应能达到高达,其中为位移延性系数。因此一
个合理的准则将是要求柱子的潜在塑性铰区域能够达到至少等于。这样照道理可以提供足够
的延性,除非由过强的或不够强的构件使柱子在一个层间内出现塑性铰。在根据受约束混凝
土和纵向钢筋的应力——应变曲线推到弯矩——曲率关系的过程中允许使用这一准则,从而使所
需要的横向钢筋的数量得以确定。

\subsubsection{弯矩——曲率分析}

在推导偏心受压矩形钢筋混凝土柱截面的弯矩——曲率特征曲线时要考虑以下因素:
\begin{enumerate}
\item 作用在柱子上的轴向荷载的水平;
\item 纵向钢筋含量;
\item 受约束柱截面的边长比例;
\item 材料的应力——应变特征曲线。
\end{enumerate}

从理论上说也应该考虑循环加载的影响,但是循环荷载分析的复杂性使得研究大范围内的各
种情况颇为困难。我们在这里的研究只分析单调加载情况,而这应能首先给出一个合理的估
计结果。

\subsection{框架在地震荷载下的承载能力设计}

由于不可能准确的预测可能发生在任何给定场地的地面运动特征,也就不可能对钢筋混凝土
多层框架在经受很大的地震扰动时的整体受力性能进行估计。然而,赋予结构一些特点来保
证它具备最符合要求的受力性能则是可能的。就损坏、延性、能量耗散或破坏这一系列词汇
的意义来看,这意味着打破框架中这个复杂的抗力锁链时所希望的顺序。它的含义是指结构
的破坏模式中所希望的一种层次。要给一个复杂锁链的破坏机构规定任何顺序,就必须了解
每一个链环的强度。这种了解必须不以偏于安全的假定或者可靠承载能力为依据,而是现实
地以将在一次灾难性地震过程中经受很大变形的结构构件的最可能发挥的强度为依据。各种
强度的定义及其相互之间的关系已在第一章中进行过讨论。

尽管作用于结构的设计荷载或结构的位移图形都具有随机性的特点,但是根据现在的认识,
对强度和延性性能作出定值性的规定将最有希望获得效果良好的反应和防止在一次灾难性地
震过程中发生倒塌。这个原理可以融会到一个合理的承载能力设计过程中去。在抗震结构承
载能力设计中要选择耗散能量机构的元件并给以适当的构造处理;而其他结构构件也要具有
足够的承载能力储备,以保证所选的各耗散能量机构在可能出现的整个变形过程能保持接近
于它们的全部强度的状态。为了说明这个承载能力设计法,让我们扼要地讨论一下框架梁的
设计剪力和框架柱的设计荷载的推导过程。

\subsubsection{对梁内剪力的承载能力设计}

若要排除延性的剪切破坏,就需要保证梁的可靠抗剪强度等于或大于与梁在地震扰动过程中
不会被超过的超常抗弯强度相伴而生的剪力。除了地震弯矩引起的剪力之外,还需要考虑到
由重力荷载和垂直加速度所引起的剪力。所以,参考,在左侧支座处我们可以得出

\subsubsection{柱子的承载能力设计}

对抗震框架中的柱子弯矩及与其同时作用的轴向荷载进行估算是更加困难的。在提出与柱子
有关的承载能力设计方法的要点之前,我们必须重申应该满足的设计准则。

在前面的讨论反复强调过,只要有可能,就应该避免在柱中形成塑性铰。除了在几个例子中
已经说明过的与柱型侧移机构相伴随的非常高的曲率延性要求之外,还有一些避免或者推迟
柱中塑性铰的理由。一根柱子的破坏比一根梁的破坏具有严重得多的后果。在一个楼层中所
有的柱子都发生屈服将导致建筑的永久性角度偏差。在柱子中最普遍存在的压力将降低可供
利用的曲率延性。柱中出现塑性铰将伴随着很大的层间位移,这就要要求不稳定问题,从而
又可能反过来危及结构对重力荷载的承受能力。

这里出现的问题是,如果设计者承认在一个较大而随机的动力干扰过程中除去在少数几个无
法避免的部位以外必须防止或推迟在柱中出现塑性铰,将如何提供一个合理的预防程度。

按照承载能力设计的基本原理,那就需要保证梁柱节点近旁的临界柱截面的可靠抗弯能力至
少等于可能与可期轴向荷载同时发生的可期最不利抗弯要求。必须指出,传入柱子的弯矩与
柱抗弯强度之间的关系不必像梁中抗剪情况那样精确,因为柱截面将是按延性要求进行设计
的。

\subsubsection{柱截面的抗弯要求}

由于在一个多层框架的较高振型的反应过程中梁柱节点周围的弯矩分布是不成比例的,所以
在临界截面处就可能得到显著大于由静力分析求出的弯矩。这一情况已经在前面参照图中指
出过了。为了减少柱子屈服的可能性,柱子在临界截面处的可靠强度必须设计成大于从毗邻
的梁传入的可期弯矩。这就是说我们需要

\subsubsection{柱子轴向荷载的确定}

在抗震设计中准确的确定在柱子中由地震引起的可期轴向荷载是很重要的。这些荷载在边柱
中特别具有关键意义。当框架是等效静力横向荷载进行设计的,对应的轴向荷载是很容易得
出的。可是这些轴向力仅对结构的第一振型反应具有代表性;它们并不反应可能在框架中产
生的实际柱子荷载。在新西兰采用的一种确定柱子荷载的方法是根据承载能力设计的基本原
理假定所有与柱子相连接的梁同时沿结构的整个高度达到它们的超常抗弯强度。这意味着在
每层楼盖处传入柱子的荷载是通过对各根梁传来的剪力求和的方法来给定的。这时采用的剪
力公式与右侧部分相似,第一项的正负号是根据在柱子的哪一边来定的。然后把这样得出的
柱子荷载用于柱截面的设计。在柱截面设计中取用的承载能力降低系数为。这个方法所要求
的这种严格程度看来似乎是不必要的,特别是对于高层框架更是如此。在较早时就曾指出过,
在一榀框架作出非弹性动力反应的过程中,梁中的塑性铰是成组出现的,而且一般是在同一
时刻遍及到二到五层,然后再向上传播到框架的全高。因此,在计算地震引起的柱子荷载时,
对所有可能出现的梁中塑性铰并不都是同时存在的这样一个事实给予某种考虑将会是更加合
理的。

现在就以所示的一个20层的结构作为例子来说明这种方法。为了求出在第六层的各根柱子中
引起的柱中荷载的临界横向荷载,可以假定直接位于第六层以上的六层楼盖处所有的梁都发
挥出超强强度。超强强度一般是梁的理想强度的125\%,即。在更上面的六层楼盖梁中就不大
可能已经形成了塑性铰,但是可以假定在这几层楼盖处的每根梁都将发挥出譬如说理想强度
的85\%。对于再上面的六层楼盖可以假定进一步降低到譬如说梁的理想强度的75\%,如所示。

这样求出的柱子荷载还必须与相应的乘了系数的重力荷载以及竖向加速度分量相叠加,从而
得出柱子荷载的上限值和下限值。由于这些轴向荷载都是以极端的和瞬时的受力性能下的承
载能力为依据的,再引入一个承载能力降低系数看是似乎就不必要了。一个与这样求出的轴
向荷载相等的理想截面强度就可能已经足够了。在任何情况下 都会具有一定强度的储备,正
如第一章中所指出的,柱截面的可期强度将会超过理想强度,特别是当较大的轴向压力使混
凝土的抗压强度所引起的作用相当显著的时候。应当记住的是,现浇混凝土的可期强度很可
能大大超过理想强度了。

\subsubsection{作用于柱子的剪力}

由上面所述可以明显看出,必须使柱子具有能够在稍大一些的程度上预防剪切破坏的能力。
这纯粹是因为柱子的剪切破坏是脆性的。按照承载能力设计的基本原则人们可能会倾向于考
虑在柱子的顶面和底面同时出现塑性铰,可是这些截面却已经分别按很可能传入的最大弯矩
设计好了。看来在柱子的每一端都按全部弯矩考虑来确定柱子的设计剪力是不合理的,因为
通常都是有一端的弯矩要比小得多。

\subsubsection{结论}

一个合理的承载能力设计法很可能会在较大的程度上预防脆性破坏和柱子的屈服。在由激烈
的地面运动所引起的不寻常受力行为的情况下,柱子的屈服是可能发生的。可是,这里建议
的方法大概能保证对于柱子的较小的延性要求。还需要进行下一步的研究工作来确定各种强
度参数的可靠数值。特别重要的是根据情况研究或者用其他的方法来估计弯矩分布系数值的
大致范围。这个范围取决于在框架作出动力反应的各个瞬间沿柱高而上的弯矩图形。为了在
地震扰动沿房屋两个主轴方向同时作用时能维持住基本的能量耗散体系(即梁发生屈服),甚
至必须规定柱子具有更大的强度储备。显然,只要沿着两个主方向设置由钢筋混凝土剪力墙
组成的独立的抗横向荷载体系,钢筋混凝土空间框架的很多设计问题,诸如柱子发生屈服、
梁柱节点的细部设计和由偏斜引起的不稳定性等就都是能够避免的了。

\chapter{细部构造技巧}

\section{引言}

在对混凝土结构受力性能的理解方面所取得的最新进展已经导致了更加完善的分析和设计方
法。面向计算机的方法能使高次超静定结构的弹性或非弹性分析得以迅速完成。有人可能认
为,这种发展是能够做出更加符合使用功能要求的和经济的结构的唯一原因。

可是每一个有经验的设计者都知道,设计内容不仅仅包括选择合适的截面或者获得偏于安全
的应力。总体的经济性、可行性和便于施工也都是在一个成功的设计中具有同等重要性的一
些方面。事实上,如果计算不能转换成成功的结构,则煞费苦心的分析就会变得毫无用处。
当用一套细部构造拙劣的图纸来体现一个结构时,情况就会这样。在设计的全部过程中,结
构分析只不过是技术熟练的工程师所采用的许多手段中的一种。它对一个好的设计来说是必
要的前提,但却不是充分的前提。

西部构造设计包括准备好用来加工和铺设结构中的钢筋的布置图、配筋细部构造和钢筋表。
但细部构造设计也包括设计者力求使其结构的每一部分在使用条件下的表现是有效的、在承
受极限荷载或变形时的表现是安全的这样一个整个思考过程。

要对一个混凝土结构正确配筋,设计者就必须对其性能具有透彻的了解,即其了解应超出建
立平衡方程和应变协调方程的范围。这种了解必须是以结构证实了的材料特性和对结构受力
性能的详尽知识为基础的,而不是以由数学模型得出的结果为依据的。以强度和破坏性能的
概念为基础的设计在很大程度上反映了这一基本观点。

还要认识到,如果有待完成的数量庞大的施工作业必须快速而有效的进行,则对钢筋混凝土
的细部构造设计就必须做一定的标准化和简化。如果我们要利用计算机在进行配筋的细部构
造设计、在制表以及在数据处理方面的潜在能力,并把这些结构用适当的方式表示出来,以
便施工人员能把这种资料不加延误和没有困难地变为竣工工程,则这种标准化和简化就是必
不可少的。

抗震方面的细部构造已由ACI315委员会进行了研究并提出报告。可是在取得细部构造设计方
面的令人满意的结论之前还有相当多的模拟地震条件的试验研究工作要做。在以下几节中提
出了以最近的研究成果为依据的有关抗震性能方面的一些建议。

以贵钢筋混凝土结构在范围日益广泛的各种情况下的性能的理解和感受,对经济上不断变化
的要求的了解以及施工实践条件的局限性为依据的细部构造设计很可能也同样需要象以数学
分析来求解结构中的各种作用力那么大的创造力。

在编写必须局限于建筑物框架中最常遇到的少数特定情况的这一章时,很多启发得自莱昂哈
特的著作,他做了大量的工作以引起人们注意在设计过程中曾经被忽视了的这个方面的问题。

\section{配筋的目的}

在研究配筋的细部构造设计之前,把它的目的弄清楚是很有帮助的。在混凝土中可以设置加
劲钢筋来达到以下目的:

\begin{enumerate}
\item 承受由分析得出的内部拉力,在分析中假定周围的混凝土发挥不出任何拉力,因此钢
筋就必须保证一个便于使用的结构具有足够的强度。
\item 保证在使用条件下的裂缝的宽度不超过建议值。这里要记住的是,在实用和经济范围
之内钢筋是不能防止开裂的。
\item 在结构构件受到约束时防止由于收缩或温度变化而可能引起的过分开裂。
\item 在单靠比较适于起抗压作用的混凝土已不能承受内部压力时补充提供压力。
\item 约束受压钢筋使之不产生侧向移动,防止压屈,并在梁、柱和节点受压区的应力较高
部位提供约束。
\item 为型钢构件表面的防火层提供保护,以防止其剥落;也为钢筋骨架在施工过程中提供
临时支承。
\end{enumerate}

在前面几章中已经为了这些目的中的一部分、特别是为了控制裂缝而提出了关于细部构造的
建议。在细部构造技巧方面的一些最重要的指导原则就是那是与粘结和锚固有关的指导原则。

\section{内力方向的改变}

只要是受荷的混凝土结构构件不直或其几何尺寸突然变化,就会引起内力。这些力的存在有
时会被忽视,可是在对这些构件进行细部构造处理时还是必须加以考虑的。

在钢筋弯折处产生的两个拉力的方向是不一致的;因此就引起的第三个力。如果由产生的拉
应力超过了混凝土的抗拉强度,那它就往往可能引起沿钢筋的劈裂裂缝。在角度变化较小的
地方可以用位于弯折处及其附近的箍筋把力传回到构件的受压区域中去。为了防止过早开裂,
把箍筋选配得偏于保守一些是适当的。对于较大的角度变化,钢筋就应以直线延伸段的方式
进行锚固,以是弯折处不致产生横向力。

在内部压力的方向突然改变或持续改变时,可以应用同样的原理。

\section{梁的细部构造设计}

在前面一些章节中我们已经相当详细地讨论了钢筋混凝土构件的抗弯性能及强度。有一点应
该是很明显的,即只有通过混凝土和钢筋有效的相互配合才能达到良好的性能和足够的强度。
精巧的细部构造将能保证这个目标的实现。本节的目的在于通过一些典型情况分析来说明切
实可行的良好的细部构造则可能导致结构性能的改善。在这些建议中有一些已经被混凝土结
构方面的很多设计者或许是仅凭直觉而加以采纳了,另一些则是从近期的研究工作中得出来
的。

\subsection{锚固部位}

从前面对粘结性能的讨论中可以明显看出,在一根被锚固的钢筋周围混凝土中的应力状态必
对粘结质量具有颇大的影响。让钢筋终止在受压区域已经肠胃一种普遍承认的做法。可是在
所有的情况下都这样布置钢筋能否带来多大好处却是值得怀疑的。譬如平行于钢筋作用的压
应力是没有横向裂缝的,而大家都知道这种裂缝是劈裂裂缝的起源。连续梁端跨的底部钢筋
在简支端比在它进入受压区的反弯点附近部位表现出更好的锚固性能。这一情况表明,横向
作用于被锚固钢筋上的压应力是更加有利的。

\end{document}






